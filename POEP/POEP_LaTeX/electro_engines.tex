\chapter{Электродвигатели}

По назначению:
\begin{itemize}
	\item общего назначения;
	\item управляемые.
\end{itemize}

По роду тока:
\begin{itemize}
	\item постоянного тока;
	\item переменного тока;
	\item универсальные.
\end{itemize}

По изменению скорости:
\begin{itemize}
	\item постоянная;
	\item регулируемая;
	\item изменяемая.
\end{itemize}

По характеру работы:
\begin{itemize}
	\item постоянно вращающийся ротор;
	\item дискретно-вращающийся ротор;
	\item поступательно-вращающийся ротор.
\end{itemize}

Применение в ОЭП:
\begin{itemize}
	\item ЭД постоянного тока: приводы механических (растровых) модуляторов, сканирующих оптических систем, блоков позиционирования, переключения элементов, в системах фокусировки;
	\item асинхронные двухфазные ЭД (отсутствие вращения при снятии напряжения): исполнительные элементы фотоэлектрических следящих систем;
	\item ЭД переменного тока: модуляторы и сканирующие системы; асинхронные ЭД обладают меньшей стабильностью частоты вращения, чем синхронные.
\end{itemize}

\begin{flushleft}
	\textbf{Основные характеристики ЭД}
\end{flushleft}
\begin{itemize}
	\item \textit{Механическая характеристика $ \omega = \omega (M) $} --- показывает степень изменения скорости вращения при изменении нагрузки.
Качество механической характеристики оценивается жёсткостью $ \alpha = -\dfrac{\Delta M}{\Delta \omega} $:
	\begin{itemize}
	\item абсолютно жёсткая характеристика $ \alpha = \infty $ --- угловая скорость не зависит от момента нагрузки на валу;
	\item жесткая характеристика (угловая скорость меняется незначительно);
	\item мягкая характеристика (угловая скорость меняется значительно).	
	\end{itemize}

Все ЭД обладают свойством саморегулирования --- двигатель всегда развивает момент, соответствующий моменту нагрузки.

	\item \textit{Кратность пускового момента} --- отношение пускового момента к номинальному $ \dfrac{M_\text{п}}{M_\text{ном}} $ или максимальному $ \dfrac{M_\text{п}}{M_\text{ном}} $.

	\item \textit{Регулировочные характеристики} --- зависимости угловой скорости $ \omega $ от значения (или фазы) напряжения управления $ U_\text{у} $ при постоянных моменте нагрузки на валу и напряжении возбуждения, т.е. $ \omega = \omega (U_\text{у}) $ при $ M_\text{н} = const, \, U_\text{в} = const $.
Регулировочные характеристики необходимы для исполнительных двигателей, работающих в следящих системах. Показателем качества является нелинейность регулировочной характеристики.

	\item \textit{Мощность} --- входная $ P_\text{вх} $, потребляемая обмотками двигателя из питающей сети, и выходная мощность $ P $~-- полезная механическая мощность на валу двигателя.
Номинальная мощность нерегулируемых ЭД --- мощность при номинальном моменте нагрузки или номинальном значении угловой скорости

Номинальная мощность исполнительных ЭД --- мощность при номинальном значении сигнала управления

Мощность управления --- мощность, потребляемая цепями управления.

	\item \textit{Номинальная угловая скорость} $ \omega_\text{ном} $ --- угловая скорость, которую ЭД развивает при номинальном значении момента нагрузки $ M_\text{ном} $.
	\item \textit{Угловая скорость холостого хода} $ \omega_0 $ --- угловая скорость, которую ЭД развивает при отсутствии нагрузки.
	\item \textit{Пусковой момент} $ M_\text{п} $ --- момент при запуске ЭД.
	\item \textit{КПД} ЭД.
	\item \textit{Номинальное значение напряжения питания и частоты питающего тока} $ f $.
	\item \textit{Напряжение трогания} исполнительных двигателей --- напряжение управления, при котором начинается вращение вала ЭД.
	\item \textit{Другие параметры}: электромеханическая постоянная времени двигателя, диапазон регулирования скорости, коэффициенты управления по моменту, скорости, мощности, передаточная функция двигателя.
	\item \textit{Прочие}: масса, габариты, стоимость, момент инерции ротора и т.п.
\end{itemize}



