
\chapter{Передача винт-гайка}
\label{ch:vint-gaika}

Назначение передачи винт-гайка --- преобразование вращательного движения в поступательное.

Передача движения осуществляется с помощью винта, представляющего собой цилиндр с наружной резьбой, и гайки в виде кольца с внутренней резьбой.

Передача винт-гайка подразделяется на: кинематическую и силовую.

Характеристики передачи зависят от типа используемой резьбы. Основные виды резьб, используемые в передаче винт-гайка: метрическая, трапецеидальная и прямоугольная.
Резьба характеризуется:
\begin{itemize}
	\item вид;
	\item шаг $ P $;
	\item число заходов $ z $;
	\item ход винта $ t = z P $;
	\item наружный диаметр $ d $ --- номинальный диаметр;
	\item внутренний диаметр $ d_1 $;
	\item средний диаметр $ d_2 $;
	\item угол подъема резьбы $ \gamma $;
	\item функция перемещения $ l = \dfrac{\varphi t}{2\pi} $.
\end{itemize}

Использование дополнительной гайки на винте позволяет реализовать дифференциальную и интегральную схему перемещений, которые позволяют увеличить точность и чувствительность к повороту винта соответственно.
Функция перемещения:
\begin{itemize}
	\item дифференциальная $ l=\dfrac{\varphi (p_1 - p_2)}{2\pi} $;
	\item интегральная $ l=\dfrac{\varphi (p_1 - p_2)}{2\pi} $.
\end{itemize}


В передаче винт-гайка\marginnote{\allcaps{КИНЕМАТИЧЕСКИЕ И\break СИЛОВЫЕ\break СООТНОШЕНИЯ}} винт в большинстве случаев является ведущим.

При ведущем винте:
\begin{itemize}
	\item $ i = \dfrac{1}{tg \gamma}$ -- передаточное отношение;
	\item $ F = F_a tg(\gamma + \rho') $ -- окружное усилие, которое приложено по касательной к окружности среднего диаметра $ d_2 $;
	\item $ M_k = F_a d_2 tg(\gamma + \rho')$;
	\item $ \eta_\text{в} = \dfrac{tg \gamma}{tg(\gamma + \rho')} $ -- КПД.
\end{itemize}

При ведущей гайке:
\begin{itemize}
	\item $ F = F_a tg(\gamma - \rho') $;
	\item $ M_k = \dfrac{F_a d_2 tg(\gamma - \rho')}{2}$;
	\item $ \eta_\text{г} = \dfrac{tg(\gamma - \rho')}{tg \gamma} $ -- КПД.
\end{itemize}

\newthought{Достоинства}:
\begin{itemize}
	\item большой выигрыш в силе;
	\item высокая точность перемещений;
	\item малые размеры;
	\item возможность обеспечения самоторможения;
	\item сравнительно высокий КПД;
	\item высокая жесткость
	\item малый износ в сравнении с передачами скольжения.
\end{itemize}

\newthought{Недостатки}:
\begin{itemize}
	\item низкий КПД в передачах скольжения;
	\item невозможность получения больших скоростей поступательного движения;
	\item сложность и дороговизна изготовления.
\end{itemize}