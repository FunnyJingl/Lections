
\chapter{Расчёт электромеханического привода}
\label{ch:EMP}

Порядок расчёта:
\begin{enumerate}
	\item Выбор электродвигателя
	\begin{enumerate}
		\item Возможные для применения типы двигателей.
		\item Выбор конкретного двигателя для ЭМП.
	\end{enumerate}
	\item Разработка кинематических схем механизмов.
	\item Силовой расчёт.
	\item Расчёт на прочность.
	\item Геометрический расчёт и конструирование ЭМП.
\end{enumerate}

\section{Выбор электродвигателя}

\newthought{Механическая характеристика}\marginnote{\allcaps{МЕХАНИЧЕСКАЯ\break ХАРАКТЕРИСТИКА}} $ \omega = \omega (M) $ --- показывает степень изменения скорости вращения при изменении нагрузки.
Качество механической характеристики оценивается её жёсткостью $ \alpha = - \dfrac{\Delta M}{\Delta \omega} $:
\begin{enumerate}
	\item абсолютно жесткая характеристика ($ \alpha = \infty $~-- угловая скорость не зависит от момента нагрузки на валу);
	\item жёсткая характеристика (угловая скорость меняется незначительно);
	\item мягкая характеристика (угловая скорость меняется значительно).
\end{enumerate}

\noindent
Каждый ЭД обладает свойством \textsc{саморегулирования}\marginnote{\allcaps{САМОРЕГУЛИРОВАНИЕ}} --- ЭД всегда развивает момент, соответствующий моменту нагрузки.

\newthought{Регулировочные характеристики}\marginnote{\allcaps{РЕГУЛИРОВОЧНЫЕ\break ХАРАКТЕРИСТИКИ}}~--- зависимости угловой скорости $ \omega $ от значения (или фазы) напряжения управления $ U_\text{у} $ при постоянных моменте нагрузки на валу и напряжении возбуждения, т.е. $ \omega = \omega (U_\text{у}) $ при $ M_\text{н} = const $, $ U_\text{в} = const $. Регулировочные характеристики необходимы для исполнительных двигателей, работающих в следящих системах. Показатель качества регулировочной характеристики -- её нелинейность.

\newthought{Мощность}\marginnote{\allcaps{МОЩНОСТЬ}} ЭД:
\begin{itemize}
	\item входная $ P_\text{вх} $ --- мощность, потребляемая обмотками двигателя из питающей сети;
	\item выходная $ P $ --- полезная механическая мощность на валу ЭД;
	\item номинальная мощность нерегулируемых ЭД --- мощность при номинальном моменте нагрузки или номинальном значении угловой скорости;
	\item номинальная мощность исполнительных ЭД --- мощность при номинальном значении сигнала управления;
	\item мощность управления --- мощность, потребляемая цепями управления.
\end{itemize}

\noindent
\textsc{Номинальная угловая скорость}\marginnote{\allcaps{НОМИНАЛЬНАЯ УГЛОВАЯ СКОРОСТЬ}} $ \omega_\text{ном} $~--- угловая скорость, которую ЭД развивает при номинальном значении момента нагрузки $ M_\text{ном} $.

\noindent
\textsc{Угловая скорость холостого хода}\marginnote{\allcaps{УГЛОВАЯ СКОРОСТЬ\break ХОЛОСТОГО ХОДА}} $ \omega_0 $~--- угловая скорость, которую ЭД развивает при отсутствии нагрузки.

\noindent
\textsc{Пусковой момент}\marginnote{\allcaps{ПУСКОВОЙ МОМЕНТ}} $ M_\text{п} $~--- момент, который ЭД во время пуска.

Другими важными параметрами, которые необходимо учитывать при выборе ЭД, являются:
\begin{itemize}
	\item КПД;
	\item номинальное значение напряжения питания и частоты питающего тока $ f $;
	\item напряжение трогания исполнительных двигателей~--- напряжение управления, при котором начинается вращение вала ЭД;
	\item электромеханическая постоянная времени двигателя;
	\item диапазон регулирования скорости;
	\item коэффициенты управления по моменту, скорости, мощности;
	\item передаточная функция двигателя;
	\item прочие: масса, габариты, стоимость, момент инерции ротора и т.п.
\end{itemize}

\section{Выбор схемотехнического состава ЭМП}
Требуемое передаточное отношение кинематических цепей $ i_0 $ может быть реализовано с помощью разных схемотехнических элементов.
Эффективность разрабатываемого ЭМП зависит от того, насколько рационально выбраны схемотехнические элементы.

При выборе схемы и схемотехнического состава ЭМП учитывают: 
\begin{itemize}
	\item закон, вид (вращательное, поступательное, сложное) и характер (непрерывное, реверсивное, с остановками) движения выходного звена;
	\item общие передаточные отношения цепей ЭМП;
	\item параметры нагрузки;
	\item требуемая точность;
	\item заданная компоновочная схема ЭМП;
	\item условия эксплуатации и долговечности;
	\item технологичность;
	\item экономические факторы.
\end{itemize}

Так как система выбора схемы и схемных элементов пока не разработана, то разработчик анализирует исходные данные и определяет те, которые являются наиболее критичными для рассматриваемого задания и ранжирует их по степени важности.
Затем с помощью метода перебора известных элементарных передач определяются необходимые типы передач и схемных элементов ЭМП.
Если возникает неоднозначность в выборе схемных элементов, то следует проанализировать целесообразность их выбора с учетом дополнительных требований (например, стоимости, технологичности конструкции, КПД, точности). 
Для намеченных схемных элементов следует назначить передаточное отношение.

После выбора и назначения передаточных отношений схемных элементов определяют произведение передаточных отношений этих элементов $ \prod{i_i} $ и сравнивают его с передаточным отношением рассматриваемой цепи $ i_0 $ и в случае необходимости добавляют (уменьшают) некоторое число схемных элементов или увеличивают (уменьшают) ранее назначенные передаточные отношения элементарных передач, исходя при этом из необходимости выполнения условия:
\begin{equation*}
i_0 = \prod_{i=1}^{n}i_i.
\end{equation*}

Отдельные ступени механизма должны работать так же, как и весь механизм в целом -- на замедление или ускорение, что обеспечивает меньшее число ступеней, а следовательно, меньшие габариты и мёртвый ход привода.

При выборе схемных элементов решается задача рационального размещения их в кинематической цепи, при этом учитывают требования компоновки привода, точности, КПД, габаритов, долговечности и т.д. 
В зубчато-червячном ЭМП размещение червячной передачи в начале кинематической цепи нецелесообразно из-за её более низкого КПД по сравнению с зубчатыми передачами. 
В редукторах элементы с большими передаточными отношениями размещают в конце кинематической цепи. При разработке ЭМП с несколькими выходными валами с целью уменьшения числа схемных элементов и упрощения конструкции рекомендуется частичное или полное совмещение кинематических цепей. Для ряда ПУ необходим учёт направления перемещения выходного вала.