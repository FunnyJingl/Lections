\chapter{Червячная передача}
\label{ch:chervak}

Назначение червячной передачи --- передача вращательного движения между скрещивающимися осями вращения.

Червячная передача состоит из червяка и червячного колеса. Червяк --- одно- или многовитковый винт, боковые поверхности витков которого являются винтовыми. Червячное колесо~--- косозубое зубчатое колесо, угол наклона зубьев которого равен углу подъема витков червяка.

\newthought{Основные параметры}\marginnote{\allcaps{ОСНОВНЫЕ ПАРАМЕТРЫ}} червячных передач и её элементов:
\begin{table}[ht]
	\centering
	\fontfamily{ppl}\selectfont
	\begin{tabular}{ll}
		\toprule
		Обозначение & Параметр \\ 
		\midrule
		$ m = \dfrac{p}{\pi}$ & осевой модуль ГОСТ 2144-76, [мм] \\
		$ q= \dfrac{d_1}{m} $ & коэффициент диаметра червяка СТ СЭВ 267-76 \\
		$ \alpha = 20^\circ $ & стандартный угол профиля \\
		$ z_1 $ & число заходов червяка (1, 2 или 4) \\
		$ s $ & ход витка червяка \\
		$ p $ & делительный шаг червяка \\
		$ \gamma = arctg(\dfrac{z_1}{q}) $ & угол подъема линии червяка \\
		$ d_{a1,2} = d_{1,2} + 2m $ & диаметр окружности вершин, [мм] \\
		$ d_{f1} = m(q-2,4) $ & диаметр впадин, [мм] \\
		$ b_1 = 2m\sqrt{z_2} + 1 $ & длина нарезамеого червяка, [мм] \\
		$ a = 0,5m(q+z_2) $ & межосеовое расстояние\\
		\bottomrule
	\end{tabular}
	\label{tab:parChervak}
\end{table}

\newthought{Свойство самоторможения передачи}\marginnote{\allcaps{САМОТОРМОЖЕНИЕ}}~--- свойство, при котором червячное колесо при отсутствии вращения червяка ведомый вал затормаживается, таким образом его невозможно повернуть. самоторможение начинается проявляться при передаточном отношении 35. 

Самоторможение: статическое и динамическое. Статическое самоторможение может быть нейтрализовано ударными нагрузками. Динамическое самоторможение оценивается временем торможения привода после отключения питания двигателя. Полное самоторможение при $ \gamma < 3,5^\circ $. Свойство самоторможения в случае отсутствия ударных нагрузок может быть использовано в качестве тормозящего устройства.

\newthought{Достоинства}:
\begin{itemize}
	\item большое передаточное отношение (от 7 до 200 (теор. 500));
	\item малые габариты;
	\item эффект самоторможения ведомого червячного колеса;
	\item плавность хода;
	\item бесшумность работы.
\end{itemize}

\newthought{Недостатки}:
\begin{itemize}
	\item меньший по сравнению с зубчатыми КПД $ \eta=0,6\ldots0,9 $;
	\item необходимость применения для выполнения колёс дорогих антифрикционных материалов (бронз);
	\item повышенные требования к точности изготовления и монтажа;
	\item значительные осевые силы, действующие на опоры червяка и усложняющие конструкцию опор.
\end{itemize}