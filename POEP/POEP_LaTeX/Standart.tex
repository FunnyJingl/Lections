\chapter{Стандартизация}
\section{Основные понятия и определения}
В настоящее время в этой области действует <<ФЗ от 27.12.2002 №184-ФЗ (ред. от 23.06.2014 с изменениями и доп., вступившими в силу с 22.12.2014) <<О ТЕХНИЧЕСКОМ РЕГУЛИРОВАНИИ>>

\textit{Международная организация по стандартизации ИСО (ISO)} --- организация по стандартизации, членство в которой открыто для соответствующего национального органа каждой страны.

\textit{Стандартизация} --- по определению ИСО~--- это установление и применение правил с целью упорядочения деятельности в определенных областях на пользу и при участии всех заинтересованных сторон.

\textit{Стандартизация} --- деятельность, направленная на достижение оптимальной степени упорядочения в определенной области, посредством установления положений для всеобщего и многократного использования в отношении реально существующих или потенциальных задач.

Стандартизация является одним из инструментов обеспечения качества продукции, работ и услуг --- важного аспекта многогранной коммерческой деятельности.

Влияние стандартизации на улучшение качества продукции осуществляется через комплексную разработку стандартов на сырье, материалы, полуфабрикаты, комплектующие изделия, оборудование, оснастку и готовую продукцию, а также через установление в стандартах технологических требований и показателей качества, единых методов испытаний и средств контроля.

\textit{Нормы} --- технические требования к изделиям, правила их изготовления и проверки, маркировки и упаковки, хранения и транспортировки, утилизации.

Документы, содержащие нормы, называются нормативными документами или стандартами. Т.е. \textit{стандартом} называется документ, в котором в целях добровольного многократного использования устанавливаются характеристики продукции, правила осуществления и характеристики процессов производства, эксплуатации, хранения, перевозки, реализации и утилизации, выполнения работ или оказания услуг. Стандарт также может содержать требования к терминологии, символике, упаковке, маркировке или этикеткам и правилам их нанесения.

Определение стандарт --- нормативный документ, действует во многих странах.

\textit{Стандарт} --- документ, разработанный на основе консенсуса и утвержденный признанным органом, в котором устанавливаются для всеобщего и многократного использования правила, общие принципы и характеристики, касающиеся различных видов деятельности или их результатов, и который направлен на достижение оптимальной степени упорядочения в определенной области.

\textit{Консенсус} --- общее согласие, характеризующееся отсутствием серьезных возражений по существенным вопросам у большинства заинтересованных сторон и достигаемое в результате процедуры, стремящейся учесть мнения всех сторон и сблизить несовпадающие точки зрения. Консенсус не обязательно предполагает полное единодушие (из проекта <<ФЗ <<О стандартизации в Российской Федерации>>).

Стандарт разрабатывается на материальные предметы (продукцию, эталоны, образцы веществ), нормы, правила и требования различного характера, т.е. объектом стандартизации обычно называют продукцию, процесс или услугу, для которых разрабатывают те или иные требования, характеристики, параметры, правила стандартизация может касаться либо объекта в целом, либо его отдельных составляющих (характеристик).

\textit{Областью стандартизации} называют совокупность взаимосвязанных объектов стандартизации.

\section{Цели и задачи стандартизации}

Цель стандартизации --- выявление наиболее правильного и экономичного варианта, т.е. нахождение оптимального решения. Найденное решение дает возможность достичь оптимального упорядочения в определенной области стандартизации. Для превращения этой возможности в действительность необходимо, чтобы найденное решение стало достоянием большего числа предприятий (организаций) и специалистов. Только при всеобщем и многократном использовании этого решения существующих и потенциальных задач возможен экономический эффект от проведенного упорядочения.

Цели стандартизации можно подразделить на общие и более узкие, касающиеся обеспечения соответствия. Общие цели вытекают, прежде всего, из содержания понятия. Конкретизация общих целей для российской стандартизации связана с выполнением тех требований стандартов, которые являются обязательными. Конкретные цели стандартизации относятся к определенной области деятельности, отрасли производства товаров и услуг, тому или другому виду продукции, предприятию, терминологии.

Целями стандартизации являются:
\begin{itemize}
\item обеспечение качества и безопасности продукции, процессов, работ и услуг для жизни и здоровья людей, окружающей среды и имущества;
\item обеспечение совместимости и взаимозаменяемости продукции;
\item обеспечение единства и сопоставимости результатов измерений;
\item рациональное использование ресурсов, в том числе повышение энергоэффективности и снижение энергопотребления;
\item рациональное сокращение неоправданного многообразия;
\item повышение конкурентоспособности продукции, работ, услуг;
\item содействие научно-техническому прогрессу и модернизации экономики РФ;
\item обеспечение устойчивого развития экономики РФ;
\item упрощение разработки нормативных правовых актов РФ и нормативных документов федеральных органов исполнительной власти за счет применения ссылок в них на национальные стандарты;
\item обеспечение охраны окружающей среды;
\item обеспечение охраны здоровья;
\item обеспечение охраны труда;
\item обеспечение безопасности хозяйственных объектов, связанной с возможностью возникновения различных катастроф (природного и техногенного характера) и чрезвычайных ситуаций;
\item содействие обеспечению национальной безопасности РФ;
\item содействие обеспечению обороноспособности и мобилизационной готовности РФ.
\end{itemize}

Основными задачами стандартизации являются:
\begin{itemize}
\item установление требований к техническому уровню и качеству продукции, сырья, материалов, полуфабрикатов и комплектующих изделий, а также норм, требований и методов в области проектирования и производства продукции, позволяющих ускорять внедрение прогрессивных методов производства продукции высокого качества и ликвидировать нерациональное многообразие видов, марок и размеров;
\item развитие унификации и агрегатирования промышленной продукции как важнейшего условия специализации производства; комплексной механизации и автоматизации производственных процессов, повышение уровня взаимозаменяемости, эффективности эксплуатации и ремонта изделий;
\item обеспечение единства и достоверности измерений в стране, создание и совершенствование государственных эталонов единиц физических величин, также методов и средств измерений высшей точности;
\item разработка унифицированных систем документации, систем классификации и кодирования технико-экономической информации;
\item принятие единых терминов и обозначений в важнейших областях науки, техники, отраслях народного хозяйства;
\item формирование системы стандартов безопасности труда, систем стандартов в области охраны природы и улучшения использования природных ресурсов;
\item создание благоприятных условий для внешнеторговых, культурных и научно-технических связей.
\end{itemize}

\section{Функции стандартизации}

Для достижения социальных и технико-экономических целей стандартизация выполняет определенные функции.
\begin{enumerate}
\item Функция упорядочения --- преодоление неразумного многообразия объектов (раздутая номенклатура продукции, ненужное многообразие документов). Она сводится к упрощению и ограничению. Житейский опыт говорит: чем объект более упорядочен, тем он лучше вписывается в окружающую предметную и природную среду с ее требованиями и законами.
\item Охранная (социальная) функция --- обеспечение безопасности потребителей продукции и услуг, изготовителей и государства, объединение усилий человечества по защите природы от техногенного воздействия цивилизации.
\item Ресурсосберегающая функция обусловлена ограниченностью материальных, энергетических, трудовых и природных ресурсов и заключается в установлении в нормативных документах обоснованных ограничений на расходование ресурсов.
\item Коммуникативная функция обеспечивает общение и взаимодействие людей, в частности специалистов, путем личного обмена или использования документальных средств, аппаратных (компьютерных, спутниковых и пр.) систем и каналов передачи сообщений. Эта функция направлена на преодоление барьеров в торговле и содействие научно-техническому и экономическому сотрудничеству.
\item Цивилизующая функция направлена на повышение качества продукции и услуг как составляющей качества жизни. Стандарты отражают степень общественного развития страны, т.е. уровень цивилизации.
\item Информационная функция. Стандартизация обеспечивает материальное производство, науку и технику и другие сферы нормативными документами (НД), эталонами мер, образцами – эталонами продукции, каталогами продукции как носителями ценной творческой и управленческой информации. Ссылка в договоре (контракте) на стандарт является наиболее удобной формой информации о качестве товара как главного условия договора (контракта).
\item Функция нормотворчества и правоприменения проявляется в узаконивании требований к объектам стандартизации в форме обязательного стандарта (или другого НД) и его всеобщем применении в результате придания документу юридической силы. Соблюдение обязательных требований НД обеспечивается, как правило, принудительными мерами (санкциями) экономического, административного и уголовного характера.
\end{enumerate}

\section{Принципы и методы стандартизации}
Стандартизация как вид деятельности базируется на определенных исходных положениях~--- принципах, задающих вектор ее развития и смысл существования. 

\textit{Принципы стандартизации} отражают основные закономерности процесса разработки стандартов, обосновывают ее необходимость в управлении бизнесом, народным хозяйством, отношениями в обществе, определяют условия эффективной реализации и тенденции развития. Можно выделить следующие важнейшие принципы стандартизации. 

\begin{flushleft}
\textbf{Основные принципы стандартизации}
\end{flushleft}

\begin{enumerate}
\item Соответствие документов по стандартизации законодательству Российской Федерации. Содействие реализации законодательства Российской Федерации.
\item Добровольное применение стандартов и обеспечение условий для их единообразного применения. Национальный стандарт применяется на добровольной основе равным образом и в равной мере независимо от страны и (или) места происхождения продукции, осуществления процессов жизненного цикла продукции (ЖЦП), выполнения работ и оказания услуг, видов или особенностей сделок и лиц (являющихся изготовителями, исполнителями, продавцами, приобретателями).
\item Применение международного стандарта как основы разработки национального стандарта. Исключение могут составить случаи, когда: соответствие требованиям международных стандартов невозможно вследствие несоответствия их требований климатическим и географическим особенностям РФ или техническим (технологическим) особенностям отечественного производства; Россия выступает против международного стандарта в рамках процедуры голосования в международной организации по стандартизации.
\item Сбалансированность интересов сторон, разрабатывающих, изготавливающих, предоставляющих и потребляющих продукцию (услугу). Иначе говоря, необходим максимальный учет законных интересов перечисленных сторон. Участники работ по стандартизации, исходя из возможностей изготовителя продукции и исполнителя услуги, с одной стороны, и требований потребителя — с другой, должны найти консенсус, который понимается как общее согласие, т.е. как отсутствие возражений по существенным вопросам у большинства заинтересованных сторон, стремление учесть мнение всех сторон и сблизить несовпадающие точки зрения. Консенсус не предполагает полного единодушия.
\item Системность стандартизации. Системность — это рассмотрение каждого объекта как части более сложной системы. Например, бутылка как потребительская тара входит частью в транспортную тару — ящик, последний укладывается в контейнер, а контейнер помещается в транспортное средство. Системность предполагает совместимость всех элементов сложной системы,  подвергнутой стандартизации.
\item Динамичность и опережающее развитие стандарта. Как известно, стандарты моделируют реально существующие закономерности в хозяйстве страны. Однако научно-технический прогресс вносит изменения в технику, в процессы управления. Поэтому стандарты должны адаптироваться к происходящим переменам. Динамичность обеспечивается периодической проверкой стандартов, внесением в них изменений, отменой НД. Для того чтобы вновь создаваемый стандарт был меньше подвержен моральному старению, он должен опережать развитие общества. Опережающее развитие обеспечивается внесением в стандарт перспективных требований к номенклатуре продукции, показателям качества, методам контроля и пр. Опережающее развитие также обеспечивается путем учета на этапе разработки НД международных и региональных стандартов, прогрессивных национальных стандартов других стран.
\item Недопустимость создания препятствий производству и обращению продукции, выполнению работ и оказанию услуг в большей степени, чем это минимально необходимо для выполнения целей стандартизации. Руководствуясь принципом опережающей стандартизации при формировании уровня требований национального стандарта или технического регламента, следует учитывать готовность страны, организаций к выполнению повышенных требований. В противном случае введение нового документа может парализовать деятельность значительной части организаций.
\item Эффективность стандартизации. Применение НД должно давать экономический или социальный эффект. Непосредственный экономический эффект дают стандарты, ведущие к экономии ресурсов, повышению надежности, технической и информационной совместимости. Стандарты, направленные на обеспечение безопасности жизни и здоровья людей, окружающей среды, обеспечивают социальный эффект.
\item Принцип гармонизации. Этот принцип предусматривает разработку гармонизированных стандартов и недопустимость установления таких стандартов, которые противоречат техническим регламентам. Обеспечение идентичности документов, относящихся к одному и тому же объекту, но принятых как организациями по стандартизации в нашей стране, так и международными (региональными) организациями, позволяет разработать стандарты, которые не создают препятствий в международной торговле.
\item Четкость формулировок положений стандарта. Возможность двусмысленного толкования нормы свидетельствует о серьезном дефекте НД.
\item Комплексность стандартизации взаимосвязанных объектов. Качество готовых изделий определяется качеством сырья, материалов, полуфабрикатов и комплектующих изделий. Поэтому стандартизация готовой продукции должна быть увязана со стандартизацией объектов, формирующих ее качество. Комплексность стандартизации предусматривает увязку стандартов на готовые изделия со стандартами на сборочные единицы, детали, полуфабрикаты, материалы, сырье, а также технические средства, методы организации производства и способы контроля.
\item Объективность проверки требований. Стандарты должны устанавливать требования к основным свойствам объекта стандартизации, которые могут быть объективно проверены, включая требования, обеспечивающие безопасность для жизни, здоровья и имущества, окружающей среды, совместимость и взаимозаменяемость. Объективная проверка требований к продукции осуществляется, как правило, техническими средствами измерения (приборами, методами химического анализа). Объективная проверка требований к услугам может осуществляться также с помощью социологических и экспертных методов. В качестве объективного доказательства используются сертификаты соответствия, заключения надзорных органов.
\item Обеспечение условий для единообразного применения стандартов. Например, указанный принцип следует учитывать при разработке стандартов организаций. Хотя порядок разработки, утверждения, учета изменения и отмены стандартов организаций устанавливается ими самостоятельно, он должен учитывать: во-первых, принципы стандартизации; во-вторых, универсальные правила, действующие в отношении стандартов любого статуса в части правил построения, изложения, оформления стандартов.
\item Открытость процессов разработки документов по стандартизации и соблюдения прав интеллектуальной собственности.
\item Доступность информации о документах по стандартизации. Данный принцип предусматривает следующее: стандартизация в России устанавливает обязательства Госстандарта России по опубликованию государственных стандартов в сети на безвозмездной основе.
\end{enumerate}

\begin{flushleft}
\textbf{Основные методы стандартизации}
\end{flushleft}

\textit{Метод стандартизации}~--- это совокупность средств достижения целей стандартизации.

\begin{enumerate}
\item Упорядочение объектов стандартизации является универсальным методом стандартизации товаров, работ и услуг. Данный метод систематизирует разнообразие продукции. Результатом применения этого метода являются перечни изделий, описания типовых конструкций, образцы форм различной документации. Упорядочение включает в себя систематизацию, селекцию, симплификацию, типизацию и оптимизацию. 
\begin{itemize}
\item Систематизация объектов стандартизации представляет собой последовательное, научно обоснованное классифицирование и ранжирование конкретных объектов стандартизации. Примерами систематизации являются различные виды общероссийских классификаторов.
\item Селекция объектов стандартизации --- это отбор целесообразных для дальнейшего производства и применения объектов стандартизации.
\item Симплификация --- деятельность, выявляющая объекты стандартизации, которые нецелесообразно применять для производства. Симплификация ограничивает перечень применяемых в производстве изделий до оптимального, удовлетворяющего потребности количества.
\item Типизация объектов стандартизации --- это разработка и утверждение типовых объектов или образцов. Типизируют конструкции, технологические нормы и правила документации. Типизация проводится с целью выделения общего признака для совокупности однородных объектов.
\item Оптимизация объектов стандартизации заключается в нахождении оптимальных главных параметров (параметров назначения), а также значений всех других показателей качества и экономичности, необходимых для данного уровня качества. Целью оптимизации является достижение оптимальной степени упорядочения и максимально возможной эффективности по выбранному критерию. В отличие от работ по селекции и симплификации, базирующихся на несложных методах оценки и обоснования принимаемых решений, например, экспертных методах, оптимизацию объектов стандартизации осуществляют путем применения специальных экономико-математических методов и моделей оптимизации.
\end{itemize}
\item  Параметрическая стандартизация. \textit{Параметр продукции}~--- это количественная характеристика ее свойств. Наиболее важными параметрами являются характеристики, определяющие назначение продукции и условия ее использования:
\begin{itemize}
\item размерные параметры (например, размер одежды и обуви, вместимость посуды);
\item весовые параметры (масса отдельных видов спортинвентаря);
\item параметры, характеризующие производительность машин и приборов (скорость движения транспортных средств);
\item энергетические параметры (мощность двигателя).
\end{itemize}

Анализируя параметры, выделяют главные и основные параметры изделий.

Главным называют параметр, который определяет важнейший эксплуатационный показатель изделия. Главный параметр не зависит от технических усовершенствований изделия и технологии изготовления, он определяет показатель прямого назначения изделия. Например, главными параметрами являются: для редуктора~-- передаточное отношение, электродвигателя~-- мощность, средства измерений~-- диапазон измерения.

Продукция определенного назначения, принципа действия и конструкции, т.е. продукция определенного типа, характеризуется рядом параметров. Набор установленных значений параметров называется параметрическим рядом. Главный параметр принимают за основу при построении параметрического ряда. Выбор главного параметра и определение диапазона значений этого параметра должны быть технически и экономически обоснованы, крайние числовые значения ряда выбирают с учетом текущей и перспективной потребности в данных изделиях, для чего проводятся маркетинговые исследования.

Параметрическим рядом является закономерно построенная в определенном диапазоне совокупность числовых значений главного параметра изделия одного функционального назначения и принципа действия. Главный параметр служит базой при определении числовых значений основных параметров, поскольку выражает самое важное эксплуатационное свойство. 

Основными называют параметры, которые определяют качество изделия как совокупности свойств и показателей, определяющих соответствие изделия своему назначению. Например, для металлорежущего оборудования за основные можно принять: точность обработки, мощность, число оборотов шпинделя, производительность. 

Для измерительных приборов основными параметрами являются: погрешность измерения, цена деления шкалы, измерительное усилие. 
Основные и главный параметры взаимосвязаны, поэтому иногда удобно выражать основные параметры через главный параметр. Например, главным параметром поршневого компрессора является диаметр цилиндра, а одним из основных~--- производительность, которые связаны между собой определенной зависимостью. 

Разновидностью параметрического ряда является размерный ряд. 

Параметрический ряд называют типоразмерным или просто размерным рядом, если его главный параметр относится к геометрическим размерам изделия. На базе типоразмерных параметрических рядов разрабатываются конструктивные ряды конкретных типов или моделей изделий одинаковой конструкции и одного функционального назначения.

Например, для тканей размерный ряд состоит из отдельных значений ширины тканей, для посуды -- отдельных значений вместимости. Каждый размер изделия (или материала) одного типа называется типоразмером. Например, сейчас установлено 105 типоразмеров мужской одежды и 120 типоразмеров женской одежды.

Параметрические, типоразмерные и конструктивные ряды машин строятся исходя из пропорционального изменения их эксплуатационных показателей (мощности, производительности, тяговой силы) с учётом теории подобия. В этом случае геометрические характеристики машин (рабочий объем, диаметр цилиндра, диаметр колеса у роторных машин) являются производными от эксплуатационных показателей и в пределах ряда машин могут изменяться по закономерностям, отличным от закономерностей изменения эксплуатационных показателей. 

Параметрическая стандартизация --- процесс стандартизации параметрических рядов --- заключается в выборе и обосновании целесообразной номенклатуры и численного значения параметров. Решается эта задача с помощью математических методов. (При создании, например, размерных рядов одежды и обуви производятся антропометрические измерения большого числа мужчин и женщин разных возрастов, проживающих в различных районах страны. Полученные данные обрабатывают методами математической статистики).

Параметрические ряды машин, приборов, тары рекомендуется строить согласно системе предпочтительных чисел - набору последовательных чисел, изменяющихся в геометрической прогрессии. Смысл этой системы заключается в выборе лишь тех значений параметров, которые подчиняются строго определенной математической закономерности, а не любых значений, принимаемых в результате расчетов или в порядке волевого решения. 

Основным стандартом в этой области является ГОСТ~8032 <<Предпочтительные числа и ряды предпочтительных чисел>>. На базе этого стандарта утвержден ГОСТ~6636 <<Нормальные линейные размеры>>, устанавливающий ряды чисел для выбора линейных размеров. Ряды номиналов резисторов и конденсаторов (E3, E6, E12, E24, E48, E96, E192, $ \ldots $), токов (0,0001; 0,001; 0,01; 0,1; 1; 10,0; $ \ldots $ А) и  напряжений (6, 12; 28,5; 42; 62; 115; 120; $ \ldots $ В), ряд номинальных частот источников электрической энергии (0,1; 0,25; 0,5; 1,0; 2,5; 5,0; 10; 25; 50; 400; 1000; 10000 Гц).

\item Унификация --- наиболее распространенный и эффективный метод стандартизации, который предусматривает приведение объектов к однотипности на основе установления рационального числа их разновидностей.

Что касается техники, то понятие унификации определяется согласно ГОСТ~23945.0-80 следующим образом: <<Унификация изделий~--- приведение изделий к единообразию на основе установления рационального числа их разновидностей>>.

\textit{Унификацией} называют приведение к оптимальному единообразию форм и объектов человеческой деятельности. Это понятие универсально и относится к любым организационным, научным, проектно-конструкторским, технологическим, экономическим, общественно-социальным и другим формам деятельности и их результатам (изделиям, постройкам, деталям, материалам, технологическим процессам, методам исследований и расчетов, формам представления результатов, законам, правилам, порядку их принятия.

Суть унификации конструкций изделий заключается в ограничении многообразия возможных частных (индивидуальных) решений на всех этапах проектно-конструкторской деятельности рамками общих свойств и признаков, приводящих изделие к единой системе типовых конструкций.
Унификация даёт возможность снизить стоимость производства новых изделий, повысить серийность и уровень автоматизации производственных процессов.

Различают следующие методы унификации конструкций: \textit{индивидуальный}, \textit{базовый} и \textit{агрегатно-модульный}, которые фактически являются соответствующими методами проектирования.

\textit{Индивидуальный метод унификации} (его называют также методом заимствования, моноблочным, пассивным методом) основывается на использовании в конструкции ранее созданных (заимствованных) решений, нормализованных и типовых устройств, элементов, деталей, а также на соответствии конструктивных решений требованиям и рекомендациям существующих стандартов, норм и условий (Например, соответствие размеров и форм конструируемых оригинальных деталей ряду предпочтительных чисел, углов, уклонов, конусностей, параметрам типовых форм. Фаски, канавки, центровые отверстия, рифления, шероховатость, допуски на погрешности размеров и форм, характеристики материалов деталей, их покрытия, термообработка должны соответствовать типовым рекомендациям и стандартам).

В конструкции желательно создавать как можно больше одинаковых узлов, деталей или их элементов (одинаковых материалов, размеров, форм поверхностей), что облегчает использование метода групповой технологии производства, существенно повышающей эффективность изготовления изделий, особенно для условий их мелкосерийного производства.

Соединительные и информационно-энергетические устройства создаваемой конструкции должны обеспечить ее совместимость с другими изделиями.
Этот метод унификации используется обычно при создании конструкций индивидуальных (оригинальных) и уникальных приборов и их составных частей, изготавливаемых в единичном, опытном или мелкосерийном производствах для решения частных технических задач, а также для улучшения тех или иных характеристик существующих прототипов (аналогов). Примерами могут служить уникальные образцы техники [космические корабли, телескопы; специализированное технологическое оборудование, оснастка, контрольно-юстировочные приспособления; модернизированные серийные приборы, имеющие более высокие сравнительные характеристики (по мощности, точности, надежности, производительности, габаритно-массовые)].

Разработка пионерских (принципиально новых) приборов, основанных на новых физических принципах, изменении схем и конструкций, использовании новых элементов, материалов, также требует применения индивидуального метода унификации. Причем для перспективных приборов (таких, например, какими были в свое время пионерские образцы персональных компьютеров, видеокамер и магнитофонов, копировальные аппараты, факсы и т.п.) должна проводиться так называемая опережающая унификация, основанная на научно-техническом прогнозировании тенденций будущего развития этих приборов, их составных частей, изменений, технико-экономических характеристик, методов производства, обслуживания, ремонта.

Индивидуальный метод унификации используется также при разработке моноблочных изделий простой конструкции, не требующей разбивки ее на функциональные блоки и узлы, а также сверхминиатюрных, не позволяющих производить их из блоков (стимуляторы сердца, медицинские зонды).

\textit{Базовый метод унификации} является активной формой унификации и заключается в создании модификаций или унифицированного ряда изделий на основе конструкции базового изделия.

В модификациях или унифицированных рядах используются единое функциональное и конструктивное решение и общие для всех основные части и элементы. Например, несущие устройства (корпуса, штативы, стойки, столы), соединительные устройства (электрические разъемы, муфты, замки, шарниры), энергетическо-информационные устройства (блоки питания, индикации, пульты, клавиатура), защитные устройства (кожуха, экраны, термостаты), функциональные устройства (измерительные, осветительные, наводящие, регистрирующие).

Примерами базового метода унификации конструкций могут служить унифицированные ряды электродвигателей, реле, зубчатых редукторов, осциллографов, теодолитов, модификации станков, автомобилей, бытовой техники (пылесосов, стиральных машин, холодильников). (Хорошо известен, например, факт создания автомобильным заводом в г. Тольятти семейства из порядка десяти модификаций автомобилей «Лада» на основе базовой модели <<Фиат-124>>. Фирмой <<Carl Zeiss>> разработано несколько унифицированных рядов микроскопов <<Mikrovab>> (Jena-med, Jenalumar, Jenapol), позволяющих путем комбинации узлов и элементов конструктивного ряда получать микроскопы не только с различными увеличением, полем, апертурой, но и для наблюдения в проходящем и отраженном свете, поляризационные, интерференционные, флюоресцентные, фотометрические)

\textit{Агрегатно-модульный метод унификации} (его называют также функционально-блочным, блочно-модульным) является наиболее прогрессивным, позволяющим проектировать и изготавливать изделие (их комплексы и ряды) из функциональных модулей (блоков).

Функциональный модуль представляет собой автономное конструктивное устройство, унифицированное по его функции, параметрам, геометрии, материалам, обладающее совместимостью необходимых свойств и параметров (информационных, энергетических, конструктивных, эксплуатационных) с другими модулями.

Синтез (и изменение) общей функции изделия, обеспечение его параметров и показателей качества достигаются комбинацией модулей, присоединением новых, их изъятием и заменой. Оригинальные детали, узлы и функциональные устройства при агрегатно-модульном проектировании применяются только в случаях, когда этого требует специфика изделия либо пионерское решение.

Модули в зависимости от выполняемых задач подразделяются: на несущие, управляющие, исполнительные (преобразовательные), соединительные, обеспечивающие, коммуникационные.

Агрегатно-модульный метод унификации и проектирования широко используется при создании электроизмерительных приборов, аудио-видеоаппаратуры, вычислительной техники, телефонных станций, а также других изделий, основанных на унифицированных и стандартизованных функциональных модулях и элементах микроэлектроники и электротехники (микросхемы, интегральные схемы, микропроцессоры, блоки питания, управления, коммуникации, индикации). (Типичным примером таких изделий могут служить IBM-совместимые персональные компьютеры, унификация которых охватывает принципы организации, архитектуры, интерфейсы, микропроцессорную элементную базу, накопители, периферийные устройства (дисплей, клавиатуру, принтер)).

В машиностроении и приборостроении агрегатно-модульный метод применяется в меньшей степени, что обусловлено разнообразием назначения и решаемых задач, различием физических принципов и энергоносителей в межвидовых изделиях, использованием функциональных устройств и узлов с различными физическими принципами действия в однотипных изделиях, затрудняющих их совместимость. 

В настоящее время чаще используются индивидуальный и базовый методы унификации, однако тенденции развития конструктивных решений изделий этих отраслей промышленности связаны с агрегатно-модульным проектированием. Чаще всего этот метод используется при создании однотипных изделий одного функционального назначения: станков, роботов, автомобилей, строительных машин. Что касается точных оптико-электронных приборов, то все передовые фирмы, их производящие, применяют агрегатно-модульную унификацию отдельных видов приборов либо используют унифицированные агрегаты и модули.

Некоторые фирмы специализируются на производстве таких агрегатов и модулей. Например, фирма <<Хайденхейн>> (Германия) специализируется на выпуске фотоэлектрических преобразователей (датчиков) линейных и угловых перемещений различных модификаций. Корпорации <<Oriel>>, <<Ealing>> (США) производят столы, оптические скамьи, рейтера, оправы, оптические узлы и элементы, из которых можно смонтировать лабораторные исследовательские или учебные установки, макеты приборов.

Исходя из практики точного приборостроения, следует отметить, что при создании конструкций приборов чаще всего используется смешанный метод унификации, включающий в себя элементы индивидуального, базового и агрегатно-модульного методов.

Уровень унификации обычно определяется коэффициентами применяемости $ K_\text{пр} $ и повторяемости $ K_\text{п} $:
\[ K_\text{пр} = \dfrac{n-n_0}{n}100\% ; \, K_\text{п} = \dfrac{N_\Sigma - N}{N_\Sigma} 100\%,   \]
где $ n_0 $ -- количество типоразмеров оригинальных составных частей, $ n $ -- общее количество типоразмеров составных частей, включающее оригинальные унифицированные, нормализованные, стандартные и покупные, $ N_\Sigma $ -- общее количество составных частей (деталей), $ N $ -- количество одинаковых частей (деталей), используемых в изделии повторно.

Основные цели унификации:
\begin{itemize}
\item сокращение сроков проектирования, подготовки производства, изготовления, проведения технического обслуживания и ремонтов изделий;
\item повышение экономической эффективности создания и эксплуатации изделий за счет снижения затрат при проектировании и специализации производства, технического обслуживания и ремонтов;
\item повышение показателей качества (надежности, технологичности), взаимозаменяемости изделий и их составных частей;
\item рациональное ограничение номенклатуры и объемов выпуска продукции при обеспечении функциональной и количественной ее потребностей.
\end{itemize}

Развитие базового и особенно агрегатно-модульного методов унификации изделий в точном приборостроении в настоящее время привели к революционным изменениям не только конструирования, но и производства приборов. Отпала необходимость, как это было ранее, конструировать и создавать все детали и элементы прибора (от заклепок до оптики и электроники) полностью на одном предприятии. Более выгодным, не только с позиции экономики, но и других вышеперечисленных факторов, стало создание их на основе унифицированных устройств и модулей, производимых специализированными фирмами. В результате стали дробить, уменьшать, сокращать номенклатуру видов изготавливаемых приборов даже такие гиганты промышленности, с именами которых связана история мирового точного приборостроения, как фирмы <<Carl Zeiss>>, <<Leitz>>, <<Siemens>>, <<ЛОМО>>.

Основными направлениями унификации являются:
\begin{enumerate}
\item разработка параметрических и типоразмерных рядов изделий, машин, оборудования, приборов, узлов и деталей;
\item разработка типовых изделий для создания унифицированных групп однородной продукции;
\item унификация технологических процессов;
\item сведение к оптимальному минимуму номенклатуры используемых изделий и материалов. 
\end{enumerate}

\item Агрегатирование~--- это метод создания машин, приборов и оборудования из отдельных стандартных унифицированных узлов, многократно используемых при создании различных изделий на основе геометрической и функциональной взаимозаменяемости.

При использовании данного метода вся конструкция прибора или машины рассматривается как совокупность независимых комплектующих (агрегатов), каждому из которых отводится определенная функция в общем механизме. Целью агрегатирования является увеличение мощности предприятий без лишних затрат на разработку каждой машины или прибора в отдельности.

Преимущества, даваемые данным методом: 
\begin{itemize}
\item Важнейшим преимуществом изделий созданных на основе агрегатирования, является конструктивная обратимость. Агрегатирование позволяет также многократно применять стандартные детали, узлы и агрегаты в новых модификациях изделий при изменении их конструкции.
\item Использование агрегатирования как метода стандартизации обеспечивает решение целого ряда актуальных задач в различных отраслях промышленности.
\item В настоящее время на повестке дня переход к производству техники на базе крупных агрегатов~--- модулей. Модульный принцип широко распространен в радиоэлектронике и приборостроении. Это основной метод создания гибких производственных систем.
\end{itemize}

\item Комплексная стандартизация. При комплексной стандартизации осуществляются целенаправленное и планомерное установление и применение системы взаимоувязанных требований как к самому объекту комплексной стандартизации в целом, так и к его основным элементам в целях оптимального решения конкретной проблемы. Применительно к продукции --- это установление и применение взаимосвязанных по своему уровню требований к качеству готовых изделий, необходимых для их изготовления сырья, материалов и комплектующих узлов, а также условий сохранения и потребления (эксплуатации). Практической реализацией этого метода выступают программы комплексной стандартизации (ПКС), которые являются основой создания новой техники, технологии и материалов.

\item Опережающая стандартизация заключается в установлении повышенных по отношению к уже достигнутому на практике уровню норм и требований к объектам стандартизации, которые согласно прогнозам будут оптимальными в последующее время.

Для того чтобы стандарты не тормозили технический прогресс, они должны устанавливать перспективные показатели качества с указанием сроков их обеспечения промышленным производством. Опережающие стандарты должны стандартизировать перспективные виды продукции, серийное производство которых еще не начато или находится в начальной стадии.

Различают следующие уровни стандартизации:
\begin{itemize}
\item Международная стандартизация. Органом по стандартизации является ИСО (ISO). Нормативным документом ИСО являются стандарты ИСО.
\item Межрегиональная стандартизация. Охватывает ряд независимых государств (СНГ, ЕЭС и др.). Нормативным документом стран СНГ является межрегиональный стандарт.
\item Национальная стандартизация. Это - стандартизация в пределах одного государства. Нормативным документом по национальной стандартизации в России установлен государственный стандарт России - ГОСТ Р.
\item Правила, нормы и рекомендации в области стандартизации, общероссийские классификаторы технико-экономической и социальной информации. 
\item Стандарты организаций -- отраслевые стандарты (ОСТ), стандарты предприятий (СТП), стандарты обществ. Это --- более узкий уровень стандартизации.
\end{itemize}

\end{enumerate}

\section{Категории и виды стандартов}

В зависимости от сферы действия в России установлены следующие категории нормативно-технической документации, определяющей требования к объектам стандартизации: государственные стандарты (ГОСТ), отраслевые стандарты (ОСТ) и стандарты предприятий (СТП).

Разработка современных ОЭП базируется на широком применении принципов стандартизации. На предприятиях оптического приборостроения используют те же основные категории стандартов~--- ГОСТы, ОСТы и СТП.

Государственные стандарты (ГОСТ) разрабатывают на продукцию, работы, услуги, потребности в которых носят межотраслевой характер. Стандарты этой категории принимает Госстандарт России. В стандартах содержатся как обязательные требования, так и рекомендательные. К  обязательным относятся: безопасность продукта, услуги, процесса для здоровья человека, окружающей среды, имущества, а также производственная безопасность и санитарные нормы, техническая и информационная совместимость и взаимозаменяемость изделий, единство методов контроля и единство маркировки. Требования обязательного характера должны соблюдать государственные органы управления и все субъекты хозяйственной деятельности независимо от формы собственности. Рекомендательные требования стандарта становятся обязательными, если на них есть ссылка в договоре (контракте).

Государственный стандарт~--- ранее основная категория стандартов в СССР, сегодня межгосударственный стандарт в СНГ. Принимается Межгосударственным советом по стандартизации, метрологии и сертификации (МГС). В настоящее время ГОСТы являются нормативными неправовыми актами.

Стандарты, принятые до 1996 года, являлись нормативно-правовыми актами и поэтому были обязательными для применения в тех областях, которые определялись преамбулой самого стандарта. Для документов, принятых после 1996 года, нормативность ГОСТов сама по себе перестала означать обязательность документа. В настоящее время документ становится обязательным нормативно-правовым актом после регистрации в Минюсте.

Отраслевые стандарты (ОСТ) разрабатываются применительно к продукции определенной отрасли. Их требования не должны противоречить обязательным требованиям государственных стандартов, а также правилам и нормам безопасности, установленным для отрасли. Принимают такие стандарты государственные органы управления (например, министерства), которые несут ответственность за соответствие отраслевых стандартов обязательным требованиям ГОСТ Р.

Под отраслью понимается совокупность субъектов хозяйственной деятельности независимо от их ведомственной принадлежности и форм собственности, разрабатывающих и (или) производящих продукцию (выполняющих работы и оказывающих услуги) определенных видов, которые имеют однородное потребительное или функциональное назначение.

Так же как и в любой другой отрасли промышленности, в оптическом приборостроении существует система ОСТов. ОСТы обязательны для всех предприятий и организаций данной отрасли (министерства, ведомства), а также для предприятий других отраслей, использующих продукцию этой отрасли. ОСТы утверждаются министерствами, являющимися ведущими в производстве данного вида продукции.

ОСТы устанавливаются на те виды продукции, которые не регламентированы ГОСТами: отдельные виды продукции ограниченного применения; технологическую оснастку, предназначенную для применения в данной отрасли; сырье, материалы, полуфабрикаты внутриотраслевого применения. ОСТы также регламентируют нормы, правила, требования и обозначения, обеспечивающие оптимальное качество продукции отрасли. В ряде случаев ОСТы устанавливают ограничения (по типоразмерам, номенклатуре) или развивают ГОСТы применительно к данной отрасли.

ОСТы разрабатываются головной научно-исследовательской организацией отрасли. В оптическом приборостроении действует значительное число ОСТов, а также руководящих материалов отрасли (РМО) и руководящих технических материалов (РТМ), развивающих, дополняющих или ограничивающих систему ГОСТов.

Диапазон применяемости отраслевых стандартов ограничивается предприятиями, подведомственными государственному органу управления, принявшему данный стандарт. Контроль за выполнением обязательных требований организует ведомство, принявшее данный стандарт.

Стандарты предприятий (СТП) разрабатываются и принимаются самими предприятиями. СТП устанавливаются на детали, сборочные единицы, технологическую оснастку, технологические процессы, нормы, требования и правила, применяемые на данном предприятии, и утверждаются руководством предприятия.

Эта категория стандартов обязательна для предприятия принявшего этот стандарт.

СТП бывают трех видов:
\begin{enumerate}
\item ограничительные, которые создаются на основе государственных или отраслевых и ограничивают применение на предприятиях установленной стандартами номенклатуры, типоразмеров, марок материалов;
\item дополняющие, создаваемые при необходимости дополнить государственные или отраслевые стандарты данными (требованиями, характеристиками), отсутствующими в этих стандартах;
\item оригинальные, разрабатываемые при условии, что на стандартизуемые объекты отсутствуют ГОСТы или ОСТы.
\end{enumerate}

Оригинальные и дополняющие СТП не должны ухудшать показателей государственных и отраслевых стандартов.

Вопросами стандартизации на предприятиях, в научно-исследовательских, проектно-конст\-рук\-торс\-ких и других организациях занимаются специальные подразделения (отделы, бюро, группы, лаборатории).

Стандарты общественных объединений (научно-технических обществ, инженерных обществ). Эти нормативные документы разрабатывают на принципиально новые виды продукции, процессов или услуг; передовые методы испытаний, а также нетрадиционные технологии и методы управления производством. Общественные объединения преследуют цель распространения перспективных результатов мировых научно-технических достижений, фундаментальных и прикладных исследований.

Эти стандарты служат важным источником информации о передовых достижениях, и по решению самого предприятия они принимаются на добровольной основе для использования отдельных положений при разработке стандартов предприятия.

Правила по стандартизации (ПР) и рекомендации по стандартизации (Р) по своему характеру соответствуют нормативным документам методического содержания. Они могут касаться порядка согласования норм документов, предоставления информации о принятых стандартах отраслей, общественных и других организаций в Госстандарт РФ, создание службы по стандартизации на предприятии, правил проведения государственного контроля за соблюдением обязательных требований ГОСТ и других вопросов организационного характера. ПР и Р разрабатываются организациями, подведомственными Госстандарту РФ и Госстрою РФ.

\textit{Общероссийский классификатор технико-экономической информации} (далее -- общероссийский классификатор) --- нормативные документы, распределяющие технико-экономическую информацию в соответствии с ее классификацией (классами, группами, видами и другим) и являющиеся обязательными для применения при создании государственных информационных систем и информационных ресурсов и межведомственном обмене информацией.

Существуют следующие виды стандартов:
\begin{itemize}
\item основополагающие стандарты;
\item стандарты на продукцию;
\item стандарты на работы и процессы;
\item стандарты на методы испытаний, контроля, анализа;
\item технические условия.
\end{itemize}

Основополагающие стандарты, в свою очередь, делятся на:
\begin{itemize}
\item общетехнические стандарты;
\item организационно-методические стандарты.
\end{itemize}

\textit{Общетехнические стандарты}, регламентирующие термины определения, обозначения, номенклатуру показателей качества выполняют функцию обеспечения информационной совместимости однозначности понимания объекта стандартизации. Общетехнические стандарты, регламентирующие общие требования и (или) нормы выполняют функцию обеспечения технического единства и взаимосвязи объектов стандартизации. Стандарты, регламентирующие методы, устанавливают общие методы проектирования подготовки производства, испытаний, хранения, транспортирования, эксплуатации и ремонта продукции.

\textit{Организационно-методические стандарты}, которые регламентируют основные (общие), положения устанавливают общие требования, обеспечивающие организационно-техническое единство объектов стандартизации. Стандарты, регламентирующие порядок (правила) обеспечивают единство и взаимосвязь процессов управления в различных областях деятельности. Стандарты, регламентирующие построение (изложение, оформление, содержание) обеспечивают информационную совместимость документации.

Стандарты на продукцию регламентируют требования к продукции и делятся на:
\begin{itemize}
\item стандарты общих технических требований;
\item стандарты общих технических условий;
\item стандарты технических условий.
\end{itemize}

Стандарты общих технических требований и общих технических условий устанавливают всесторонние требования к группе однородной продукции по ее разработке, производству, обращению и потреблению (эксплуатации).

Стандарты, регламентирующие параметры и (или) размеры, типы, марки, сортамент, конструкцию устанавливают требования к типоразмерным и параметрическим рядам, обеспечивающим унификацию и взаимозаменяемость продукции.

Стандарты, регламентирующие правила приемки, методы контроля, маркировку, упаковку, транспортирование, хранение, эксплуатацию и ремонт данной продукции выполняют функцию по обеспечению заданного качества продукции при ее производстве, сохранении качества при ее транспортировании и хранении, полноценного использования продукции при потреблении, восстановление продукции.

Стандарты технических условий регламентируют требования не к группе однородной продукции, а к конкретной выпускаемой продукции.

Стандарты на работу и процессы устанавливают правила проведения различного рода работ, процессов. Главным их требованием является обеспечение безопасности жизни, здоровья и имущества при проведении данных работ (процессов).

Стандарты на методы испытаний, контроля, анализа регламентируют требования к методам испытаний, проведению научно-исследовательских работ, испытаниям при сертификации продукции.

\textit{Технические условия} (ТУ) --- это нормативный документ, который имеет отраслевое подчинение, (может иметь временное значение до введения ГОСТа на данную продукцию). Технические условия  разрабатываются предприятиями и другими субъектами хозяйственной деятельности в том случае, когда стандарт создавать нецелесообразно. Объектом ТУ может быть продукция разовой поставки, выпускаемая малыми партиями, а также произведения художественных промыслов. Особенность процедуры принятия ТУ состоит в том, что во время приемки новой продукции происходит их окончательное согласование с приемочной комиссией. Перед этим предварительно рассылается проект ТУ тем организациям, представители которых будут на приемке продукции. ТУ считаются окончательно согласованными, если подписан акт приемки опытной партии (образца).

Основополагающими ГОСТами при проектировании являются общетехнические и ор\-га\-ни\-за\-ци\-он\-но-методические. Особое место среди них занимает Единая система конструкторской документации~--- ЕСКД (группа Т52).

Все стандарты ЕСКД распределены по классификационным группам, приведенным в табл. 1.

Табл. 1
Перечень классификационных групп стандартов ЕСКД


Наряду с ЕСКД при проектировании ОЭП широко используют другие комплексы ГОСТов, в частности:
\begin{itemize}
\item ГСИ -- Государственную систему обеспечения единства измерений (класс Т8);
\item ЕСТД -- Единую систему технологической документации (группа Т53);
\item ЕСЗКС -- Единую систему защиты от коррозии старения материалов и изделий (класс Т9);
\item ЕСТПП -- Единую систему технологической подготовки производства (группа Т53);
\item ЕСДП -- Единую систему допусков и посадок (группа Г12);
\item ССБТ -- Систему стандартов безопасности труда (группы ТОО, Т58);
\item ССНТ -- Систему стандартов «Надежность в технике» (группа Т51);
\item ССЭТЭ -- Систему стандартов эргономики и технической эстетики (группа Т58).
\end{itemize}

Нумерация государственных стандартов, а также стандартизация их по видам и происхождению производятся в следующей форме: В начале номера ставится аббревиатура <<ГОСТ>>, что означает <<ГОсударственный Стандарт>>. Затем следует символ «Р», если стандартизация ГОСТ проводилась в России, а не в СССР. Далее, через пробел в номере ГОСТ следует стандартизация аббревиатуры аутентичности данного ГОСТ текстам международных стандартов ИСО, МЭК. После перечисления всех аббревиатур ставится номер ГОСТ. Возможны следующие варианты стандартизации:
\begin{itemize}
\item ГОСТ
\item ГОСТ Р
\item ГОСТ Р ИСО
\item ГОСТ Р МЭК
\item ГОСТ Р ИСО/ТС
\item ГОСТ Р ИСО/ТУ
\item ГОСТ Р ИСО ТО
\item ГОСТ Р ИСО/МЭК
\item ГОСТ Р ИСО/МЭК ТО 
\item ГОСТ Р ИСО/МЭК МФС 
\item ГОСТ Р ЕЭК ООН 
\item ГОСТ Р ЕН ИСО
\end{itemize}