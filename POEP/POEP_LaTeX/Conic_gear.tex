\chapter{Коническая передача}
\label{ch:conic}

Предназначены для передачи движения между двумя пересекающимися осями (валами). Угол пересечения между осями $ \Sigma $ составляет, как правило, $ 90^\circ $.

Боковые поверхности конического колеса образованы перекатывающейся без скольжения плоскости, касающейся основания конуса. При перекатывании любая точка лежит на образующей конуса.

Делительная окружность конического колеса --- окружность, получаемая в пересечении делительного конуса и внешнего дополнительного конуса; к этой делительной окружности относится и выбираемый СТ СЭВ 310-76 внешний окружной делительный модуль $ m_e $. \marginnote{Индекс $ e $~-- к внешнему диаметру,\break $ i $~-- к внутреннему, $ m $~-- для параметров, относящихся к профилю зуба в нормальной к его направлению плоскости, проходящей через середину зуба, $ a $~-- к вершине зуба, $ f $~-- к впадине зуба}