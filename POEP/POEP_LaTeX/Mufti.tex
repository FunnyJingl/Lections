\chapter{Муфты}

\newthought{Муфты}\marginnote{\allcaps{МУФТЫ}} --- устройства, предназначенные для соединения концов валов или для соединения валов с расположенными на них деталями. 

Основное назначение муфт --- передача вращающего момента без изменения его модуля и направления. Муфты могут выполнять и другие функции: предохранять механизм от перегрузок, компенсировать несоосность валов, разъединять или соединять валы во время работы и др.

Муфты: 
\begin{itemize}
	\item соединительные:
	\begin{itemize}
		\item глухие (втулочные);
		\item пальцевые (поводковые);
		\item эластичные пальцевые;
		\item упругие муфты с винтовыми пружинами сжатия;
		\item мембранные муфты;
		\item крестовые муфты;
	\end{itemize}
	\item предохранительные:
	\begin{itemize}
		\item с разрушаемыми элементами;
		\item самоуправляемые.
	\end{itemize}
\end{itemize}

\section{Соединительные муфты}

\newthought{Глухие муфты}\marginnote{\allcaps{ГЛУХИЕ МУФТЫ}} предназначются для жёсткого соединения двух валов:
\begin{itemize}
	\item можно скомпенсировать только продольные смещения при отсутствии поперечных смещений и смещений по углу;
	\item малые габариты в радиальном направлении;
	\item динамические нагрузки не демпфируются;
	\item конструкция неразборная;
	\item расчет муфт производится по срезу штифта;
	\item посадка втулки на вал с зазором $ \dfrac{H9}{d9} $.
\end{itemize}

\newthought{Поводковые муфты}\marginnote{\allcaps{ПАЛЬЦЕВЫЕ\break (ПОВОДКОВЫЕ)\break МУФТЫ}}:
\begin{itemize}
	\item компенсирует несоосность (до 0,5 мм) и небольшие продольные смещения и перекосы;
	\item проста в конструкции и эксплуатации;
	\item недостаток --- наличие люфта (зазора) между пальцем и пазом, что приводит к увеличению мёртвого хода всего механизма;
	\item расчёт муфты --- на срез штифтов и на срез пальца;
	\item передаточное отношение не остаётся постоянным вследствие перекоса и поперечного смещения;
	\item материалы полумуфты и пальца: сталь 45 и 45Х;
	\item величину зазора между	пальцем и пазом назначают из условия климатического	исполнения по ГОСТ 15150-69.
\end{itemize}

\newthought{Эластичные пальцевые муфты}\marginnote{\allcaps{ЭЛАСТИЧНЫЕ\break ПАЛЬЦЕВЫЕ\break МУФТЫ}}:
\begin{itemize}
	\item по свойствам идентична пальцевым;
	\item снижает динамические нагрузки в механизмах вследствие деформации упругого промежуточного диска;
	\item материал полумуфты и пальца --- конструкционные стали, материал диска – кожа либо резина;
	\item недостатки – большой упругий мёртвый ход и усиленный износ упругого диска, поэтому не применяются в точных кинематических цепях.
\end{itemize}

\newthought{Упругие муфты с винтовыми пружинами сжатия}\marginnote{\allcaps{УПРУГИЕ МУФТЫ С\break ВИНТОВЫМИ\break ПРУЖИНАМИ СЖАТИЯ}}:
\begin{itemize}
	\item применяют при необходимости демпфирования больших ударных нагрузок;
	\item конструкция: 2 полумуфты + 2 пружины сжатия + стопорное кольцо.
\end{itemize}

\newthought{Мембранные муфты}\marginnote{\allcaps{МЕМБРАННЫЕ МУФТЫ}}:
\begin{itemize}
	\item жёсткое неразборное соединение валов;
	\item компенсируют несовпадение длины валов, несоосность и перекос;
	\item могут иметь малый упругий мёртвый ход;
	\item используются в кинематических цепях средней и высокой точности;
	\item материалы: полумуфты из конструкционных сталей, упругие элементы – из материалов для изготовления пружин;
	\item расчёты --- на устойчивость упругих элементов;
	\item мембраны применяются редко по причине высокой жёсткости;
	\item вместо этого применяются либо части мембран, либо другие конструкции.
\end{itemize}

\newthought{Крестовые муфты}\marginnote{\allcaps{КРЕСТОВЫЕ МУФТЫ}}: безлюфтовый вариант крестовой муфты применяют в цепях самой высокой точности.

\section{Предохранительные муфты}
\begin{itemize}
	\item передаточное отношение предохранительных муфт $ 1\ldots \infty $ в зависимости от принципа предохранения и защиты механизмов при различных перегрузках, запрещённых направлениях движения, превышении скоростей;
	\item применяют для предотвращения выхода из строя при различных видах перегрузки: статическим и динамическим моментом, при повышении или уменьшении допустимой скорости вращения, изменения направления вращения;
	\item предохранительные муфты: самоуправляемые (без разрушения элементов) и неуправляемые (с разрушением элементов).
\end{itemize}

