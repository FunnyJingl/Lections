\chapter{Электрорадиоэлементы и электронные узлы}

Электронный элемент -- это конструктивно самостоятельное образование, выполняющее одну элементарную функцию (резисторы, конденсаторы, катушки
индуктивности).

Электронная схема реализуется на основе многих дискретных элементов (запоминающий элемент, усилительный каскад, логический элемент).

Функциональный модуль образуется при соединении нескольких элементарных схем в одну конструктивно законченную сборочную единицу.

Узел -- конструктивное объединение нескольких модулей.

Полевые и биполярные транзисторы, полупроводниковые диоды и резисторы, конденсаторы и прочие электронные приборы и радиодетали часто называют элементами радиоэлектронной аппаратуры~(РЭА), или электрорадиоэлементами, так как они составляют основу функциональных структур, реализующих обусловленные назначением аппаратуры алгоритмы формирования, преобразования хранения, обработки и воспроизведения сигналов. 

Использование специальной технологии изготовления тонких слоев различной проводимости на изоляционной подложке или целенаправленное изменение проводимости в определенных зонах полупроводникового материала позволило реализовать и объединить различные электрические функции в едином технологическом процессе. При установке такого элемента в корпус с необходимыми выводами получают микросхему (МС). Одна МС заменяет несколько элементарных схем, выполненных на основе дискретных элементов. 

В настоящее время используют две разновидности технологических процессов изготовления МС:
\begin{itemize}
\item тонкопленочные процессы;
\item полупроводниковые процессы.
\end{itemize}

Интегральная (микро) схема (ИС, ИМС), чип, микрочип~-- микроэлектронное устройство~-- электронная схема произвольной сложности, изготовленная на полупроводниковой подложке и помещённая в неразборный корпус, или без такового, в случае вхождения в состав микросборки. Микросхемы бывают полупроводниковыми, пленочными и гибридными.

Полупроводниковые микросхемы изготавливаются путем формирования в монокристаллическом теле полупроводника структуры интегральной схемы (ИС) при помощи технологических операций. Создаются различные области, обладающие дырочной ($ р $-область) и электронной ($ n $-область) проводимостями.

Образованные области в полупроводнике соответствуют по своим функциям определенным элементам: активным (транзистор, диод) и пассивным (резистор, конденсатор и др.). Объемные токоведущие дорожки создаются нанесением на поверхность полупроводника инверсного слоя высокой проводимости. Такая полупроводниковая ИС может представлять собой законченную конструкцию микроэлектронного изделия, т. е. конструкцию электрической цепи, непосредственно реализующей параметры и характеристики этой цепи. Таким образом в полупроводниковой микросхеме все элементы и межэлементные соединения выполнены в объеме и на поверхности полупроводника. Такие микросхемы выпускаются обычно предприятиями электронной промышленности и используются разработчиками ОЭП в качестве покупных элементов.

Пленочные ИС имеют подложку (плату) из диэлектрика (стекло, керамика). Пассивные элементы, т. е. резисторы, конденсаторы, катушки и соединения между элементами, выполняются в виде различных пленок, нанесенных на подложку. Понятия плёночная технология включает в себя процессы термовакуумного испарения и катодного распыления, также трафаретная печать. Активные элементы (диоды, транзисторы) не делаются пленочными, так как не удалось добиться их хорошего качества. Таким образом, пленочные ИС содержат только пассивные элементы.

Принято различать ИС \textit{тонкопленочные}, у которых толщина пленок не более 2 мкм, и \textit{толстопленочные}, у которых толщина пленок значительно больше. Разница между этими ИС заключается не столько в толщине пленок, сколько в различной технологии их нанесения. Подложки представляют собой диэлектрические пластинки толщиной 0,5-1,0 мм. тщательно отшлифованные и отполированные. При изготовлении пленочных резисторов на подложку наносят резистивные пленки. Если сопротивление резистора не должно быть очень большим, то пленка делается из сплава высокого сопротивления, например из нихрома. А для резисторов высокого сопротивления применяется смесь металла с керамикой. На концах резистивной пленки делаются выводы в виде металлических пленок, которые вместе с тем являются линиями, соединяющими резистор с другими элементами. Сопротивление пленочного резистора зависит от толщины и ширины пленки, ее длины и материала. Для увеличения сопротивления делают пленочные резисторы зигзагообразной формы. 

Удельное сопротивление пленочных резисторов выражают в особых единицах --- омах на квадрат, так как сопротивление данной пленки в форме квадрата не зависит от размеров этого квадрата. Действительно, если сделать сторону квадрата, например, в два раза больше, то длина пути тока увеличится вдвое, но и площадь поперечного сечения пленки для тока также возрастет вдвое: следовательно, сопротивление останется без изменения. 

Тонкопленочные резисторы по точности и стабильности лучше толстопленочных, но производство их сложнее и дороже. У тонкопленочных резисторов удельное сопротивление может быть от 10 до 300~Ом на квадрат. В течение длительного времени эксплуатации сопротивление этих резисторов мало изменяется.

Толстопленочные резисторы имеют удельное сопротивление от 2~Ом до 1~МОм на квадрат. Их стабильность во времени хуже, чем у тонкопленочных резисторов.

Пленочные конденсаторы чаще всего делаются только с двумя обкладками. Одна из них наносится на подложку и продолжается в виде соединительной линии, затем на нее наносится диэлектрическая пленка, а сверху располагается вторая обкладка, также переходящая в соединительную линию. В зависимости от толщины диэлектрика конденсаторы бывают тонко- и толстопленочными.

Пленочные катушки индуктивности делаются в виде плоских спиралей, чаще всего прямоугольной формы. Ширина проводящих полосок и просветов между ними обычно составляет несколько десятков микрометров. 

Так как тонкопленочная технология позволяет изготовлять только пассивные элементы, а полупроводниковая -- активные элементы, то целесообразно использовать их комбинацию. Это приводит к созданию гибридных интегральных МС. Гибридные ИС~--- интегральные схемы, в которых применяются плёночные пассивные элементы и навесные элементы (резисторы, конденсаторы, диоды, оптроны, транзисторы), называемые компонентами.

Электрические связи между элементами и компонентами осуществляются с помощью плёночного или проволочного монтажа. Реализация функциональных элементов в виде ГИС экономически целесообразна при выпуске малыми сериями специализированных вычислительных устройств и другой аппаратуры. Высоких требований к точности элементов в ТЗ нет. Условия эксплуатации изделия нормальные. Навесными элементами в микроэлектронике называют миниатюрные, обычно бескорпусные диоды и транзисторы, представляющие собой самостоятельные элементы. 

Иногда в гибридных ИС навесными могут быть и некоторые пассивные элементы, например, миниатюрные конденсаторы с такой большой емкостью, что их невозможно осуществить в виде пленок. Это могут быть и миниатюрные трансформаторы. В некоторых случаях в гибридных ИС навесными являются целые полупроводниковые ИС. Проводнички от транзистора или от других навесных элементов присоединяются к соответствующим точкам схемы чаше всего методом термокомпрессии (провод при высокой температуре прижимается под большим давлением).

Достоинства гибридных микросхем:
\begin{itemize}
\item возможность предварительного выбора дискретных элементов;
\item низкая стоимость подложек и возможность применения значительно больших номиналов тонкоплёночных конденсаторов и мощных резисторов.
\end{itemize}

Недостатком являются дополнительные контактные площадки для монтажа дискретных элементов или полупроводниковых ИМ, которые можно выполнить по тонкоплёночной технологии.

\section{Компоновка электронного тракта}

Компоновка электронного тракта ОЭП --- часть процесса конструирования, связанного с размещением в пространстве или на плоскости различных радиодеталей, микросхем и блоков. Эту задачу чаще всего решают, используя готовые элементы с заданными формами, размерами и массой, которые следует расположить с учетом электрических, механических, тепловых и других видов связей. Компоновочные характеристики электронного узла должны находиться в соответствии с параметрами среды, в которой работает прибор, т.е. необходимо учитывать дополнительные объемы для устройства герметизации и уплотнений, установки виброзащитных амортизаторов и систем охлаждения.

Для обеспечения рациональной компоновки электронного тракта необходимо прежде всего выяснить его иерархическую структуру (рис.~\ref{pic:7struct}).
\begin{figure}[h!]
	\caption{ Иерархическая структура электронного тракта }
	\includegraphics[width=1\textwidth]{7struct.png}
	\label{pic:7struct}
\end{figure}

В зависимости от сложности ОЭП его электронный тракт может включать ряд устройств, выполненных в виде шкафов, стоек, сложных по конструкции блоков, пультов управления, соединенных между собой и с оптическими блоками с помощью кабелей. В свою очередь, устройства конструктивно представляют собой корпусы, в которых расположены блоки, содержащие электронные узлы и различные вспомогательные устройства, например для вентиляции, охлаждения, герметизации. 

Корпусы электронных устройств обычно имеют несущий каркас из профильных элементов, соединенных сваркой, клепкой или резьбовыми крепежными элементами, и кожух из металлического листа или пластмасс. Кожухи устройств могут быть съемными и несъемными. В последнем случае кожухи могут иметь съемные стенки, отверстия, люки, дверцы для доступа к внутренним частям в целях регулировки и обслуживания. К корпусным элементам крепятся розетки и вилки электрических соединителей. Соединители для подключения блоков и внешних устройств коммутируются между собой с помощью проводного монтажа. При этом внутренний проводной монтаж выполняют в виде жгутов или плоских кабелей. Жгуты состоят из изолированных проводов, объединенных пространственно перевязкой. Для их фиксации используются скобы и зажимы. 

Плоские кабели представляют собой параллельные проводники, соединенные изоляционным материалом. Их достоинством является строго фиксированное взаимное положение проводников, что облегчает монтаж и проверку блоков и важно для устройств, работающих на высоких частотах. Блоки внутри стоек могут крепиться как на неподвижных, так и на вращающихся или выдвижных рамах и шасси.

Блоки конструктивно могут быть выполнены в виде панелей, на которых закрепляются отдельные электронные узлы, электрически связанные проводным или печатным монтажом, и которые снабжены соответствующими соединительными элементами. В качестве соединительных элементов между узлами и электромонтажом панели используют разъемные, зажимные, пружинные узлы.

Наиболее распространенной конструкцией электронных узлов является сменная ячейка с соединителем, состоящая из платы, на одной или обеих сторонах которой установлены электрорадиоэлементы и интегральные микросхемы, соединенные между собой в соответствии со схемой объемным (проводным) или печатным электромонтажом.

В настоящее время объемный (проводной) монтаж применяют редко, в основном в тех случаях, когда на панелях или платах устанавливаются крупногабаритные элементы (трансформаторы, индикаторы). Кроме того, иногда такой вид монтажа используется в мелкосерийном производстве.
Пример конструктивного оформления электронного узла способом объемного монтажа приведен на рис.~\ref{pic:7volume}. Сборка этого узла осуществляется следующим образом. В отверстия текстолитовой платы~2 вставляют и развальцовывают контакты~1. Затем к контактам припаивают радиоэлементы, например, диоды резисторы. С другой стороны, к контактам в соответствии со схемой припаивают перемычки 3.

\begin{figure}[h!]
	\caption{ Функциональный электронный узел с объемным монтажом }
	\includegraphics[width=1\textwidth]{7volume.png}
	\label{pic:7volume}
\end{figure}

Основным видом электромонтажа в настоящее время является печатный. Печатный монтаж обладает следующими преимуществами в сравнении с объемным: 
\begin{itemize}
\item объединение электрорадиоэлементов и электромонтажа в единую конструктивную единицу;
\item повышение плотности компоновки и монтажа;
\item уменьшение массы и размеров;
\item технологичность, сокращение времени изготовления и экономия материалов;
\item уменьшение ошибок при монтаже;
\item повышение надежности и прочности соединений;
\item возможность автоматизации процессов разработки, изготовления и сборки.
\end{itemize}

Вместе с тем следует указать и некоторые недостатки печатных плат:
\begin{itemize}
\item нежелательные емкостные и индуктивные связи;
\item трудность внесения изменений;
\item увеличение времени разработки.
\end{itemize}

Обычно при разработке печатных плат исходят из того, что печатный монтаж размещается с одной стороны платы, а навесные элементы -- с другой.
При конструировании печатных плат необходимо учитывать технологические особенности существующих методов изготовления, наиболее распространенными из которых являются химический, электрохимический и комбинированный

Печатная плата с навесными элементами называется печатным узлом. Для его изготовления разрабатывается сборочный чертеж, дающий полное представление о компоновке навесных электро- и радиоэлементов и других деталей на печатной плате.

При компоновке электронных узлов и блоков необходимо обеспечивать допустимый минимум паразитных электрических взаимодействий. Кроме того, для узлов с повышенным тепловыделением следует проводить расчет тепловых режимов. При этом учитывают мощность и расположение источников выделения теплоты в узле, физические свойства материалов деталей, конструкций и расположение самого узла, параметры окружающей среды и др. В зависимости от результатов расчетов в конструкции узла могут быть предусмотрены радиаторы для отвода и равномерного распределения тепловой энергии. В некоторых случаях может потребоваться обдув элементов или вентиляция внутреннего объема, в котором размещается электронный узел.

\section{Электрические контакты }

В любой конструкции, которая должна содержать отдельно изготовляемые электрические устройства, необходимо обеспечить между ними электрическую связь. Эта связь в основном обеспечивается с помощью электрических контактов, представляющих собой конструктивно оформленное соединение токопроводящих частей. Электрическим контактом называется соприкосновение тел, обеспечивающее непрерывность электрической цепи, а также устройство, содержащее соприкасающиеся детали. Качество мест соприкосновения токопроводящих частей должно быть таким, чтобы контакты не оказывали влияния на параметры передаваемого сигнала, т.е. главным назначением электрических контактов является беспрепятственное прохождение электрического тока через поверхность контакта.

Роль электрических контактов в современной технике существенно возросла в связи с тенденцией к миниатюризации устройств управления и контроля, появлением огромного количества портативных устройств. При этом номинальные площади электрического контакта сократились, приблизившись к микрометровому и нанометровому диапазонам размеров, а количество контактов в единице объема увеличилось на порядки. Требования к надежности контактов резко возросли, а необходимость снижения переходного сопротивления и, следовательно, тепловыделения в контактах стала очень актуальной.
При проектировании электрических контактов необходимо учитывать также, что соединение и разъединение цепей (коммутация) происходит в течение некоторого времени, а между контактами после их механического разрыва существует электрическая связь вследствие газоразрядных процессов. Надежная работа контактов определяется также конструкцией контактного устройства, спроектированной с учетом конкретных параметров коммутируемых цепей и условий эксплуатации.

Классификация электрических контактов может быть выполнена по функциональным признакам, эксплуатационным параметрам (величина тока и напряжения) и конструктивным признакам (кинематика движения контактных элементов и геометрия).

По функциональному признаку контакты могут быть разделены на:
\begin{itemize}
\item неразъемные;
\item разъемные;
\item разрывные;
\item скользящие.
\end{itemize}

Классификация электрических контактов по эксплуатационным параметрам в самой общей постановке может быть выполнена, базируясь на трех обширных отраслях техники - электроэнергетика (производство и распределение электроэнергии в сетях энергоснабжения), промышленная и бытовая электротехника (использование электроэнергии для работы промышленных установок, освещения и бытовых электроприборов), радиоэлектроника (телекоммуникации и микроэлектроника). Поэтому ключевыми параметрами классификации служат величины тока в контактах (сильно и слаботочные цепи) и контактное напряжение (высоко и низковольтные цепи).

Классификация по конструктивным признакам приведена на рис.~\ref{pic:7contact}.
\begin{figure}[h!]
	\caption{ Классификация электрических контактов }
	\includegraphics[width=1\textwidth]{7contact.png}
	\label{pic:7contact}
\end{figure}

В целом, по конструкции все виды контактов могут быть разделены на два класса~-- неподвижных (контактные соединения) и подвижных. Неподвижные контакты можно разделить на неразборные (неразъемные), разборные и разъемные.

В зависимости от формы поверхности соприкосновения все контакты разделяют на точечные, линейные и плоскостные (табл. 1). 
Таблица 1

Точечные контакты применяют при малых токах (доли и единицы ампера) и требуют небольших контактных усилий. 

Линейные контакты применяют при токах от нескольких до десятков ампер. Контактные усилия должны быть значительно больше, чем у точечных контактов. Объем контактов также больше. Поэтому для экономии материалов линейные контакты часто выполняют пластинчатыми. 

Плоскостные контакты используют при больших токах и требуют значительных контактных усилий. Для обеспечения соприкосновения контактов по всей контактной поверхности требуется или точная установка контактов, или упругое соединение подвижного контакта с контактонесущей системой, или упругое соединение неподвижного контакта с основанием.

В работе контактов можно выделить четыре состояния: замкнутое, размыкание, разомкнутое и замыкание. Замкнутое состояние контактов характеризуется искажением параметров цепи из-за нестабильности сопротивления, емкости и индуктивности в месте контактирования. Поэтому основное требование к контактам~--- ограничение этих искажений. При определении активного сопротивления в контактной области необходимо учитывать, что контакт между двумя телами происходит не по всей поверхности соприкосновения, а лишь на отдельных участках. Сумма площадей этих участков получила название эффективной (действительной) поверхности.

Из-за волнистости и шероховатости поверхностей и присутствия непроводящих пленок эффективная (реальная) площадь контакта в сотни раз меньше, чем номинальная, и, кроме того, под действием нагрузки различные участки площади контакта деформируются по-разному, выступы шероховатости или волны~-- упруго, а микровыступы~-- пластически. Общая схема контакта представлена на рис. 4.

\begin{figure}[h!]
%	\caption{ Схема прохождения тока через контакт твердых тел:\\	1 -- номинальная (кажущаяся) площадь контакта; 2 -- контурная площадь, несущая нагрузку (упругая деформация) и охватывающая пятна фактического контакта;\\ 3 -- фактическая (реальная) площадь контакта (пластическая деформация); 4 -- площадь с квазиметаллической проводимостью (пленки); 5 -- пятна металлического контакта }
	\includegraphics[width=1\textwidth]{7scheme.png}
	\label{pic:7scheme}
\end{figure}

\section{Контактные материалы}
Идеальный контактный материал должен обладать следующими физическими характеристиками: 
\begin{itemize}
\item низким удельным сопротивлением $ \rho $ для уменьшения потерь энергии при прохождении тока через контакт; 
\item низким и постоянным температурным коэффициентом сопротивления для исключения значительного изменения $ R_\text{пер} $ (переходное сопротивление, вызванное наличием микронеровностей на поверхностях соприкасающихся контактов (сопротивление стягивания)) при возрастании тока $ I $ и температуры;
\item высокой удельной теплопроводностью, чтобы быстрее отвести из зоны контактов теплоту, возникающую как в результате прохождения электрического тока, так и в результате искрения и дугообразования при размыкании и замыкании; 
\item высокой износостойкостью, чтобы противостоять механическому истиранию; 
\item высокой температурой плавления, чтобы уменьшить возможность сваривания и снизить эрозию контактов; 
\item высокой удельной теплоемкостью для увеличения теплоемкости контактов и ограничения вследствие этого температуры в зоне контакта;
\item неокисляемостью и антикоррозионностью при использовании контактов в любых средах; 
\item малой твердостью, чтобы уменьшить дребезг контактов при замыкании; хорошими технологическими свойствами; 
\item низкой стоимостью; 
\item большими значениями минимальных напряжения $ U_0 $ и тока $ I_0 $  дугообразования. 
\end{itemize}

Столь противоречивые требования не позволяют рекомендовать определенный материал для работы в любых условиях. Выбор материала определяется параметрами коммутируемых цепей - постоянный или переменный ток, значения номинального и разрываемого тока, характер нагрузки в цепи; условиями эксплуатации~-- частота включений, возможность профилактического ухода и т.п.; условиями работы контактов, определяемыми параметрами контактонесущей системы~--- контактное усилие, наличие или отсутствие скольжения контактов, скорости замыкания и размыкания и т.п.

%\section{Характеристики наиболее распространенных\\ контактных материалов}

Металлы, вследствие, прежде всего высокой тепло-~и электропроводности в наилучшей степени сочетают свойства, необходимые для эффективной передачи тока через контакт с наименьшими потерями. В общем случае твердые металлические проводники могут быть разделены на две группы:
\begin{itemize}
\item технически чистые металлы, прежде всего, широко применяемые в электрических контактах медь и алюминий, иногда включающие небольшие добавки других металлов для улучшения механических свойств;
\item сплавы со специфическими свойствами, например, повышенной износостойкостью и низким трением, среди которых наиболее часто используются бронзы, латуни и некоторые алюминиевые сплавы.
\end{itemize}

В процессе эксплуатации металлические проводники подвергаются различным механическим и тепловым напряжениям, воздействиям окружающей среды.

Поэтому их практическое использование требует детального знания различных свойств материала проводника~-- электрических, тепловых, химических, и механических. Медь, алюминий и их сплавы преимущественно используются для сильноточных электрических контактов, а благородные металлы и их сплавы~-- для слаботочных, при этом благородные металлы используются преимущественно в виде покрытий. 

\section{ Неподвижные контакты }

Неразборные контакты предназначены для постоянного соединения электрических цепей. Они обладают большой прочностью и обеспечивают стабильный электрический контакт с низким переходным сопротивлением. К неразъемным контактам предъявляют следующие требования:
\begin{itemize}
\item удобное и быстрое соединение;
\item минимальное сопротивление;
\item механическая прочность;
\item минимальные размеры соединения, чтобы не увеличивать межконтактную емкость и не создавать замыканий между соседними контактами.
\end{itemize}

В процессе эксплуатации не предусматривается разъединение цепей в месте неразборного контакта. Поэтому для получения неразборных контактов часто используют такие технологические процессы, как пайка, сварка, обжатие (опрессовка). 

Пайка обладает рядом недостатков:
\begin{itemize}
\item припои имеют повышенное удельное сопротивление по сравнению с материалами соединяемых проводов;
\item используемые флюсы выделяют органические пары, которые приводят к образованию пленок на соединяемых проводах, что увеличивает сопротивление в месте соединения.
\end{itemize}

При соединении пайкой целесообразно осуществить механическую разгрузку соединяемых электрических частей. Для этого в соединяемых частях делают отверстия, в которые сначала вставляют провода и закручивают и лишь потом их опаивают (рис.~\ref{pic:7razgruzka}). 

\begin{figure}[h!]
	\caption{ Опаивание провода }
	\includegraphics[width=1\textwidth]{7razgruzka.png}
	\label{pic:7razgruzka}
\end{figure}

Сварочное соединение обладает высокой механической прочностью, способностью выдерживать циклические температурные воздействия, обеспечивает высокую плотность монтажа и рекомендуется для применения при разработке микроминиатюрной аппаратуры.

Характерной особенностью неразборных сварных и паяных контактов является отсутствие заметного износа и длительный срок службы.

При опрессовке два провода со снятой с концевых частей изоляцией вводят в соединительную металлическую трубчатую гильзу, которая механически обжимается, в результате чего между проводами через гильзу будет иметь место электрический контакт. Надежность соединения во многом зависит от соотношения размеров гильзы и диаметра провода, усилия обжатия и герметизации места соединения. Размеры и материал гильзы для каждого случая соединения тщательно подбираются экспериментально.

Разборные контакты соединяются болтами, винтами или скруткой (накруткой), а также промежуточными деталями. Наиболее распространенными контактами такого типа являются шины~--- плоские пластины, стянутые болтами. Винтовое соединение является основным видом соединения проводов к электрическим машинам и приборам и позволяет коммутировать провода независимо друг от друга. Медные проводники малых сечений изгибают в кольцо под винт, а чтобы не расходились жилы многожильных проводов, пропаиваются или опрессовываются кольцевыми наконечниками. Материалы проводников и винта различны. Предотвращение возможного ослабления контактного давления при циклических температурных воздействиях и вибрациях возможно введением под винт пружинной шайбы или шайбы-звездочки.

Для подсоединения одножильных проводов используют метод накрутки (рис.~\ref{pic:7nakrutka}~а). Вывод~1 должен иметь не менее двух острых ребер. На нем рекомендуется делать поперечные насечки глубиной около 0,1~мм. Число витков~2 должно лежать в пределах~4$ \ldots $7. Провод накручивают с натягом. При работе соединения в условиях вибрации к обычной намотке добавляют 1$ \ldots $2 витка провода с изоляцией~(рис.~\ref{pic:7nakrutka}~б). 

\begin{figure}[h!]
	\caption{ Метод накрутки }
	\includegraphics[width=0.6\textwidth]{7nakrutka.png}
	\label{pic:7nakrutka}
\end{figure}

Соединение накруткой получают без разогрева материалов путем накручивания под натягом вокруг жесткого вывода нескольких витков одножильного провода. В сечении вывод представляет квадратную или прямоугольную форму с острыми углами. Материал вывода должен быть достаточно прочным, чтобы противостоять скручивающим усилиям, обладать хорошим сопротивлением на сминание накручиваемым проводом и низким омическим сопротивлением. Подобными свойствами обладают фосфористая и бериллиевая бронзы. В качестве материала проводника используется относительно мягкий и пластичный материал, сохраняющий форму накрутки. Соединение обеспечивает высокую надежность при жестких механических и климатических воздействиях.

Основными причинами отказа соединения является ухудшение переходного сопротивления из-за коррозии соединения и появления усталостных явлений в накрученном проводе.

Разъемные контакты могут периодически размыкаться, как правило, при отсутствии тока в цепи. Наиболее распространены разъемные контакты типа штырь-гнездо. Они служат для соединения электрических цепей, которое производят до работы устройства, или для замены электроэлементов. Их применяют также для соединения конструктивно автономных приборных устройств. В этих случаях разъемные контакты выполняют в виде штепсельных разъемов, позволяющих одновременно соединять несколько цепей. 

В неразборных контактах нет физической границы раздела между проводниками, а в разборных эта поверхность есть. Она контролируется сжимающей нагрузкой и способностью материала к пластической деформации. Кроме того, очень важны отсутствие загрязнений на поверхности и ее коррозионная стойкость, поэтому контакты часто покрывают мягкими коррозионно-стойкими материалами (олово, серебро, кадмий и т. д.), а также подвергают очистке различными методами.

\section{Подвижные контакты }

В подвижных контактах, по крайней мере, один из неподвижных компонентов прижимается к подвижному компоненту и отводится от него при замыкании, переключении и размыкании электрической цепи, находящейся под токовой нагрузкой. В зависимости от назначения выделяют коммутирующие (разрывные) и токосъемные (скользящие и катящиеся) контакты. 

Скользящие контакты~--- это совокупность двух перемещающихся относительно друг друга тел, через которые от одного к другому проходит ток. Они обеспечивают непрерывную коммутацию тока между подвижной и неподвижной частями электрических машин, аппаратов и приборов. К скользящим контактам относятся, например, коллектор, кольца и щетки в электрических машинах, обмотки и ползунки в реостатах и потенциометрах. 

Скользящий контакт должен обеспечивать непрерывное замыкание электрической цепи, в противном случае возникают электрошумы, например при регулировке переменных сопротивлений и т.п. Однако достичь непрерывного замыкания цепи скользящим контактом трудно. Объясняется это тем, что при движении по неподвижному контакту подвижный контакт все время соударяется с микронеровностями неподвижной поверхности или с витками резистивной обмотки~(рис.~\ref{pic:7electronoise}). 

\begin{figure}[h!]
	\caption{ Электрошумы при соударении подвижного контакта по неподвижному }
	\includegraphics[width=0.5\textwidth]{7electronoise.png}
	\label{pic:7electronoise}
\end{figure}

При совпадении частоты соударений с частотой собственных колебаний подвижного контакта последний начинает вибрировать. Вероятность разрыва цепи зависит от скорости относительного движения подвижного контакта, шероховатости поверхности, шага намотки проволоки неподвижного контакта, силы прижатия подвижного контакта к неподвижному и геометрических характеристик подвижного контакта, определяющих собственную частоту $ f_0 $ колебаний. Если скользящий контакт применяют для относительно больших токов и напряжений, его работа может сопровождаться эрозионными процессами, возникающими при кратковременных разрывах цепи. Эрозия (разъедание) недопустима в скользящих контактах прецизионной аппаратуры. 

Во всех этих случаях для повышения надежности подвижных контактов используют несколько скользящих контактов, включенных параллельно (рис.~\ref{pic:7par}), так как увеличение числа контактных пятен повышает стабильность контактного сопротивления.

\begin{figure}[h!]
	\caption{ Повышение надежности подвижных контактов }
	\includegraphics[width=1\textwidth]{7par.png}
	\label{pic:7par}
\end{figure}

Работа таких контактов более надежна, так как они построены на принципе повышения надежности параллельным дублированием. Кроме того, таким образом можно улучшить ряд качественных показателей контактного узла. Например, в случае применения нескольких подвижных контактов (мухолапок) разной длины $ l_i $ в потенциометрах (рис.~\ref{pic:7par}~б) можно добиться работоспособности контактного узла в большом диапазоне частот. При вхождении одной из мухолапок в резонанс другие функционируют нормально.

Характерной особенностью скользящих и катящихся электрических контактов (СК) является также их изнашивание в процессе работы, как и в обычных парах трения, с той особенностью, что рабочие нагрузки контактов обычно невелики. Электрическое изнашивание проявляется в переносе материала одного элемента на другой, искрении и дугообразовании, приводящих к резкому ухудшению качества поверхности, что в свою очередь увеличивает скорость механического изнашивания.

Взаимосвязь между фрикционными и электрическими процессами является важной чертой подвижных контактов, так как пятна, через которые передается ток, полностью или частично совпадают с пятнами, воспринимающими механическую нагрузку. Поэтому состояние поверхности раздела и поведение граничных пленок оказывает влияние одновременно как на процессы трения, так и токопрохождения.

Роль граничных пленок. Факторы действия граничного слоя можно разделить на группы, выделив в них позитивные и негативные (рис.~\ref{pic:7factor}).

\begin{figure}[h!]
	\caption{ Факторы действия граничного слоя в скользящем электрическом контакте }
	\includegraphics[width=1\textwidth]{7factor.png}
	\label{pic:7factor}
\end{figure}

Основные пути повышения надежности подвижных контактов. Можно выделить три основных направления повышения эффективности контактов (рис.~\ref{pic:7way}):
\begin{itemize}
\item разработка новых контактных материалов, покрытий и смазок;
\item применение специальных методов воздействия на структуру и состояние поверхности раздела;
\item усовершенствование конструкции контакта.
\end{itemize}

\begin{figure}[h!]
	\caption{ Основные пути повышения надежности подвижных контактов }
	\includegraphics[width=1\textwidth]{7way.png}
	\label{pic:7way}
\end{figure}

Одной из причин возможных разрывов цепи является истирание контактов в процессе работы. Продукты трения (содранная окисная пленка, окисленные частицы материалов контактов), попадая между контактами, могут привести к временному разрыву цепи. Следует учитывать, что для некоторых контактных материалов трение усиливает окисление. Для снижения окисления контакты часто смазывают различными смазками. Смазка уменьшает износ и благоприятствует удалению продуктов износа. Общей чертой методов, связанных с использованием смазок и новых материалов, является стремление к созданию на поверхностях контакта материалов тонких переходных слоев, не влияющих на процесс передачи тока через контакт, но резко снижающих вероятность схватывания, сваривания и интенсивного механического изнашивания. Наиболее известно и разработано применение композиционных материалов, содержащих в составе твердые смазки, обладающие электропроводностью. Кроме того, внимание специалистов, проектирующих различные типы контактов, привлекают жидкие и консистентные (пластичные) контактные смазки.

Коммутирующие (разрывные) контакты работают в прерывистом режиме (многочисленные слаботочные контакты реле и контакты электрических аппаратов в силовых цепях). Они используются при необходимости замыкания или размыкания цепей, находящихся под током. Разрывные контакты используются в исполнительных контактно-коммутационных (ИККУ) устройствах. Работа большого числа электрических приборов (реле, контакторов, выключателей) основана на использовании разрывных контактов. Подробнее ниже.

\section{Исполнительные контактно-коммутационные устройства}

Исполнительные контактно-коммутационные устройства (ИККУ) по типу управления подразделяются на устройства с электромагнитным (электромагнитные реле, герконы) и механическим (микровыключатели, кнопки и др.) управлением. В измерительных и автоматических системах применяют также ИККУ с управлением по теплоэнергетическим параметрам (давление, температура).

\begin{flushleft}
\textbf{Электромагнитное реле (ЭР)}
\end{flushleft}

\textit{Реле} --- электромагнитный переключатель, предназначенный для коммутации электрических цепей (скачкообразного изменения выходных величин) при заданных изменениях электрических или не электрических входных величин. Различают электрические, пневматические, температурные, механические виды реле, но наибольшее распространение получили электрические (электромагнитные) реле.

\textit{Электромагнитное реле} --- это ИККУ с разрывными контактами, скачкообразно срабатывающее при достижении управляющим током или напряжением определенного значения. ЭР состоит из трех основных частей: электромагнитной системы, преобразующей энергию электрического тока в энергию магнитного поля; промежуточного органа (якоря), преобразующего энергию магнитного поля в механическую энергию подвижных частей; контактной системы. Существуют две основные конструктивные схемы реле: с угловым движением якоря (рис.~\ref{pic:7anchor}~а) и поступательно перемещающимся втягивающимся якорем (рис.~\ref{pic:7anchor}~б). 

При прохождении тока через катушку 2 возникает магнитный поток, проходящий через сердечник~1, ярмо~7, якорь~4 и воздушный зазор. Под действием магнитного поля возникает сила, перемещающая якорь. Происходит замыкание контактов~5, закрепленных на контактонесущих пружинах~6. При обесточивании катушки якорь занимает первоначальное положение либо под действием собственного веса (рис.~\ref{pic:7anchor}~б), либо под действием пружины~3 (рис.~\ref{pic:7anchor}~а). Конструкции реле весьма разнообразны. 

Реле характеризуют: током срабатывания $ I_\text{ср} $, при котором происходит замыкание контактов; током отпускания $ I_\text{от} $, при котором происходит размыкание контактов; временем срабатывания $ t_\text{ср} $ от момента подачи напряжения до замыкания контактов и временем отпускания  $ t_\text{от} $ от момента отключения реле до размыкания контактов.

\begin{figure}[h!]
	\caption{ Основные конструктивные схемы реле: а -- с угловым движением якоря, б -- поступательно перемещающимся втягивающимся якорем }
	\includegraphics[width=0.25\textwidth]{7anchor.png}
	\label{pic:7anchor}
\end{figure}

По назначению реле подразделяют на:
\begin{itemize}
\item пусковые -- для ввода в действие различных устройств;
\item максимальные и минимальные - для отключения или включения цепей при токе или напряжении, больших или меньших определеного значения;
\item для создания требуемой выдержки времени при включении цепи.
\end{itemize}

По мощности управления реле $ P_y $ подразделяют на:
\begin{itemize}
\item маломощные ($ \leq 1 \text{Вт} $);
\item средней мощности ($ 1 \ldots 10 \text{Вт} $);
\item мощные ($ \geq 10 \text{Вт} $).
\end{itemize}

По времени срабатывания реле $ t_\text{ср} $ подразделяют на:
\begin{itemize}
\item безынерционные ( $ \leq 0,001 $ с);
\item быстродействующие  ($ 0,005\ldots0,05 $ с);
\item нормальные ($ 0,05\ldots0,15 $ с);
\item замедленные ($ 0,1 \ldots 1,0 $ с).
\end{itemize}

Схемы коммутации реле разнообразны (рис.~\ref{pic:7comm}). Они могут содержать контакты, которые при подаче тока в обмотку катушки замыкаются (условное обозначение <<з>>), размыкаются (условное обозначение -- <<р>>) или переключаются (условное обозначение -- <<п>>). Если на схеме указывается <<1з-2п>>, то это означает, что реле имеет один замыкающий и два переключающих контакта.

\begin{figure}[h!]
	\caption{ Схемы коммутации реле }
	\includegraphics[width=0.2\textwidth]{7comm.png}
	\label{pic:7comm}
\end{figure}

\begin{flushleft}
\textbf{Герконы }
\end{flushleft}

В вычислительной технике и автоматических системах большое распространение получили герметизированные контакты~-- герконы (рис.~\ref{pic:7gerkon} и \ref{pic:7gercon}). Геркон с электромагнитной катушкой составляет герконовое реле.

\textit{Геркон}\footnote{Сокращение от <<герметичный [магнитоуправляемый] контакт>>}~--- электромеханическое устройство, представляющее собой пару ферромагнитных контактов, запаянных в герметичную стеклянную колбу. При поднесении к геркону постоянного магнита или включении электромагнита контакты замыкаются. Герконы используются как бесконтактные выключатели, датчики близости.

\begin{figure}[h!]
	\caption{ Схемы коммутации геркона }
	\includegraphics[width=0.5\textwidth]{7gerkon.png}
	\label{pic:7gerkon}
\end{figure}

\begin{figure}[h!]
	\caption{ Геркон }
	\includegraphics[width=0.7\textwidth]{7gercon.png}
	\label{pic:7gercon}
\end{figure}

Геркон представляет собой герметичную оболочку~1, в которую заключены пластины~2~и~3, выполненные из материала с высокой магнитной  проницаемостью (например, из сплава железа с никелем). На пластинах укреплены контакты 4 и 5. Оболочка находится внутри управляющей катушки~6. При подаче в катушку напряжения магнитный поток замыкается через пластины~2~и~3. В межконтактном промежутке возникает электромагнитная сила. 

При определенном значении этой силы пластины 2 и 3 изгибаются так, что происходит замыкание контактов~4 и 5. При снятии управляющего электрического сигнала контакты размыкаются. Стеклянная оболочка 1 вакуумирована или заполнена инертным газом~-- смесью азота с гелием или водородом. 

Вакуумирование позволяет:
\begin{itemize}
\item применять для контактов износо- и эрозионноустойчивые материалы --- вольфрам, молибден, окисляющиеся в обычных условиях;
\item применять малые контактные силы, так как окисные и газовые пленки на контактах отсутствуют; 
\item использовать малые межконтактные зазоры, так как электрическая прочность вакуума больше, чем у воздуха (при расстоянии 1 мм $ U_\text{вак} $ = 80 кВ, а $ U_\text{возд} $ = 20 кВ).
\end{itemize}

Преимущества вакуумирования контактов позволяют также получить коммутационные устройства с малыми управляющими мощностями (50$ \ldots $150 мВт) и с большой износостойкостью (с числом срабатываний $ 10^7 $\ldots$ 10^9 $). Отсутствие массивных подвижных частей определяет малую инерционность герконов (время срабатывания 0,5\ldots2 мс). 
Малые размеры некоторых конструкций герконов позволяют получить сравнительно малые межконтактные емкости и индуктивности, что важно при коммутации ВЧ-цепей.

Для создания минимума переходного сопротивления контактов, поверхности касания покрывают золотом, радием, палладием или серебром. При отключении тока в обмотке управления (электромагнита) электромагнитная сила исчезает, и под действием сил упругости контакты размыкаются.
Существуют также герконы, размыкающие цепь при возникновении магнитного поля, и герконы с переключающей группой контактов.

Герконы различаются также по конструктивным особенностям. Они бывают сухими (с сухими контактами) и ртутными, в которых капля ртути смачивает контактирующие поверхности, уменьшая их электрическое сопротивление и предотвращая вибрацию пластин в процессе работы.

Можно выделить герконовое реле на ферритах, которое обладает свойством памяти. В таких реле для переключения в катушку необходимо подать импульс тока обратной полярности с целью размагничивания ферритового сердечника. Они называются герметизированными запоминающими контактами или гезаконами.

Гезакон~-- герметизированный запоминающий контакт. Является разновидностью геркона. Отличительной особенностью гезакона является сохранения положения (вкл/выкл) после снятия управляющего магнитного поля.

Сохранение положение после снятия воздействия магнитного поля происходит за счёт того, что у гезакона подвижная часть пружины-контакта изготовлена из магнитного материала с прямоугольной петлёй гистерезиса, обладающего достаточной намагниченностью для удержания контакта в замкнутом состоянии.

Преимущества: 
\begin{itemize}
\item герметизирование позволяет использовать их в любых климатических условиях; 
\item простота конструкции;
\item малая масса и габариты;
\item высокое быстродействие;
\item надежная работа в широком диапазоне температур (-60$^\circ$\ldots+120$^\circ$С).
\end{itemize}

Недостатки:
\begin{itemize}
\item восприимчивость к внешним магнитным полям;
\item хрупкость;
\item малая мощность коммутируемых цепей (возможность саморазмыкания при высоких токах).
\end{itemize}

\begin{flushleft}
\textbf{Микровыключатели}
\end{flushleft}

Микровыключатель (MB) --- это ИККУ разрывного типа с механическим управлением, характеризующееся релейной зависимостью между управляющей силой и ходом приводного элемента и отличающееся практической независимостью времени коммутации (времени срабатывания) от скорости перемещения приводного элемента.

MB имеют малые удельный вес и объем. MB широко используют во многих областях техники. Применение MB в системах автоматики и телемеханики позволяет рассматривать их как элементы автоматических и телемеханических  устройств. Высокие метрологические качества позволяют использовать MB в качестве измерительных преобразователей типа «перемещение~-- электрический сигнал» для работы совместно с чувствительными элементами, воспринимающими различные теплоэнергетические параметры: давление, температуру. Высокая надежность, малые масса и габариты определяют применение MB в электронике, авиации и судостроении.

Любой MB состоит из упругой перекидной системы~2, на которой укреплен один (или более) подвижный контакт~1, неподвижных контактов~4, приводного элемента~3 и корпусных деталей (рис.~\ref{pic:7switcher}). К приводному элементу извне прикладывается механическая сила. Иногда между приводным элементом 3 и управляющей частью прибора помещается вспомогательный элемент~5 для согласования движения управляющей части прибора с движением приводного элемента MB. 

Основными техническими характеристиками MB являются:
\begin{itemize}
\item номинальные значения тока и напряжения;
\item характер нагрузки (омическая, индуктивная);
\item значение коммутируемой мощности;
\item электрическая и механическая износоустойчивость в миллионах циклов, гарантируемых при заданных условиях эксплуатации, степень защищенности от окружающей среды.
\end{itemize}

\begin{figure}[h!]
	\caption{ Схема коммутации микровыключателя }
	\includegraphics[width=0.5\textwidth]{7switcher.png}
	\label{pic:7switcher}
\end{figure}

\begin{flushleft}
\textbf{Переключатели}
\end{flushleft}

Выбор переключателя определяется принятым типом управляющего движения рукоятки (угловое, нажимное или поворотное), задачами, решаемыми при управлении, схемой коммутации и параметрами коммутируемых цепей. Переключатели с угловым перемещением рукоятки типа тумблер выполняют по двум конструктивным схемам: с врубными контактами (рис.~\ref{pic:7switcher1}~а), коромыслового типа (рис.~\ref{pic:7switcher1}~б). 

\begin{figure}[h!]
	\caption{ Переключатели: а -- с врубными контактами, б -- коромыслового типа }
	\includegraphics[width=0.6\textwidth]{7switcher1.png}
	\label{pic:7switcher1}
\end{figure}

В обоих случаях конструкция имеет два устойчивых положения. При перемещении рукоятки~1 пружина~2 сжимается, аккумулируя энергию. При достижении положения, показанного штрих-пунктирной линией, конструкция находится в положении неустойчивого равновесия. При малейшем дальнейшем перемещении происходит резкий переброс ее в устойчивое положение. Подвижный контакт~3 или закрепленный на коромысле~4 контакт~5 скачком соединяются с неподвижным контактом~6.

По схеме коммутации переключатели типа тумблер с врубными контактами (Т-1, Т-2, Т-3) подразделяют на однополюсные (рис.~\ref{pic:7switcher2}~а), однополюсные сдвоенные (рис.~\ref{pic:7switcher2}~б), двухполюсные на два положения (рис.~\ref{pic:7switcher2}~в,~г). Эти переключатели имеют два фиксированных положения рукоятки. Переключатели коромыслового типа могут иметь до трех фиксированных положений рукоятки. Схемы коммутации этих переключателей весьма разнообразны.

\begin{figure}[h!]
	\caption{ Переключатели: а -- однополюсные, б -- сдвоенные, в, г --  двухполюсные на два положения}
	\includegraphics[width=0.7\textwidth]{7switcher2.png}
	\label{pic:7switcher2}
\end{figure}

Рассмотренные коммутирующие устройства являются одними из наиболее надежных элементов приборных устройств. Важное значение имеет повышение их надежности: за счет правильного подбора (режима работы и условий эксплуатации), рациональное размещение (вибрация, температура), строгое соблюдение инструкции по монтажу, дублирование.
