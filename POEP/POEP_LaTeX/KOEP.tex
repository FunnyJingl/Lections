\documentclass[a4paper,oneside,openany]{tufte-book}
\usepackage{amsmath}
\usepackage{amssymb}
%\usepackage{array}
\usepackage[T1]{fontenc}
\usepackage[english, russian]{babel}

\usepackage{cmap}
\usepackage{comment}
\usepackage{bookmark}
%\usepackage{hyperref}
\hypersetup{colorlinks} % uncomment this line if you prefer colored hyperlinks (e.g., for onscreen viewing)
\usepackage{pscyr}

\usepackage[utf8]{inputenc}

\usepackage{indentfirst}

\usepackage{geometry}
% For A4 paper
\geometry{
	left=15mm, % left margin
	textwidth=133mm, % main text block
	marginparsep=8mm, % gutter between main text block and margin notes
	marginparwidth=50mm % width of margin notes
}
% For graphics / images
\usepackage{graphicx}
\setkeys{Gin}{width=\linewidth,totalheight=\textheight,keepaspectratio}
\graphicspath{{pictures/}}

%\usepackage{textcomp}
\usepackage{multirow}
\usepackage{icomma}
\usepackage{enumitem}
\usepackage{verbatim}

% Prints a trailing space in a smart way.
\usepackage{xspace}

%%
% Book metadata
\title[Конструирование оптико-электронных приборов]{Конструирование\breakоптико-электронных\breakприборов}
\author[Городничев В.А., Филимонов П.А.]{Городничев В.А., Филимонов П.А.}
\publisher[Кафедра РЛ-5 <<Элементы приборных устройств>> МГТУ им. Баумана]{Кафедра РЛ-5 <<Элементы приборных устройств>>\breakМГТУ им. Баумана}

%\usepackage{microtype}

\usepackage{lscape}
% Just some sample text
\usepackage{lipsum}

% For nicely typeset tabular material
\usepackage{booktabs}

% The fancyvrb package lets us customize the formatting of verbatim
% environments.  We use a slightly smaller font.
\usepackage{fancyvrb}
\fvset{fontsize=\normalsize}

%%
% Prints argument within hanging parentheses (i.e., parentheses that take
% up no horizontal space).  Useful in tabular environments.
\newcommand{\hangp}[1]{\makebox[0pt][r]{(}#1\makebox[0pt][l]{)}}

%%
% Prints an asterisk that takes up no horizontal space.
% Useful in tabular environments.
\newcommand{\hangstar}{\makebox[0pt][l]{*}}

%%


%%
% Some shortcuts for Tufte's book titles.  The lowercase commands will
% produce the initials of the book title in italics.  The all-caps commands
% will print out the full title of the book in italics.
\newcommand{\vdqi}{\textit{VDQI}\xspace}
\newcommand{\ei}{\textit{EI}\xspace}
\newcommand{\ve}{\textit{VE}\xspace}
\newcommand{\be}{\textit{BE}\xspace}
\newcommand{\VDQI}{\textit{The Visual Display of Quantitative Information}\xspace}
\newcommand{\EI}{\textit{Envisioning Information}\xspace}
\newcommand{\VE}{\textit{Visual Explanations}\xspace}
\newcommand{\BE}{\textit{Beautiful Evidence}\xspace}

\newcommand{\TL}{Tufte-\LaTeX\xspace}

% Prints the month name (e.g., January) and the year (e.g., 2008)
\newcommand{\monthyear}{%
  \ifcase\month\or Январь\or Февраль\or Март\or Апрель\or Май\or Июнь\or
  Июль\or Август\or Сентябрь\or Октябрь\or Ноябрь\or
  Декабрь\fi\space\number\year
}


% Prints an epigraph and speaker in sans serif, all-caps type.
\newcommand{\openepigraph}[2]{%
  %\sffamily\fontsize{14}{16}\selectfont
  \begin{fullwidth}
  \sffamily\large
  \begin{doublespace}
  \noindent\allcaps{#1}\\% epigraph
  \noindent\allcaps{#2}% author
  \end{doublespace}
  \end{fullwidth}
}

% Inserts a blank page
\newcommand{\blankpage}{\newpage\hbox{}\thispagestyle{empty}\newpage}

\usepackage{units}

% Typesets the font size, leading, and measure in the form of 10/12x26 pc.
\newcommand{\measure}[3]{#1/#2$\times$\unit[#3]{pc}}

% Macros for typesetting the documentation
\newcommand{\hlred}[1]{\textcolor{Maroon}{#1}}% prints in red
\newcommand{\hangleft}[1]{\makebox[0pt][r]{#1}}
\newcommand{\hairsp}{\hspace{1pt}}% hair space
\newcommand{\hquad}{\hskip0.5em\relax}% half quad space
\newcommand{\TODO}{\textcolor{red}{\bf TODO!}\xspace}
\newcommand{\ie}{\textit{i.\hairsp{}e.}\xspace}
\newcommand{\eg}{\textit{e.\hairsp{}g.}\xspace}
\newcommand{\na}{\quad--}% used in tables for N/A cells
\providecommand{\XeLaTeX}{X\lower.5ex\hbox{\kern-0.15em\reflectbox{E}}\kern-0.1em\LaTeX}
\newcommand{\tXeLaTeX}{\XeLaTeX\index{XeLaTeX@\protect\XeLaTeX}}
% \index{\texttt{\textbackslash xyz}@\hangleft{\texttt{\textbackslash}}\texttt{xyz}}
\newcommand{\tuftebs}{\symbol{'134}}% a backslash in tt type in OT1/T1
\newcommand{\doccmdnoindex}[2][]{\texttt{\tuftebs#2}}% command name -- adds backslash automatically (and doesn't add cmd to the index)
\newcommand{\doccmddef}[2][]{%
  \hlred{\texttt{\tuftebs#2}}\label{cmd:#2}%
  \ifthenelse{\isempty{#1}}%
    {% add the command to the index
      \index{#2 command@\protect\hangleft{\texttt{\tuftebs}}\texttt{#2}}% command name
    }%
    {% add the command and package to the index
      \index{#2 command@\protect\hangleft{\texttt{\tuftebs}}\texttt{#2} (\texttt{#1} package)}% command name
      \index{#1 package@\texttt{#1} package}\index{packages!#1@\texttt{#1}}% package name
    }%
}% command name -- adds backslash automatically
\newcommand{\doccmd}[2][]{%
  \texttt{\tuftebs#2}%
  \ifthenelse{\isempty{#1}}%
    {% add the command to the index
      \index{#2 command@\protect\hangleft{\texttt{\tuftebs}}\texttt{#2}}% command name
    }%
    {% add the command and package to the index
      \index{#2 command@\protect\hangleft{\texttt{\tuftebs}}\texttt{#2} (\texttt{#1} package)}% command name
      \index{#1 package@\texttt{#1} package}\index{packages!#1@\texttt{#1}}% package name
    }%
}% command name -- adds backslash automatically
\newcommand{\docopt}[1]{\ensuremath{\langle}\textrm{\textit{#1}}\ensuremath{\rangle}}% optional command argument
\newcommand{\docarg}[1]{\textrm{\textit{#1}}}% (required) command argument
\newenvironment{docspec}{\begin{quotation}\ttfamily\parskip0pt\parindent0pt\ignorespaces}{\end{quotation}}% command specification environment
\newcommand{\docenv}[1]{\texttt{#1}\index{#1 environment@\texttt{#1} environment}\index{environments!#1@\texttt{#1}}}% environment name
\newcommand{\docenvdef}[1]{\hlred{\texttt{#1}}\label{env:#1}\index{#1 environment@\texttt{#1} environment}\index{environments!#1@\texttt{#1}}}% environment name
\newcommand{\docpkg}[1]{\texttt{#1}\index{#1 package@\texttt{#1} package}\index{packages!#1@\texttt{#1}}}% package name
\newcommand{\doccls}[1]{\texttt{#1}}% document class name
\newcommand{\docclsopt}[1]{\texttt{#1}\index{#1 class option@\texttt{#1} class option}\index{class options!#1@\texttt{#1}}}% document class option name
\newcommand{\docclsoptdef}[1]{\hlred{\texttt{#1}}\label{clsopt:#1}\index{#1 class option@\texttt{#1} class option}\index{class options!#1@\texttt{#1}}}% document class option name defined
\newcommand{\docmsg}[2]{\bigskip\begin{fullwidth}\noindent\ttfamily#1\end{fullwidth}\medskip\par\noindent#2}
\newcommand{\docfilehook}[2]{\texttt{#1}\index{file hooks!#2}\index{#1@\texttt{#1}}}
\newcommand{\doccounter}[1]{\texttt{#1}\index{#1 counter@\texttt{#1} counter}}

% Generates the index
\usepackage{makeidx}
\makeindex

\begin{document}

% Front matter
\frontmatter

% r.3 full title page
\maketitle


% v.4 copyright page
\newpage
\begin{fullwidth}
~\vfill
\thispagestyle{empty}
\setlength{\parindent}{0pt}
\setlength{\parskip}{\baselineskip}
Copyright \copyright\ \the\year\ \thanklessauthor

\par\smallcaps{Кафедра РЛ-5 <<Элементы приборных устройств>> МГТУ им. Баумана}

\par Лекции по курсу <<Конструирование оптико-электронных приборов>> предназначены для потока студентов 2 курса кафедры РЛ-2 <<Лазерные и оптико-электронные системы>> МГТУ им. Баумана.

\par\textit{\monthyear}
\end{fullwidth}

% r.5 contents
\tableofcontents

\listoffigures

\listoftables
%%
% Start the main matter (normal chapters)
\mainmatter

\chapter{Оптико-электронные приборы}
\label{ch:tufte-design}

\newthought{Оптико-электронными приборами}\marginnote{\allcaps{ОПТИКО-ЭЛЕКТРОННЫЕ ПРИБОРЫ}} называются приборы, в которых информация об исследуемом или наблюдаемом объекте переносится оптическим излучением (содержится в оптическом сигнале), а ее первичная обработка сопровождается преобразованием этого излучения (оптического сигнала) в электрическую энергию (в электрический сигнал). 

В\marginnote{Далее может встречаться также термин <<оптические приборы>>, что будет подразумевать --- <<оптические приборы, содержащие в своем составе механические, электронные и оптические функциональные устройства и элементы>>, т.е. фактически --- ОЭП.} состав этих приборов входят как оптические, так и электронные звенья, причем и те и другие выполняют основные функции данного прибора, а не являются вспомогательными устройствами (например, узлами подсветки отсчетных шкал, устройствами термостабилизации).

ОЭП является сложной системой, включающей в себя большое число различных по своей физической природе и принципу действия звеньев -- аналоговых и цифровых преобразователей электрических сигналов, микропроцессоров, оптических, механических и электромагнитных узлов и др. Поэтому ОЭП часто называют оптико-электронными системами (ОЭС). 

Учитывая большое разнообразие ОЭП и их широкое применение в самых различных областях науки и техники  в курсе лекций рассмотрены общие для большинства ОЭП вопросы проектирования, достаточно общие и часто используемые на практике методы расчета и выбора основных параметров ОЭП, особенности конструкции и методы расчета параметров типовых узлов ОЭП.

Помимо исследуемого объекта (<<полезный>> излучатель) на рис.~\ref{pic:1OEPscheme} показаны и возможные на практике <<вредные>> излучатели (фоны, помехи). Взаимное расположение звеньев может быть и несколько иным. Отдельные звенья на практике представляют собой весьма сложные устройства, например, в состав источника излучения могут входить передающая оптическая система, фильтры, модулятор. Иногда в состав ОЭП не входят некоторые из перечисленных звеньев. Это определяется, как правило, методом работы прибора.

Все ОЭП предназначены для получения информации об объектах окружающей среды, переносимой оптическими сигналами. Хорошо известны ОЭП, используемые для локации, исследования природных ресурсов, измерения оптических свойств различных объектов. Многие ОЭП работают в составе следящих систем, используемых в навигации и ориентации, в системах технического зрения, устройствах автоматического контроля и управления, системах управления летательными аппаратами, системах наведения и во многих других устройствах для измерения линейных, угловых величин и определения координат объектов. Определенной спецификой обладают оптико-электронные системы противодействия и подавления оптических схем противника.

\begin{figure}[h!]
	\begin{center}
		\includegraphics[width=0.8\linewidth]{OEPscheme.png}
		\caption{Обобщенная схема ОЭП}
		\label{pic:1OEPscheme}
	\end{center}
\end{figure}

Главный элемент --- оптическая система (ее сложно разработать, обеспечить нужное качество), которая определяется оптической схемой. С оптической схемы начинается разработка ОЭП.

\textsc{Оптической схемой}\marginnote{\allcaps{ОПТИЧЕСКАЯ СХЕМА}} называется графическое представление процесса изменения света в оптической системе.

Различия в принципах работы звеньев ОЭП, в способах обработки сигналов, проходящих через них, а также разнообразие условий эксплуатации ОЭП обусловливают сложность и многоступенчатость процесса проектирования этих приборов и требуют тщательного анализа как условий работы ОЭП, так и состояния имеющейся в распоряжении разработчика элементной базы.

\section{Классификация ОЭП}

Классификация ОЭП возможна по широкому кругу признаков в зависимости от принципов построения приборов и характера их применения (рис.~\ref{pic:1Classification}). К числу таких признаков могут быть отнесены параметры оптического сигнала, метод измерений, спектральный диапазон работы, режим работы, степень автоматизации, вид измерений, назначение и область применения, условия эксплуатации.

\begin{figure*}[h]
	\includegraphics[width=\linewidth]{OEPclassification.png}
	\caption{Классификация ОЭП}
	\label{pic:1Classification}
\end{figure*}

Как известно, в ОЭП носителем полезной информации служит оптический сигнал в виде потока излучения, являющийся функцией координат $(x, y, z)$ измеряемого объекта, спектрального состава излучения ($\lambda$), времени ($t$), состояния и положения плоскости поляризации излучения ($A_\text{п}$), т.е. $\Phi(x, y, z, \lambda, t, A_\text{п})$. 

По методу работы ОЭП с учетом особенностей их построения и возможности управления параметрами излучения ОЭП делят на:
\begin{itemize}
	\item активные (наблюдаемый объект освещается (с помощью передающей оптической системы) лазером, при этом часть отраженного излучения поступает на вход ОЭП);
	\item полуактивные (один источник освещает все объекты);
	\item пассивные (используется собственное излучение исследуемого объекта; тепловое  (отраженное) от других источников излучения (солнца, луны); рассеянное излучение атмосферы и подстилающей поверхности).
\end{itemize}

В зависимости от спектрального состава используемого излучения ОЭП подразделяют на приборы, работающие в следующих областях спектра:
\begin{enumerate}
	\item ультрафиолетовая (УФ)~--- 200--380~нм ;
	\item видимая --- 400--700~нм;
	\item инфракрасная (ИК) --- 700 нм--1~мм.
\end{enumerate}

В свою очередь, УФ делится на: УФ-А (400--315 нм), УФ-Б (315--280 нм), УФ-С: (280--10 нм). Излучение с длиной волны менее 200 нм называют вакуумным ультрафиолетом, так как излучение не распространяется из-за полного поглощения в атмосфере.

ИК диапазон условно разделяют на: ИК-А или ближний ИК (700--1400 нм), ИК-Б или средний ИК (1400--3000 нм), ИК-С или дальний ИК (3 мкм--1 мм).

По степени автоматизации различают ОЭП:
\begin{enumerate}
	\item автоматические, работающие без участия оператора (обычно в следящем режиме);
	\item полуавтоматические, функционирование которых частично зависит от действий оператора;
	\item неавтоматические, выходная информация которых рассчитана на восприятие оператором.
\end{enumerate}

Существенные различия в принципах построения, функционирования и обслуживания имеют ОЭП, работающие в различных условиях эксплуатации:
\begin{enumerate}
	\item лабораторные;
	\item цеховые;
	\item полевые;
	\item бортовые.
\end{enumerate}

Наиболее емким из приведенных признаков классификации является назначение (область применения). Практически невозможно найти область техники, где бы в настоящее время не применялись ОЭП. Поэтому в схеме классификации указаны только некоторые области техники, в которых применение ОЭП является решающим фактором их дальнейшего развития: 
\begin{enumerate}
	\item навигация (гирометры);
	\item геодезия (теодолит);
	\item астрофизика (телескопы);
	\item робототехника (система машинного зрения);
	\item телевизионная техника (подсветка и формирование изображения);
	\item медицина (микроскопы);
	\item контрольно-измерительная техника (интерферометры);
	\item военная техника (оптические прицелы).
\end{enumerate}

ОЭП внутри каждой из рассмотренных классификационных групп могут подразделяться по конструктивным, функциональным и иным признакам. Кроме того, между всеми классификационными признаками существуют прямые и косвенные связи. Например, контрольно-измерительные приборы могут быть угломерными, автоматическими, цеховыми.

\section{Основные критерии оценки качества ОЭП}

\newthought{Качеством прибора}\marginnote{\allcaps{КАЧЕСТВО ПРИБОРА}} называется совокупность свойств прибора, обуславливающих его пригодность удовлетворять определенные потребности в соответствии с его назначением. Для объективной оценки качества прибора его свойства характеризуют количественно~--- показателями качества.

\noindent
\textsc{Показатели качества}\marginnote{\allcaps{ПОКАЗАТЕЛИ КАЧЕСТВА}}~--- это комплекс показателей, используемых для оценки свойств прибора, а также решений, принимаемых на различных этапах проектирования. 
Оценка качества ОЭП связана с рассмотрением широкого круга показателей, представленных на рис.~\ref{pic:1QualityOEP}.

\begin{figure*}[h]
	\includegraphics[width=0.7\textwidth]{1QualityOEP.png}
	\caption{Схема оценки качества ОЭП}
	\label{pic:1QualityOEP}
\end{figure*}

Всесторонняя оценка современных изделий может быть выполнена лишь при комплексном учете всех указанных показателей. Вместе с тем при проектировании разработчики чаще всего оценивают качество будущего прибора по показателям функционирования, надежности и технологичности.

\newthought{Показатели функционирования}\marginnote{\allcaps{ПОКАЗАТЕЛИ\break ФУНКЦИОНИРОВАНИЯ}} являются основными, они характеризуют техническую сущность прибора, и именно поэтому они стоят на первом месте в техническом задании.

Ввиду большого разнообразия ОЭП показатели функционирования могут быть самыми различными. Достаточно обобщенными являются информационные характеристики, к которым относят:
\begin{enumerate}
	\item входной язык, посредством которого осуществляется связь прибора с наблюдаемым или контролируемым объектом;
	\item энергия, необходимая для формирования единицы информации;
	\item функция преобразования, описывающая зависимость информативного параметра выходного сигнала от информативного параметра входного сигнала при номинальных значениях неинформативных параметров;
	\item выходной язык, посредством которого осуществляется связь прибора с потребителем информации;
	\item скорость выдачи информации прибором и восприятия ее потребителем (быстродействие).
\end{enumerate}

Наряду с перечисленными к показателям функционирования могут быть отнесены также вид потребляемой энергии и мощность потребления, габаритные размеры и масса прибора.

\newthought{Надёжность}\marginnote{\allcaps{НАДЁЖНОСТЬ}} определяется как свойство объекта сохранять во времени в установленных пределах значения всех параметров, характеризующих способность выполнить требуемые функции в заданных режимах и условиях применения, технического обслуживания, ремонта, хранения и транспортирования.

Сложность ОЭП, включающих оптические, механические и электронные узлы, требования к работоспособности этих приборов в резко изменяющихся условиях эксплуатации ставят перед конструктором задачу~--- создать прибор, обладающий высокой надежностью в течение всего срока службы.

Надёжность прибора зависит от количества и качества входящих в него элементов, условий работы (температуры, влажности, механических воздействий), схемного и конструктивного выполнения прибора, технологии изготовления и качества материала элементов.

\newthought{Технологичность}\marginnote{\allcaps{ПОКАЗАТЕЛИ\break ТЕХНОЛОГИЧНОСТИ}} деталей, узлов и конструкций, удобство сборки может быть охарактеризована следующими показателями: 
\begin{itemize}
	\item минимальными затратами труда на изготовление;
	\item минимальным ассортиментом средств изготовления;
	\item минимумом сложных и трудоемких производственных процессов;
	\item простотой подготовки производства;
	\item минимальным числом операций и временем их проведения;
	\item правильным выбором допусков на изготовление;
	\item простота монтажа деталей в узлы без дополнительной обработки;
	\item законченность узлов, входящих в прибор;
	\item простота сборки прибора в целом.
\end{itemize}

\textsc{Рациональный выбор материалов}: материалы, необходимые для изготовления деталей, следует выбирать с учетом не только функциональных и эксплуатационных особенностей прибора, но и технологии его изготовления. Для единичного производства целесообразно использовать материалы, хорошо поддающиеся обработке резанием. При крупносерийном и массовом производстве более экономичны способы изготовления без снятия стружки, что и определяет в значительной степени выбор материалов.

\textsc{Минимальная номенклатура элементов, материалов, полуфабрикатов} упрощает снабжение производства. 
Кроме того, необходимо иметь в виду, что некоторые детали и элементы часто не соответствуют специфике и профилю предприятия. В этих случаях целесообразнее идти по пути кооперации с другими предприятиями, чем осваивать производство соответствующих изделий.

\textsc{Обеспечение взаимозаменяемости деталей, узлов и блоков} предполагает идентичность конструктивных и присоединительных размеров, соединителей, а также входных и выходных параметров. Взаимозаменяемость позволяет обеспечить замену одного узла или блока другим без дополнительной подгонки и регулирования. Это обстоятельство имеет важное значение при сборке приборов, особенно при крупносерийном и массовом производстве, а также при обслуживании и ремонте приборов. Прежде всего необходимо стремиться к взаимозаменяемости электронных узлов и блоков. Взаимозаменяемость обеспечивается рациональными допусками на размеры и параметры узлов и блоков.

\textsc{Максимальная нормализация и унификация конструкций} основана на применении нормализованных, унифицированных или стандартизованных деталей и узлов. Нормализованные детали включены в нормаль данного предприятия или группы родственных предприятий. Унифицированные детали применяются на предприятиях всей отрасли промышленности. Стандартизованные детали используются на предприятиях различных отраслей промышленности.

Унифицированные и стандартизованные детали, узлы и блоки изготовляются централизованно, что позволяет автоматизировать процесс их производства, обеспечить высокую надежность и минимальную стоимость. Показатели унификации и стандартизации характеризуют степень использования и применения в данном приборе стандартизованных, унифицированных и заимствованных узлов и деталей. Чем больше таких элементов будет в проектируемом приборе, тем меньше затраты на их конструирование, технологическую подготовку производства, выше, как правило, надежность функционирования, проще организовать обслуживание и ремонт.

\textsc{Обеспечение возможности изготовления деталей при единичном и мелкосерийном производстве на универсальном оборудовании} имеет смысл при изготовлении уникальных и экспериментальных приборов, для выпуска которых в единичных образцах или малыми сериями нецелесообразно делать специальную технологическую оснастку. Повысить качество таких приборов и уменьшить технологические и трудовые затраты на их изготовление можно путем использования типовых узлов и деталей, о которых говорилось выше.

\textsc{Простота и удобство выполнения сборки, монтажа и юстировки} имеет особое значение для качественной настройки прибора, как в заводских условиях, так и в процессе дальнейшего использования. При этом снижаются трудовые затраты и требования к уровню подготовки производственного и обслуживающего персонала, а также требования к сложности юстировочного и стендового оборудования.

\newthought{Эстетические показатели}\marginnote{\allcaps{ЭСТЕТИЧЕСКИЕ\break ПОКАЗАТЕЛИ}} характеризуют внешний вид прибора, его соответствие современному стилю, гармоничность сочетания отдельных элементов прибора друг с другом, соответствие формы прибора его назначению, качество и совершенство отделки внешних элементов, поверхностей и упаковки, выразительность и качество надписей, знаков, технической документации (проспекта, каталога, инструкции, паспорта).

\newthought{Патентно-правовые показатели}\marginnote{\allcaps{ПАТЕНТНО-ПРАВОВЫЕ ПОКАЗАТЕЛИ}} характеризуют степень новизны заложенных в ОЭП технических решений а также вопросы патентно-правовой  защиты и определяются патентоспособностью и патентной чистотой. Патентоспособным является решение, которое может быть признано изобретением в одной или нескольких странах. Патентной чистотой обладают решения, не попадающие под действие (не нарушающие прав) других патентов.

\newthought{Показатели техники безопасности}\marginnote{\allcaps{ПОКАЗАТЕЛИ ТЕХНИКИ БЕЗОПАСНОСТИ}} характеризуют степень защищенности людей и животных от опасного воздействия ОЭП (защита от электрического удара, электромагнитных полей, теплового воздействия, радиации, оптических излучений, шума, токсичных и газовых выделений, вибраций), а также самих приборов от климатических, механических, биологических и других воздействий на них. Такими показателями, например, являются категория и класс исполнения и эксплуатации.

\newthought{Экономические показатели}\marginnote{\allcaps{ЭКОНОМИЧЕСКИЕ\break ПОКАЗАТЕЛИ}} выражаются прежде всего в стоимости прибора. К основным факторам, определяющим стоимость прибора, относятся область применения, условия эксплуатации, технологичность конструкции, требования по надежности, серийность выпуска, стоимость материалов и комплектующих изделий, простота и удобство обслуживания, юстировок и ремонта. 

Экономические показатели характеризуют уровень затрат на производство и эксплуатацию ОЭП. Среди них выделяют полную себестоимость и оптовую цену прибора. 

\newthought{Эргономические показатели}\marginnote{\allcaps{ЭРГОНОМИЧЕСКИЕ\break ПОКАЗАТЕЛИ}} характеризуют степень приспособленности прибора к взаимодействию с человеком с позиции удобства работы, гигиены, безопасности труда. 

Эргономические показатели разделены на гигиенические (уровень шума, амплитуда и частота вибраций, уровень радиации, температура, степень загазованности, токсичности), антропометрические (размеры и расположение экранов, индикаторов, рукояток, наглазников, налобников, форма сиденья), психофизиологические (диапазоны усилий на рукоятках, скорость выполнения движений, уровень освещенности, цвет и яркость световых сигналов, тембр и сила звуковых сигналов), психологические (объем и интенсивность потока информации, количество и частота выполняемых операций, количество и расположение контрольных, сигнальных, управляемых элементов).

\newthought{Экологические показатели}\marginnote{\allcaps{ЭКОЛОГИЧЕСКИЕ\break ПОКАЗАТЕЛИ}} характеризуют степень вредного влияния на окружающую среду и ее загрязнение при изготовлении, эксплуатации и утилизации приборов.

\newthought{Передаточные системы ОЭП}\marginnote{\allcaps{ПЕРЕДАТОЧНЫЕ\break СИСТЕМЫ ОЭП}} выделяются по виду преобразуемого сигнала: механическая, оптическая и электронная.


\chapter{Зубчатые элементарные передачи}
\label{ch:gears}

\newthought{Механическая система ОЭП} выполняет функцию позиционирования различных элементов оптической системы в пространстве с заданной точностью.
Вид движения оптических элементов зависит от их назначения, но в большинстве случаев можно выделить вращательное и поступательное движения.
Передача энергии в механической системе ОЭП осуществляется различными механизмами.

\noindent
\textsc{Механизм}\marginnote{\allcaps{МЕХАНИЗМ}} --- система тел, предназначенная для преобразования движения одного или нескольких твердых тел в требуемое движение других твердых тел.

\noindent
\textsc{Кинематическая пара}\marginnote{\allcaps{КИНЕМАТИЧЕСКАЯ ПАРА}} --- соединение двух соприкасающихся звеньев, допускающих их относительное движение.

\noindent
\textsc{Кинематическая цепь}\marginnote{\allcaps{КИНЕМАТИЧЕСКАЯ ЦЕПЬ}} --- система звеньев, связанных между собой кинематическими парами.

\noindent
\textsc{Зубчатая элементарная передача}\marginnote{\allcaps{ЗУБЧАТАЯ\break ЭЛЕМЕНТАРНАЯ\break ПЕРЕДАЧА}} --- механизм, состоящий из зубчатых колёс, назначение которого --- передача вращательного движения между валами, обычно с изменением скоростей вращения или направления и характера движения.

\noindent
ГОСТ 16530-83: \textsc{зубчатое колесо}\marginnote{\allcaps{ЗУБЧАТОЕ КОЛЕСО}}~--- зубчатое звено с замкнутой системой зубьев, обеспечивающее непрерывное движение другого зубчатого звена.

\begin{marginfigure}
	\includegraphics[width=0.6\linewidth]{osi1.png}
\end{marginfigure}

Классификация передач осуществляется по различным признакам, основные из которых:
\begin{itemize}
	\item передаточное отношение: постоянное или переменное;
	\item характер движения: меняет или не меняет направление;
	\item взаимное расположение осей валов: параллельное, пересекающиеся и скрещивающиеся;
	\item характер изменения скорости (замедление или ускорение);
	\item вид зубьев колес: прямозубые, косозубые, с внешним или внутренним зацеплением.
\end{itemize}

\begin{marginfigure}
	\includegraphics[width=0.6\linewidth]{osi2.png}
\end{marginfigure}

За начальную, предшествующую зубчатой, можно принять передачу посредством двух соприкасающихся, вращающихся без проскальзывания окружностей диаметра $ d_1 $ и $ d_2 $.
В этом случае обеспечивается постоянное передаточное отношение $ i_{12}=\omega_1 / \omega_2 = d_1 / d_2 $. 
В подобных реальных фрикционных передачах требуется большая сила прижатия цилиндров.

\begin{figure*}[h!]
	\includegraphics[width=\textwidth]{ZK_nachalo.png}
	\caption{Начальные окружности}
	\label{pic:ZK_nachalo}
\end{figure*}

\begin{marginfigure}
	\includegraphics[width=0.7\linewidth]{osi3.png}
\end{marginfigure}

\begin{marginfigure}
	\includegraphics[width=0.8\linewidth]{osi4.png}
\end{marginfigure}

\begin{marginfigure}
	\includegraphics[width=0.8\linewidth]{osi6.png}
\end{marginfigure}

Проскальзывание можно полностью устранить, если на ободах цилиндров нарезать зубья.
В процессе работы зубья ведущего колеса \textsc{1} входят во впадины, давят на зубья колеса 2 и поворачивают его, преодолевая полезный момент сопротивления и силы трения в опорах и между зубьями.
При наличии зубьев сила прижатия колёс, как у фрикционных передач, не нужна, поэтому значительно сокращается давление на опоры.
При идеально точном изготовлении зубчатая передача должна воспроизводить такую же передачу вращения, как и посредством соприкасающихся окружностей.

Такие соприкасающиеся окружности, которые мысленно можно представить существующими в сечении цилиндрической зубчатой передачи плоскостью, перпендикулярной их осям, называют начальными.
Диаметры эти окружностей обозначаются $ d_{w1} $ и $ d_{w2} $.
Передаточное отношение такой зубчатой передачи:
\begin{equation*}
i_{12} = \dfrac{n_1}{n_2} = \dfrac{d_{w1}}{d_{w2}} = \dfrac{z_2}{z_1},
\end{equation*}
\noindent
где $ n_1,\,n_2 $ -- частоты вращения колёс, $ \text{мин}^{-1} $; $ z_1,\,z_2 $ -- числа зубьев колёс.

ГОСТ 16530-83: передаточное число --- передаточное отношение, выраженное через числа зубьев колёс.

В реальной передаче передаточное отношение не является постоянным.
Оно колеблется за полный цикл изменения взаимного положения колёс около номинального значения, определяемого отношением чисел зубьев колёс передачи.
Отклонения передаточного отношения зависят от правильного выбора формы профилей зубьев колёс и неизбежных погрешностей изготовления зубчатых передач.

Исходным требованием к форме профилей зубьев является получение постоянства передаточного отношения в процессе зацепления зубьев колёс. 
Для обеспечения этого требования форма профиля зуба должна определяться в соответствии с основной теоремой зацепления.

\newthought{Основная теорема зацепления}\marginnote{\allcaps{ОСНОВНАЯ\break ТЕОРЕМА\break ЗАЦЕПЛЕНИЯ}}: общая нормаль к профилям зубьев колес, проведенная через точку касания профилей, делит межосевое расстояние колес на отрезки, пропорциональные угловым скоростям вращения.


\begin{figure*}[h!]
	\includegraphics[width=0.6\textwidth]{OTZ.png}
	\caption{Основная теорема зацепления}
	\label{pic:OTZ}
\end{figure*}

Известна угловая скорость $ \omega_1 $ зубчатого колеса 1, а следовательно, и окружные скорости точек профиля его зуба, в том числе и точки $ K $ касания профилей зубьев, $ v_{1K} = \omega_1 O_1 K $. 
Для точки $ K $ профиля зуба ведомого колеса известно направление окружной скорости $ v_{2K} $, оно перпендикулярно радиусу $ O_2 K $. Из очевидного условия, что проекции скоростей соприкасающихся точек $ K $ профилей зубьев колёс 1 и 2 на общую нормаль $ n-n $ должны быть одинаковы, т.е. $ v_{1n} = v_{2n}$, получаем:
\begin{equation*}
\omega_1 O_1 N_1 = \omega_2 O_2 N_2, \text{или}\, i=\dfrac{\omega_1}{\omega_2} = \dfrac{O_2 N_2}{O_1 N_1}.
\end{equation*}


Так как $ \triangle O_1 N_1 P \propto O_2 N_2 P $, то $ \dfrac{O_2 N_2}{O_1 N_1} = \dfrac{O_2 P}{O_1 P}$. Отсюда следует, что
\begin{equation*}
i=\dfrac{\omega_1}{\omega_2} = \dfrac{O_2 N_2 }{O_1 N_1} = \dfrac{O_2 P}{O_1 P}.
\end{equation*}

Для получения постоянного передаточного отношения на всём участке зацепления зубьев необходимо, чтобы $ i=\dfrac{\omega_1}{\omega_2} = \dfrac{O_2 P}{O_1 P} = const $. 
Таким образом, при передаче зацеплением общая нормаль к профилям зубьев в любой точке их касания при повороте колёс должна проходить через одну и ту же точку $ P $, которая делит межосевое расстояние $ a_w $ на отрезки, обратное отношение которых $ \dfrac{O_2 P}{O_1 P} $ равно передаточному отношению $ i=\dfrac{\omega_1}{\omega_2} $. Профили зубьев колёс передачи называют сопряжёнными, если они соответствуют основной теореме зацепления.

Определённым\marginnote{\allcaps{СКОЛЬЖЕНИЕ ПРОФИЛЕЙ}} недостатком зубчатых зацеплений является \allcaps{скольжение профилей зубьев}. Скорость скольжения $ v_\text{ск} $ профилей зубьев равна разности проекций скоростей контактирующих точек зубьев на направление касательной $ t-t $ к профилям в точке их касания: 
\begin{equation}
\label{eq:gears_skolzhenie}
v_\text{ск}=v_{1t} - v_{2t} = -PK (\omega_1 + \omega_2).
\end{equation}

\begin{marginfigure}
	\includegraphics[width=0.6\linewidth]{osi5.png}
\end{marginfigure}

Из выражения \ref{eq:gears_skolzhenie} следует, что чем дальше от полюса $ P $ происходит контакт профилей зубьев, тем больше скольжение профилей.
Скольжение профилей отсутствует, когда точка их касания находится в полюсе $ P $. 
Полюс $ P $ является также мгновенным центром качения начальных окружностей, которые всегда касаются в полюсе. Это окружности диаметров $ d_{w1}, \, d_{w2} $. 
Таким образом:
\begin{equation*}
i=\dfrac{O_2 P}{O_1 P} = \dfrac{d_{w2}}{d_{w1}}.
\end{equation*}

\noindent
Следовательно, сопряжённые профили воспроизводят такую же передачу движения, как и исходные начальные окружности при их относительном качении без проскальзывания.

Наиболее распространённым профилем зубьев колёс, отвечающим требованиям основной теоремы зацепления, является эвольвента окружности.

\noindent
\textsc{Эвольвента}\marginnote{\allcaps{ЭВОЛЬВЕНТА}} --- кривая, представляющая собой траекторию движения любой точки прямой, перекатывающейся без скольжения по окружности.

\begin{figure}[h!]
	\includegraphics[width=0.8\textwidth]{evolventa1.png}
	\caption{Эвольвентный профиль зубчатого колеса}
	\label{pic:evolventa1}
\end{figure}

\begin{table}[ht]
	\centering
	\fontfamily{ppl}\selectfont
	\begin{tabular}{ll}
		\toprule
		Обозначение & Параметр \\ 
		\midrule
		$ m $ & модуль зацепления, [мм] \\
		$ z_1,\,z_2 $ & количество зубьев \\
		$ d_1,\,d_2 $ & диаметр делительной окружности, [мм] \\
		$ p=s+e $ & шаг зубьев, [мм] \\
		$ s $ & толщина зубьев, [мм] \\
		$ e $ & расстояние между профилями зубьев, [мм] \\
		$ d_a $ & диаметр окружности впадин, [мм] \\
		$ d_a $ & диаметр окружности вершин, [мм] \\
		$ h $ & высота зуба, [мм] \\
		$ h_f $ & высота головки зуба, [мм] \\
		$ \tau $ & центральный угол делительной окружности \\
		$ b $ & наименьшее расстояние между торцами зубьев \\
		\bottomrule
	\end{tabular}
	\caption{Основные параметры зубчатого колеса}
	\label{tab:parZ}
\end{table}


\newthought{Достоинства} прямозубой передачи:
\begin{itemize}
	\item высокое передаточное отношение (до 15);
	\item надежность и простота обслуживания;
	\item технологичность;
	\item малые габариты;
	\item высокий КПД (до 0,99);
	\item постоянство передаточного отношения $ i $;
	\item применение в широком диапазоне вращающих моментов, скоростей и передаточных отношений;
	\item малые нагрузки на валы и опоры;
	\item долговечность (до 50000 ч).
\end{itemize}

\newthought{Недостатки} прямозубой передачи:
\begin{itemize}
	\item высокие требования к точ­ности изготовления и монтажа и, как следствие, дороговизна;
	\item шум при работе со значительными скоростями;
	\item высокая жесткость, не позволяющая компенсировать динамические нагрузки;
	\item невозможность безступенчатого регулирования передаточного отношения.
\end{itemize}

\textsc{Методы нарезания зубьев}\marginnote{\allcaps{МЕТОДЫ НАРЕЗАНИЯ ЗУБЬЕВ}}:
\begin{itemize}
	\item метод деления (копирования);
	\item метод обката (огибания).
\end{itemize}

Для\marginnote{\allcaps{КОРРИГИРОВАННЫЕ\break ЗУБЧАТЫЕ\break ПЕРЕДАЧИ}} устранения подрезания зубьев применяют \textsc{корригированные зубчатые передачи}.
Корригирование определяется смещением режущего инструмента $ x = \xi m $ при нарезании зубьев:
\begin{itemize}
	\item отсутствие коррекции (нулевая передача): $ \xi_1 = \xi_2 = 0, \, \xi_\Sigma=0 $;
	\item высотная коррекция (равносмещённая передача): $ \xi_1 = -\xi_2 \neq 0, \, \xi_\Sigma=0 $;
	\item угловая коррекция, при которой происходит изменение угла зацепления: $ \xi_1 \neq \xi_2, \, \xi_\Sigma \neq 0 $, различают положительные ($\xi_\Sigma > 0 $) и отрицательные ($\xi_\Sigma < 0 $) передачи.
\end{itemize}

\marginnote{\allcaps{МАТЕРИАЛЫ ЗК}}

\section{Расчёт на прочность цилиндрических эвольвентных зубчатых колёс}

\section{Точность зубчатых колёс и передач}

\newthought{Точные приборные устройства}\marginnote{\allcaps{ТОЧНЫЕ ПРИБОРНЫЕ УСТРОЙСТВА}}~--- приборные устройства, точность работы которых регламентируется допусками.\marginnote{Допуск на точность обозначается $ [\delta_0 S] $}

Расчеты на точность подразделяются на:
\begin{itemize}
	\item прямой (проектный) -- определение точностных требований к составляющим устройствам, узлам и деталям;
	\item обратный (проверочный) -- определение погрешности ПУ на основе разработанных точностных требований к звеньям устройства.
\end{itemize}

Существует 12 степеней точности. В приборостроении обычно применяют 6-9 степени точности.

\newthought{Показатели точности} регулируются стандартами (контрольные комплексы):
\begin{itemize}
	\item \allcaps{кинематическая точность}\marginnote{\allcaps{КИНЕМАТИЧЕСКАЯ\break ТОЧНОСТЬ}} --- наибольшая погрешность функции положения при работе передачи в одном направлении или наибольшая погрешность $ i $;
	\item \allcaps{плавность работы}\marginnote{\allcaps{ПЛАВНОСТЬ РАБОТЫ}} --- плавность изменения кинематической погрешности – колебания скорости за один оборот, источник динамической нагрузки;
	\item \allcaps{пятно контакта}\marginnote{\allcaps{ПЯТНО КОНТАКТА}} --- полнота прилегания зубьев и концентрация нагрузки на их поверхности;
	\item \allcaps{боковой зазор}\marginnote{\allcaps{БОКОВОЙ ЗАЗОР}} между работающими профилями зубев --- для компенсации температурных деформаций, смазки, погрешностей сборки и изготовления; боковой зазор нормируется независимо от степени точности зубчатых колёс и передач, определение которого основано на величине минимального гарантированного бокового зазора $ j_{n\,min} $.
\end{itemize}

Для всех видов передач предпочтительными являются функциональные показатели  и суммарное пятно контакта.

\section{Проектный расчёт зубчатых передач на прочность}
\subsection{Виды разрушений зубчатых колёс}
Основными видами разрушения зубчатых колёс являются излом от напряжений изгиба в материале зубьев и выкрашивание рабочих поверхностей зубьев от контактных напряжений, если в обоих случаях напряжения превосходят допускаемые значения. 
Износ от контактных напряжений является характерным для зубчатых колёс, находящихся в масле (закрытые передачи).
Масло, находящееся в месте контакта зубьев, под давлением заполняет поверхностные микротрещины в зубьях, вызывая постепенное разрушение.

\begin{marginfigure}
	\includegraphics[width=1\linewidth]{polomkaZ1.png}
\end{marginfigure} 
\textsc{Выкрашивание} поверхностных слоев зубьев характерно для закрытых хорошо смазываемых передач. 
При циклическом нагружении на поверхности зубьев у полюсной линии разрастаются микротрещины, что приводит к образованию оспинок, переходящих в раковины. Выкрашивание может быть ограниченным или прогрессирующим.

\begin{marginfigure}
	\includegraphics[width=0.6\linewidth]{polomkaZ.png}
\end{marginfigure}
\textsc{Поломка зубьев} может носить усталостный характер или являться следствием значительных перегрузок. При циклическом нагружении микротрещины у корня зуба разрастаются, что приводит к излому по сечению у основания зуба прямозубых колёс или по косому сечению – косозубых и шевронных колёс.

\begin{marginfigure}
	\includegraphics[width=0.6\linewidth]{polomkaZ2.png}
\end{marginfigure}
\textsc{Абразивный износ} характерен для открытых передач, а также закрытых, работающих при скудной смазке и наличии абразивов.

\begin{marginfigure}
	\includegraphics[width=0.6\linewidth]{polomkaZ3.png}
\end{marginfigure}
\textsc{Заедание} характерно для высоконагруженных передач. При высокой удельной нагрузке происходит разрыв масляной плёнки, нагрев и схватывание сопряжённых поверхностей с образованием следов задира в направлении скольжения зубьев.

Цель проектного расчёта зубчатых передач на прочность~--- определение модуля зацепления и размеры передач, обеспечивающие их работоспособность в течение заданного срока службы. 

В зависимости от вида разрушения и условий работы передачи необходимо проводить расчёт на изгибную прочность и расчёт зубьев на контактную прочность.

Расчётные формулы содержат ряд коэффициентов.
Эти коэффициенты, общие для расчёта на изгибную прочность и на контактные напряжения, обозначают $ K $.
Коэффициенты, характерные только для расчёта на изгиб, обозначают $ Y $.
Чтобы показать, что расчёт на изгибную прочность проводится по опасному сечению, находящемуся в основании ножки зуба, применяют букву $ F $, т.е. $ Y_F $.
Коэффициенты, характерные только для расчёта по контактным напряжениям, обозначают $ Z $.
Индекс $ H $ у коэффициентов $ Z_H $ при расчёте на контактную прочность введён в честь Герца.
\subsection{Исходные соотношения между окружными и нормальными силами и давлениями в прямозубых эвольвентных зубчатых передачах}
Формулы для расчёта на прочность зубчатых колёс содержат моменты сил, выраженные через окружные силы.
Изгибные и контактные напряжения рассчитывают по нормальным силам и давлениям.
Для вывода расчётных формул необходимо учитывать соотношения между окружными и нормальными силами и давлениями.
Значения нормальных сил и давлений определяют по моменту нагрузки $ M_2 $, приложенному к ведомому колесу:
\begin{equation}
F_n = \dfrac{2 M_2 K}{d_2 \cos\alpha}.
\end{equation}



\section{Косозубые зубчатые колёса}
\newthought{Косозубые колёса}\marginnote{\allcaps{КОСОЗУБЫЕ КОЛЁСА}} применяют для увеличения плавности хода и при повышенных нагрузках вместо прямозубых.

\newthought{Достоинства}:
\begin{itemize}
	\item высокий коэффициент торцевого перекрытия , что обеспечивает высокую плавность хода, повышенную прочность, снижение шума, уменьшение динамических нагрузок
	\item возможность подбора заданного межосевого расстояния, когда известно передаточное отношение и задан стандартный модуль, но нет возможности подобрать прямозубое колесо
	\item возможность работы при повышенных окружных скоростях~(до~30~м/с)
\end{itemize}

\newthought{Недостатки}: появление осевой силы, которая может быть компенсирована использованием шевронной передачи.

\chapter{Коническая передача}
\label{ch:conic}

Предназначены для передачи движения между двумя пересекающимися осями (валами). Угол пересечения между осями $ \Sigma $ составляет, как правило, $ 90^\circ $.

Боковые поверхности конического колеса образованы перекатывающейся без скольжения плоскости, касающейся основания конуса. При перекатывании любая точка лежит на образующей конуса.

Делительная окружность конического колеса --- окружность, получаемая в пересечении делительного конуса и внешнего дополнительного конуса; к этой делительной окружности относится и выбираемый СТ СЭВ 310-76 внешний окружной делительный модуль $ m_e $. \marginnote{Индекс $ e $~-- к внешнему диаметру,\break $ i $~-- к внутреннему, $ m $~-- для параметров, относящихся к профилю зуба в нормальной к его направлению плоскости, проходящей через середину зуба, $ a $~-- к вершине зуба, $ f $~-- к впадине зуба}
\chapter{Червячная передача}
\label{ch:chervak}

Назначение червячной передачи --- передача вращательного движения между скрещивающимися осями вращения.

Червячная передача состоит из червяка и червячного колеса. Червяк --- одно- или многовитковый винт, боковые поверхности витков которого являются винтовыми. Червячное колесо~--- косозубое зубчатое колесо, угол наклона зубьев которого равен углу подъема витков червяка.

\newthought{Основные параметры}\marginnote{\allcaps{ОСНОВНЫЕ ПАРАМЕТРЫ}} червячных передач и её элементов:
\begin{table}[ht]
	\centering
	\fontfamily{ppl}\selectfont
	\begin{tabular}{ll}
		\toprule
		Обозначение & Параметр \\ 
		\midrule
		$ m = \dfrac{p}{\pi}$ & осевой модуль ГОСТ 2144-76, [мм] \\
		$ q= \dfrac{d_1}{m} $ & коэффициент диаметра червяка СТ СЭВ 267-76 \\
		$ \alpha = 20^\circ $ & стандартный угол профиля \\
		$ z_1 $ & число заходов червяка (1, 2 или 4) \\
		$ s $ & ход витка червяка \\
		$ p $ & делительный шаг червяка \\
		$ \gamma = arctg(\dfrac{z_1}{q}) $ & угол подъема линии червяка \\
		$ d_{a1,2} = d_{1,2} + 2m $ & диаметр окружности вершин, [мм] \\
		$ d_{f1} = m(q-2,4) $ & диаметр впадин, [мм] \\
		$ b_1 = 2m\sqrt{z_2} + 1 $ & длина нарезамеого червяка, [мм] \\
		$ a = 0,5m(q+z_2) $ & межосеовое расстояние\\
		\bottomrule
	\end{tabular}
	\label{tab:parChervak}
\end{table}

\newthought{Свойство самоторможения передачи}\marginnote{\allcaps{САМОТОРМОЖЕНИЕ}}~--- свойство, при котором червячное колесо при отсутствии вращения червяка ведомый вал затормаживается, таким образом его невозможно повернуть. самоторможение начинается проявляться при передаточном отношении 35. 

Самоторможение: статическое и динамическое. Статическое самоторможение может быть нейтрализовано ударными нагрузками. Динамическое самоторможение оценивается временем торможения привода после отключения питания двигателя. Полное самоторможение при $ \gamma < 3,5^\circ $. Свойство самоторможения в случае отсутствия ударных нагрузок может быть использовано в качестве тормозящего устройства.

\newthought{Достоинства}:
\begin{itemize}
	\item большое передаточное отношение (от 7 до 200 (теор. 500));
	\item малые габариты;
	\item эффект самоторможения ведомого червячного колеса;
	\item плавность хода;
	\item бесшумность работы.
\end{itemize}

\newthought{Недостатки}:
\begin{itemize}
	\item меньший по сравнению с зубчатыми КПД $ \eta=0,6\ldots0,9 $;
	\item необходимость применения для выполнения колёс дорогих антифрикционных материалов (бронз);
	\item повышенные требования к точности изготовления и монтажа;
	\item значительные осевые силы, действующие на опоры червяка и усложняющие конструкцию опор.
\end{itemize}

\chapter{Передача винт-гайка}
\label{ch:vint-gaika}

Назначение передачи винт-гайка --- преобразование вращательного движения в поступательное.

Передача движения осуществляется с помощью винта, представляющего собой цилиндр с наружной резьбой, и гайки в виде кольца с внутренней резьбой.

Передача винт-гайка подразделяется на: кинематическую и силовую.

Характеристики передачи зависят от типа используемой резьбы. Основные виды резьб, используемые в передаче винт-гайка: метрическая, трапецеидальная и прямоугольная.
Резьба характеризуется:
\begin{itemize}
	\item вид;
	\item шаг $ P $;
	\item число заходов $ z $;
	\item ход винта $ t = z P $;
	\item наружный диаметр $ d $ --- номинальный диаметр;
	\item внутренний диаметр $ d_1 $;
	\item средний диаметр $ d_2 $;
	\item угол подъема резьбы $ \gamma $;
	\item функция перемещения $ l = \dfrac{\varphi t}{2\pi} $.
\end{itemize}

Использование дополнительной гайки на винте позволяет реализовать дифференциальную и интегральную схему перемещений, которые позволяют увеличить точность и чувствительность к повороту винта соответственно.
Функция перемещения:
\begin{itemize}
	\item дифференциальная $ l=\dfrac{\varphi (p_1 - p_2)}{2\pi} $;
	\item интегральная $ l=\dfrac{\varphi (p_1 - p_2)}{2\pi} $.
\end{itemize}


В передаче винт-гайка\marginnote{\allcaps{КИНЕМАТИЧЕСКИЕ И\break СИЛОВЫЕ\break СООТНОШЕНИЯ}} винт в большинстве случаев является ведущим.

При ведущем винте:
\begin{itemize}
	\item $ i = \dfrac{1}{tg \gamma}$ -- передаточное отношение;
	\item $ F = F_a tg(\gamma + \rho') $ -- окружное усилие, которое приложено по касательной к окружности среднего диаметра $ d_2 $;
	\item $ M_k = F_a d_2 tg(\gamma + \rho')$;
	\item $ \eta_\text{в} = \dfrac{tg \gamma}{tg(\gamma + \rho')} $ -- КПД.
\end{itemize}

При ведущей гайке:
\begin{itemize}
	\item $ F = F_a tg(\gamma - \rho') $;
	\item $ M_k = \dfrac{F_a d_2 tg(\gamma - \rho')}{2}$;
	\item $ \eta_\text{г} = \dfrac{tg(\gamma - \rho')}{tg \gamma} $ -- КПД.
\end{itemize}

\newthought{Достоинства}:
\begin{itemize}
	\item большой выигрыш в силе;
	\item высокая точность перемещений;
	\item малые размеры;
	\item возможность обеспечения самоторможения;
	\item сравнительно высокий КПД;
	\item высокая жесткость
	\item малый износ в сравнении с передачами скольжения.
\end{itemize}

\newthought{Недостатки}:
\begin{itemize}
	\item низкий КПД в передачах скольжения;
	\item невозможность получения больших скоростей поступательного движения;
	\item сложность и дороговизна изготовления.
\end{itemize}

\chapter{Расчёт электромеханического привода}
\label{ch:EMP}

Порядок расчёта:
\begin{enumerate}
	\item Выбор электродвигателя
	\begin{enumerate}
		\item Возможные для применения типы двигателей.
		\item Выбор конкретного двигателя для ЭМП.
	\end{enumerate}
	\item Разработка кинематических схем механизмов.
	\item Силовой расчёт.
	\item Расчёт на прочность.
	\item Геометрический расчёт и конструирование ЭМП.
\end{enumerate}

\section{Выбор электродвигателя}

\newthought{Механическая характеристика}\marginnote{\allcaps{МЕХАНИЧЕСКАЯ\break ХАРАКТЕРИСТИКА}} $ \omega = \omega (M) $ --- показывает степень изменения скорости вращения при изменении нагрузки.
Качество механической характеристики оценивается её жёсткостью $ \alpha = - \dfrac{\Delta M}{\Delta \omega} $:
\begin{enumerate}
	\item абсолютно жесткая характеристика ($ \alpha = \infty $~-- угловая скорость не зависит от момента нагрузки на валу);
	\item жёсткая характеристика (угловая скорость меняется незначительно);
	\item мягкая характеристика (угловая скорость меняется значительно).
\end{enumerate}

\noindent
Каждый ЭД обладает свойством \textsc{саморегулирования}\marginnote{\allcaps{САМОРЕГУЛИРОВАНИЕ}} --- ЭД всегда развивает момент, соответствующий моменту нагрузки.

\newthought{Регулировочные характеристики}\marginnote{\allcaps{РЕГУЛИРОВОЧНЫЕ\break ХАРАКТЕРИСТИКИ}}~--- зависимости угловой скорости $ \omega $ от значения (или фазы) напряжения управления $ U_\text{у} $ при постоянных моменте нагрузки на валу и напряжении возбуждения, т.е. $ \omega = \omega (U_\text{у}) $ при $ M_\text{н} = const $, $ U_\text{в} = const $. Регулировочные характеристики необходимы для исполнительных двигателей, работающих в следящих системах. Показатель качества регулировочной характеристики -- её нелинейность.

\newthought{Мощность}\marginnote{\allcaps{МОЩНОСТЬ}} ЭД:
\begin{itemize}
	\item входная $ P_\text{вх} $ --- мощность, потребляемая обмотками двигателя из питающей сети;
	\item выходная $ P $ --- полезная механическая мощность на валу ЭД;
	\item номинальная мощность нерегулируемых ЭД --- мощность при номинальном моменте нагрузки или номинальном значении угловой скорости;
	\item номинальная мощность исполнительных ЭД --- мощность при номинальном значении сигнала управления;
	\item мощность управления --- мощность, потребляемая цепями управления.
\end{itemize}

\noindent
\textsc{Номинальная угловая скорость}\marginnote{\allcaps{НОМИНАЛЬНАЯ УГЛОВАЯ СКОРОСТЬ}} $ \omega_\text{ном} $~--- угловая скорость, которую ЭД развивает при номинальном значении момента нагрузки $ M_\text{ном} $.

\noindent
\textsc{Угловая скорость холостого хода}\marginnote{\allcaps{УГЛОВАЯ СКОРОСТЬ\break ХОЛОСТОГО ХОДА}} $ \omega_0 $~--- угловая скорость, которую ЭД развивает при отсутствии нагрузки.

\noindent
\textsc{Пусковой момент}\marginnote{\allcaps{ПУСКОВОЙ МОМЕНТ}} $ M_\text{п} $~--- момент, который ЭД во время пуска.

Другими важными параметрами, которые необходимо учитывать при выборе ЭД, являются:
\begin{itemize}
	\item КПД;
	\item номинальное значение напряжения питания и частоты питающего тока $ f $;
	\item напряжение трогания исполнительных двигателей~--- напряжение управления, при котором начинается вращение вала ЭД;
	\item электромеханическая постоянная времени двигателя;
	\item диапазон регулирования скорости;
	\item коэффициенты управления по моменту, скорости, мощности;
	\item передаточная функция двигателя;
	\item прочие: масса, габариты, стоимость, момент инерции ротора и т.п.
\end{itemize}

\section{Выбор схемотехнического состава ЭМП}
Требуемое передаточное отношение кинематических цепей $ i_0 $ может быть реализовано с помощью разных схемотехнических элементов.
Эффективность разрабатываемого ЭМП зависит от того, насколько рационально выбраны схемотехнические элементы.

При выборе схемы и схемотехнического состава ЭМП учитывают: 
\begin{itemize}
	\item закон, вид (вращательное, поступательное, сложное) и характер (непрерывное, реверсивное, с остановками) движения выходного звена;
	\item общие передаточные отношения цепей ЭМП;
	\item параметры нагрузки;
	\item требуемая точность;
	\item заданная компоновочная схема ЭМП;
	\item условия эксплуатации и долговечности;
	\item технологичность;
	\item экономические факторы.
\end{itemize}

Так как система выбора схемы и схемных элементов пока не разработана, то разработчик анализирует исходные данные и определяет те, которые являются наиболее критичными для рассматриваемого задания и ранжирует их по степени важности.
Затем с помощью метода перебора известных элементарных передач определяются необходимые типы передач и схемных элементов ЭМП.
Если возникает неоднозначность в выборе схемных элементов, то следует проанализировать целесообразность их выбора с учетом дополнительных требований (например, стоимости, технологичности конструкции, КПД, точности). 
Для намеченных схемных элементов следует назначить передаточное отношение.

После выбора и назначения передаточных отношений схемных элементов определяют произведение передаточных отношений этих элементов $ \prod{i_i} $ и сравнивают его с передаточным отношением рассматриваемой цепи $ i_0 $ и в случае необходимости добавляют (уменьшают) некоторое число схемных элементов или увеличивают (уменьшают) ранее назначенные передаточные отношения элементарных передач, исходя при этом из необходимости выполнения условия:
\begin{equation*}
i_0 = \prod_{i=1}^{n}i_i.
\end{equation*}

Отдельные ступени механизма должны работать так же, как и весь механизм в целом -- на замедление или ускорение, что обеспечивает меньшее число ступеней, а следовательно, меньшие габариты и мёртвый ход привода.

При выборе схемных элементов решается задача рационального размещения их в кинематической цепи, при этом учитывают требования компоновки привода, точности, КПД, габаритов, долговечности и т.д. 
В зубчато-червячном ЭМП размещение червячной передачи в начале кинематической цепи нецелесообразно из-за её более низкого КПД по сравнению с зубчатыми передачами. 
В редукторах элементы с большими передаточными отношениями размещают в конце кинематической цепи. При разработке ЭМП с несколькими выходными валами с целью уменьшения числа схемных элементов и упрощения конструкции рекомендуется частичное или полное совмещение кинематических цепей. Для ряда ПУ необходим учёт направления перемещения выходного вала.
\include{Potentiometer}
\include{IEMM}
\chapter{Муфты}

\newthought{Муфты}\marginnote{\allcaps{МУФТЫ}} --- устройства, предназначенные для соединения концов валов или для соединения валов с расположенными на них деталями. 

Основное назначение муфт --- передача вращающего момента без изменения его модуля и направления. Муфты могут выполнять и другие функции: предохранять механизм от перегрузок, компенсировать несоосность валов, разъединять или соединять валы во время работы и др.

Муфты: 
\begin{itemize}
	\item соединительные:
	\begin{itemize}
		\item глухие (втулочные);
		\item пальцевые (поводковые);
		\item эластичные пальцевые;
		\item упругие муфты с винтовыми пружинами сжатия;
		\item мембранные муфты;
		\item крестовые муфты;
	\end{itemize}
	\item предохранительные:
	\begin{itemize}
		\item с разрушаемыми элементами;
		\item самоуправляемые.
	\end{itemize}
\end{itemize}

\section{Соединительные муфты}

\newthought{Глухие муфты}\marginnote{\allcaps{ГЛУХИЕ МУФТЫ}} предназначются для жёсткого соединения двух валов:
\begin{itemize}
	\item можно скомпенсировать только продольные смещения при отсутствии поперечных смещений и смещений по углу;
	\item малые габариты в радиальном направлении;
	\item динамические нагрузки не демпфируются;
	\item конструкция неразборная;
	\item расчет муфт производится по срезу штифта;
	\item посадка втулки на вал с зазором $ \dfrac{H9}{d9} $.
\end{itemize}

\newthought{Поводковые муфты}\marginnote{\allcaps{ПАЛЬЦЕВЫЕ\break (ПОВОДКОВЫЕ)\break МУФТЫ}}:
\begin{itemize}
	\item компенсирует несоосность (до 0,5 мм) и небольшие продольные смещения и перекосы;
	\item проста в конструкции и эксплуатации;
	\item недостаток --- наличие люфта (зазора) между пальцем и пазом, что приводит к увеличению мёртвого хода всего механизма;
	\item расчёт муфты --- на срез штифтов и на срез пальца;
	\item передаточное отношение не остаётся постоянным вследствие перекоса и поперечного смещения;
	\item материалы полумуфты и пальца: сталь 45 и 45Х;
	\item величину зазора между	пальцем и пазом назначают из условия климатического	исполнения по ГОСТ 15150-69.
\end{itemize}

\newthought{Эластичные пальцевые муфты}\marginnote{\allcaps{ЭЛАСТИЧНЫЕ\break ПАЛЬЦЕВЫЕ\break МУФТЫ}}:
\begin{itemize}
	\item по свойствам идентична пальцевым;
	\item снижает динамические нагрузки в механизмах вследствие деформации упругого промежуточного диска;
	\item материал полумуфты и пальца --- конструкционные стали, материал диска – кожа либо резина;
	\item недостатки – большой упругий мёртвый ход и усиленный износ упругого диска, поэтому не применяются в точных кинематических цепях.
\end{itemize}

\newthought{Упругие муфты с винтовыми пружинами сжатия}\marginnote{\allcaps{УПРУГИЕ МУФТЫ С\break ВИНТОВЫМИ\break ПРУЖИНАМИ СЖАТИЯ}}:
\begin{itemize}
	\item применяют при необходимости демпфирования больших ударных нагрузок;
	\item конструкция: 2 полумуфты + 2 пружины сжатия + стопорное кольцо.
\end{itemize}

\newthought{Мембранные муфты}\marginnote{\allcaps{МЕМБРАННЫЕ МУФТЫ}}:
\begin{itemize}
	\item жёсткое неразборное соединение валов;
	\item компенсируют несовпадение длины валов, несоосность и перекос;
	\item могут иметь малый упругий мёртвый ход;
	\item используются в кинематических цепях средней и высокой точности;
	\item материалы: полумуфты из конструкционных сталей, упругие элементы – из материалов для изготовления пружин;
	\item расчёты --- на устойчивость упругих элементов;
	\item мембраны применяются редко по причине высокой жёсткости;
	\item вместо этого применяются либо части мембран, либо другие конструкции.
\end{itemize}

\newthought{Крестовые муфты}\marginnote{\allcaps{КРЕСТОВЫЕ МУФТЫ}}: безлюфтовый вариант крестовой муфты применяют в цепях самой высокой точности.

\section{Предохранительные муфты}
\begin{itemize}
	\item передаточное отношение предохранительных муфт $ 1\ldots \infty $ в зависимости от принципа предохранения и защиты механизмов при различных перегрузках, запрещённых направлениях движения, превышении скоростей;
	\item применяют для предотвращения выхода из строя при различных видах перегрузки: статическим и динамическим моментом, при повышении или уменьшении допустимой скорости вращения, изменения направления вращения;
	\item предохранительные муфты: самоуправляемые (без разрушения элементов) и неуправляемые (с разрушением элементов).
\end{itemize}


\chapter{Упругие чувствительные элементы}

\newthought{Упругими элементами} называют детали, упругие деформации которых полезно используются в работе различных механизмов и устройств приборов, аппаратов, информационных машин. 
Упругие чувствительные элементы~(УЧЭ) подразделяют на два класса~--- стержневые пружины (плоские, спиральные и винтовые) и оболочки.

Классификация УЧЭ по назначению:
\begin{itemize}
	\item измерительные;
	\item натяжные;
	\item заводные;
	\item пружины кинематических устройств;
	\item пружины амортизаторов;
	\item разделители сред;
	\item токоведущие упругие элементы;
	\item пружины фрикционных и храповых муфт.
\end{itemize}

Основные характеристики УЧЭ:
\begin{itemize}
	\item упругая характеристика;
	\item чувствительность пружин;
	\item жёсткость;
	\item жёсткость системы соединения пружин;
	\item индекс винтовой пружины;
	\item материал;
	\item упругий гистерезис и упругое последействие;
	\item погрешности.
\end{itemize}

Основные параметры винтовых пружин:
\begin{itemize}
	\item диаметр проволоки $ d $;
	\item средний диаметр $ d_\text{ср} $;
	\item угол подъема витков $ \alpha $
	\item число рабочих витков $ i_\text{р} $;
	\item высота пружины $ h_0 $;
	\item шаг витка $ t $;
	\item зазор между витками $ s $.
\end{itemize}

Проектирование винтовых пружин, как правило, производится в следующей последовательности:
\begin{enumerate}
	\item Выбор проволоки.
	\item Расчёт $ d_\text{ср} $, $ d $ и $ i_\text{р} $.
	\item Расчёт на устойчивость.
	\item Определение нелинейности.
\end{enumerate}

\chapter{Проектирование ОЭП}
\newthought{Проектирование}\marginnote{\allcaps{ПРОЕКТИРОВАНИЕ}}~--- разработка проектной, конструкторской и другой технической документации, предназначенной для создания новых видов и образцов продукции промышленности.

\noindent
\textsc{Цель проектирования}\marginnote{\allcaps{ЦЕЛЬ ПРОЕКТИРОВАНИЯ}}~--- разработка нового изделия.

В процессе проектирования происходит поиск вариантов создания оптико-электронных приборов, его возможных конструкций, разработка и уточнение схем, теоретическое и экспериментальное исследование характеристик предполагаемых инженерных решений.

\noindent
\textsc{Конструирование}\marginnote{\allcaps{КОНСТРУИРОВАНИЕ}} является составной частью проектирования и заключается в разработке конкретного варианта изделия на основе проведенных предварительных исследований. При этом создается конструкция проектируемого изделия: устройство, состав, взаимное расположение частей и элементов, способ их соединения и взаимодействия с учетом используемых материалов, технологии изготовления.

В процессе проектирования выпускают чертежи сборочных единиц и деталей, схемы, рассчитывают допуски на погрешности и технологию изготовления и сборки деталей, устанавливают технические условия на прибор, составляют техническое описание, разрабатывают другую конструкторскую документацию, необходимую для изготовления и эксплуатации изделия.

\section{Уровни проектирования}
Разработка сложных систем, какими являются ОЭП, проводится в определенной последовательности.

Отправной точкой создания любой системы являются выбор и формулировка цели проектирования. Необходимость создания нового изделия определяется как развитием конкретного направления техники, так и запросами потребителей (научных и производственных учреждений, человека-оператора). 

Обоснование исходных данных требует учета назначения системы, основных видов ее взаимодействия с другими системами или подсистемами, если она является подсистемой, входящей в состав другой более крупной системы, влияния внешних факторов.

В результате указанного рассмотрения должна быть получена полная совокупность исходных данных для проектирования прибора. 

Результатом проделанной работы является техническое задание~(ТЗ) на прибор, после утверждения которого можно переходить к собственно проектированию.

Различают следующие основные уровни проектирования:
\begin{itemize}
	\item информационно-логический;
	\item системотехнический;
	\item схемотехнический;
	\item конструкторский;
	\item технологический.
\end{itemize}

Первые три уровня иногда объединяют в единый функциональный, или схемный уровень проектирования.

В\marginnote{\allcaps{ИНФОРМАЦИОННО-ЛОГИЧЕСКИЙ УРОВЕНЬ}} процессе проектирования на информационно-логическом уровне определяется конкретная структура данного прибора, определяются связи функциональных устройств между собой и устанавливаются требования технических заданий на проектирование отдельных функциональных устройств, исходя из требований ТЗ на прибор в целом. ТЗ на проектирование того или иного устройства содержит требования к сигналам, информации и командам, вырабатываемым этим устройством.

Таким образом, проектирование на этом уровне состоит из определения сначала структуры проектируемого объекта, а затем в определении оптимальных значений параметров этой структуры, т.е. составляющих ее элементов.

На\marginnote{\allcaps{СИСТЕМОТЕХНИЧЕСКИЙ УРОВЕНЬ}} системотехническом уровне функционального проектирования производится проектирование отдельных функциональных устройств, т.е. процесс разбивается на отдельные ветви. 
Каждое из функциональных устройств рассматривается здесь как структура, состоящая из взаимосвязанных функциональных блоков. 

Процесс проектирования заключается в определении оптимального состава и параметров блоков, например, оптической системы, приемника излучения, электронного тракта, системы отображения.

Все эти отдельные блоки рассматриваются на этом уровни как преобразователи сигналов, безотносительно к их внутреннему устройству. Здесь определяются требования к преобразованию сигналов тем или иным блоком, т.е. к его передаточным и прочим характеристикам.

На\marginnote{\allcaps{СХЕМОТЕХНИЧЕСКИЙ УРОВЕНЬ}} схемотехническом уровне производится проектирование отдельных блоков, входящих в состав функциональных устройств, в соответствии с техническими заданиями, определенными на предыдущем уровне.

Каждому блоку соответствует своя ветвь, причем, начиная с этого уровня, различные ветви имеют различную «специализацию» в соответствии с физической природой блоков, игнорируемой на предыдущем уровне.

Схемотехнический уровень является важнейшим при функциональном проектировании. 
В настоящее время он занимает наибольший объем работы и именно на этом уровне определяются основные параметры различных схем прибора, в конечном итоге обеспечивающие правильную работу прибора в соответствии с техническим заданием. 
Например, на этом уровне выделяется оптическая ветвь и производится расчет оптической системы прибора.

Целью проектирования оптической системы на этом уровне является определение как ее структуры, т.е. количества входящих в нее элементов и их типов, так и численных значений параметров этих элементов.

На электронной ветви схемотехнического уровня производится проектирование электронных схем блоков, преобразующих сигналы. 
Здесь, как и на оптической ветви, определяется структура схемы, т.е. состав и соединения ее функциональных элементов (резисторов, конденсаторов, транзисторов, интегральных схем), а затем и значения их параметров.

На механической ветви производятся аналогичные действия по проектированию кинематической схемы какого-либо устройства прибора.

Таким образом, в процессе схемотехнического проектирования разработчик определяет элементную базу будущего прибора.

Как показывает практика, очень часто проектирование новых элементов на этом уровне не требуется, и работа сводится к подбору элементов из имеющихся стандартных или покупных.

Рассмотренные уровни функционального проектирования являются типичными для ОЭП средней сложности. 
В более простых случаях некоторые уровни могут исключаться, например, информационно-логический или системотехнический.

Конструкторское\marginnote{\allcaps{КОНСТРУКТОРСКИЙ\break УРОВЕНЬ}} проектирование, или просто конструирование, идет обычно параллельно функциональному проектированию или с некоторым отставанием и является важнейшей ветвью процесса проектирования, поскольку именно здесь оптико-электронный прибор приобретает не только схемную, но и материальную (правда пока только в документации) реализацию. 
Разработчик, работающий на этом уровне, называется обычно просто конструктором. 

В большинстве проектных организаций эти два уровня проектирования выполняются разными людьми и даже разными подразделениями. 
Так, например, проектирование оптической системы (оптической схемы) прибора выполняет обычно оптик-расчетчик или оптик-вычислитель, работающий в специализированном оптическом вычислительном бюро. 
Результатом этого проектирования является оптический выпуск, содержащий всю необходимую информацию об оптической схеме, включая ее параметры и их допустимые отклонения. 

На основании этой информации другой разработчик --- конструктор оптик-механик --- выполняет конструирование соответствующего оптического узла, например, объектива, диафрагм, механизма фокусировки объектива. 
Он выпускает чертежи всех деталей этого объектива, включая оптические сборочные чертежи отдельных узлов и объектива в целом.

Естественно, что этот процесс может быть итерационным. 
Так, в случае, если конструктору никак не удается надежно закрепить какую-либо оптическую деталь из-за неудачных с конструктивной точки зрения ее параметров, например, слишком крутых радиусов кривизны, приходится возвращаться на ветвь функционального проектирования и пересчитывать оптическую схему с изменением ее параметров.

Аналогичная картина наблюдается для электронных и кинематических схем. 
После того, как они разработаны на уровне функционального проектирования, конструктор материализует эти схемы в виде определенного монтажа на печатной плате, в виде деталей и узлов механизма.

Конструирование, также как и функциональное проектирование, разделяется на уровни.

Верхний уровень -- это компоновочный, на котором определяется общая компоновка всего прибора, взаимное расположение его отдельных узлов.

Один или несколько следующих уровней, в зависимости от сложности прибора --- это уровни узлов (сборочных единиц), где разрабатываются конструкции отдельных частей прибора. Сразу за компоновочным уровнем процесс конструирования может разделяться на ветви, соответствующие различным узлам, например: механическим, оптико-механическим, электронным или электромеханическим узлам. 
И, наконец, последний уровень --- это уровень деталей, на котором разрабатываются и выпускаются рабочие чертежи отдельных деталей.

На\marginnote{\allcaps{ТЕХНОЛОГИЧЕСКИЙ\break УРОВЕНЬ}} уровне технологического проектирования производится разработка технологических процессов изготовления прибора. 
Как и на других стадиях разработки, здесь выделяются различные уровни:
\begin{itemize}
	\item верхний уровень -- испытание прибора (методики испытаний);
	\item уровень юстировки прибора (достижение верного взаиморасположения элементов);
	\item уровень сборки всего прибора;
	\item техпроцессы сборки, юстировки, контроля сборочных единиц;
	\item техпроцессы изготовления деталей. 
\end{itemize}

Верхним уровнем является уровень испытаний прибора, на котором разрабатываются методики испытаний прибора на соответствие различным пунктам ТЗ.

Следующим идет уровень юстировки, где разрабатываются методики юстировки прибора.

Затем уровень сборки всего прибора, который разветвляется по отдельным узлам (сборочным единицам). На этих уровнях разрабатываются техпроцессы сборки, юстировки и контроля различных сборочных единиц прибора. Наконец, на низших уровнях разрабатываются технологические процессы изготовления отдельных деталей.

Результатами работы на ветви технологического проектирования являются технологические карты, методики юстировки и контроля.

\section{Проектирование с использованием системного подхода}
Сущность системного подхода состоит в том, что объект проектирования рассматривается как система, т.е. как единство взаимосвязанных элементов, которые образуют единое целое и действуют в интересах реализации единой цели. 

Системный подход включает в себя выявление структуры системы, типизацию связей, определение свойств (атрибутов) системы, анализ влияния внешней среды, он требует рассматривать каждый элемент системы во взаимосвязи и взаимозависимости с другими элементами, вскрывать закономерности, присущие данной конкретной системе, выявлять оптимальный режим ее функционирования. 

Системный подход проявляется прежде всего в попытке создать целостную картину исследуемого или управляемого объекта. 
Исследование или описание отдельных элементов при этом производится с учетом роли и места элемента во всей системе.

Процесс функционирования сложной системы происходит на многих уровнях. 
Система расчленяется на подсистемы, которые представляют собой компоненты, необходимые для существования и действия системы.

Методическим средством реализации системного подхода к проектированию служит системный анализ, под которым понимается совокупность приемов и методов исследования объектов (процессов) посредством представления их в виде систем и их последующего анализа.

Системный анализ предполагает системный подход и к изучению связей между элементами, между подсистемами и системой.

В основе системного подхода лежат следующие основные принципы и положения:
\begin{enumerate}
	\item Принцип цели: при разработке конструкции исходят из необходимости учета требований и показателей, которые должны быть реализованы. Должна быть ясна цель проектирования.
	\item Принцип целостности: объект рассматривается как единая система, состоящая из устройств, сборочных единиц элементов, функциональных устройств.
	\item Иерархичность строения: всякая система допускает разделение на подсистемы, что приводит к ступенчатости конструкции. Выявление и представление иерархии структуры объекта проводится с целью установления связей между частями объекта.
	\item Необходимо обобщение опыта и оценка перспектив развития систем данного или близких классов.
	\item Всестороннее рассмотрение взаимодействия системы с внешней средой и учет основных видов взаимодействия элементов и узлов внутри системы.
	\item Выбор критерия и показателей качества. Установление перспектив развития объектов.
	\item Правильное сочетание различных методов проектирования, в первую очередь, математических, эвристических и экспериментальных. Итерационный метод проектирования.
\end{enumerate}

\section{Блочно-иерархический подход к проектированию}

Если в системном подходе прибор рассматривается как сложная система, состоящая из связанных и взаимодействующих частей, то при блочно иерархическом подходе прибор рассматривается как иерархическая структура, состоящая из большого количества уровней и ветвей, наподобие некоторого опрокинутого дерева (рис.~\ref{pic:2block}).

\begin{figure*}[h]
	\includegraphics[width=0.9\textwidth]{2block.png}
	\caption{Блочно-иерархический подход к проектированию}
	\label{pic:2block}
\end{figure*}

В соответствии с блочно-иерархическим подходом в объекте проектирования может быть выделен ряд иерархических уровней (рис.~\ref{pic:2block}). На верхнем уровне подлежащий проектированию сложный объект состоит из ряда менее сложных элементов (например, для ОЭП -- приемная система, электронный блок обработки сигналов с выхода приемника излучения). Указанные элементы на более низком иерархическом уровне, в свою очередь, являются системами элементов менее сложной структуры (например, в приемную оптическую систему могут входить объектив, компенсатор, анализатор изображения, сканирующий блок). 

Далее подобное разделение на элементы может продолжаться до некоторого уровня, на котором дальнейшее разделение становится уже невозможным. Элементы, полученные на этом уровне, по отношению к объекту проектирования являются базовыми. Применительно к ОЭП такими базовыми элементами будут детали (оптические, механические) и различные комплектующие изделия (электро- и радиоэлементы, подшипники, электродвигатели).

Иерархия составных частей ОЭП при блочно-иерархическом подходе:
\begin{enumerate}
	\item Приборное устройство (или его конструируемая часть).
	\item Функциональная единица.
	\item Сборочная единица --- изделие, составные части которого подлежат соединению между собой сборочными операциями.
	\item Детали --- изделие, изготовленное из однородного (по наименованию и марке) материала без применения сборочных операций. (Или неделимые однородные тела, состоящие из элементов формы (геометрических поверхностей тел) и материала).
\end{enumerate}

Общий процесс проектирования при таком подходе представляется в виде движения по рассматриваемому дереву, при котором выполняются элементарные проектные операции на каждом уровне и на каждой ветви, т.е. структура проектирования также является блочно-иерархической, причем на каждом уровне и ветви процесс проектирования имеет дело с небольшим количеством элементов, рассматриваемых как целые, благодаря чему этот процесс достаточно несложен и вполне реализуем при нормальных ресурсах. Весь процесс проектирования, сплетающийся в виде блочно-иерархической структуры таких элементарных процессов, также теперь становится вполне реализуемым.

Такая структура позволяет осуществлять общий процесс проектирования, используя различные направления движения по блочно-иерархическому дереву. В зависимости от направления движения различают нисходящее, восходящее и смешанное проектирование.

\newthought{Нисходящее проектирование}\marginnote{\allcaps{НИСХОДЯЩЕЕ\break ПРОЕКТИРОВАНИЕ}}, как следует из его названия, начинается с верхнего уровня, где прибор рассматривается как целое, затем проектируется его структура первого уровня, затем второго. Результатом проектирования на данном уровне является техническое задание для проектирования на следующем, более низком уровне.

Нисходящее проектирование всегда гарантирует выполнение требований технического задания на каждом уровне и поэтому должно бы считаться наиболее рациональным, но на каком-то уровне процесс проектирования может остановиться из-за того, что при существующих физических, технических, технологических или экономических ограничениях решение обратной задачи и соблюдение технического задания данного уровня становится невозможным. 

В этом случае приходится возвращаться на предыдущий уровень или даже выше, искать там другое решение своей обратной задачи, а затем опять пробовать вернуться на тот уровень, на котором процесс остановился, но с уже другим техническим заданием. 

Таким образом, блочно-иерархическая структура, позволяя в принципе реализовать процесс проектирования, делает неизбежным его итерационный характер, заключающийся в возврате к повторению проектирования на предыдущих уровнях с измененными условиями.

\newthought{Восходящее проектирование}\marginnote{\allcaps{ВОСХОДЯЩЕЕ\break ПРОЕКТИРОВАНИЕ}} выполняется в обратном порядке; при этом происходит как бы сборка отдельных узлов, а затем сборка всего прибора. 

Восходящее проектирование, как нетрудно увидеть, обычно гарантирует реализуемость проекта на любом уровне, но отнюдь не гарантирует соблюдения всех требований технического задания, поэтому процесс может остановится на каком-либо уровне из-за несоблюдения требований технического задания высшего уровня. При этом необходим возврат на предыдущие низшие уровни с попыткой <<собрать>> структуру данного уровня из других элементов. 
Таким образом и восходящее проектирование также неизбежно имеет итерационный характер.


\newthought{При смешанном проектировании }\marginnote{\allcaps{СМЕШАННОЕ\break ПРОЕКТИРОВАНИЕ}} по части ветвей мы имеем нисходящий процесс, а по части~-- восходящий, которые в определенных точках встречаются. Итерационный характер такого проектирования также очевиден.

Из рассмотренных процессов предпочтительным является все-таки нисходящее проектирование. На практике, особенно для сложных приборов, процесс проектирования носит обычно смешанный характер с преобладанием нисходящих потоков, а восходящее проектирование применяется к тем частям приборов, которые собираются из стандартных, хорошо отработанных деталей, элементов и узлов.

Из рассмотренного становится ясен также эвристический характер проектирования, т.е. невозможность его полной алгоритмизации, автоматизации, поскольку ввиду сложности процесса и невозможности заранее определить полностью его ход необходимо принимать решения на основании опыта, интуиции, с привлечением творческих способностей разработчика, т.е. на базе так называемого алгоритма принятия решения. 

Все этапы проектирования выполняются на основе ТЗ. В случае, когда проектирование объекта проводится по сформулированным на более высоком иерархическом уровне ТЗ, оно носит название внутреннего. Внешнее проектирование предполагает разработку ТЗ на систему высшего иерархического уровня. При внешнем проектировании необходим правильный учет современного состояния техники, возможностей технологии, перспектив их развития, экономических факторов.

\section{Аспекты проектирования}
Любое техническое решение обладает положительным и отрицательным эффектом. 

Применение САПР качественно улучшает процесс проектирования. 
Использование САПР позволяет:
\begin{itemize}
	\item проанализировать большое число различных схемных и конструктивных решений за короткий интервал времени;
	\item использовать более точные математические методы для расчёта и проектирования ОЭП;
	\item создавать конструкции, оптимально отвечающие предъявляемым к ним техническим требованиям;
	\item повышать качество конструкторской и технологической документации создаваемых ОЭП;
	\item проводить численные эксперименты, которые могут с достаточной степенью точности имитировать натурные испытания.
\end{itemize}

Идеальное техническое устройство --- устройство, которое выполняет свои функции, но его нет.

Существует большое количество методов проектирования, в том числе и большая группа эвристических\marginnote{Эвристика --- область знаний, основанная на творческом, неосознанном мышлении человека}:
\begin{itemize}
	\item теория решения изобретательских задач (ТРИЗ);
	\item мозговой штурм;
	\item метод итераций;
	\item метод контрольных вопросов;
	\item функционально-стоимостной анализ.
\end{itemize}

\newthought{ТРИЗ}\marginnote{\allcaps{ТРИЗ}} разработана в СССР Г.С.~Альтшуллером в 1956~г. 

ТРИЗ\marginnote{http://www.altshuller.ru/} является своего рода попыткой формализовать процесс проектирования и решения изобретательских задач. 

ТРИЗ развивает иной, более творческий тип мышления.
В ТРИЗ при проектировании используются различные приёмы, позволяющие решить технические противоречия.

\newthought{Мозговой штурм}\marginnote{\allcaps{МОЗГОВОЙ ШТУРМ}}:
\begin{itemize}
	\item Постановка задачи.
	\item Генерация идей.
	\begin{itemize}
		\item запрещена критика идей;
		\item приветствуются любые идеи;
		\item количество идей не ограничено.
	\end{itemize}
	\item Отбор идей.
\end{itemize}


\section{Основные требования, предъявляемые к ОЭП}

\newthought{Требования по внешним условиям и условиям эксплуатации}: к внешним условиям, оказывающим влияние на работу ОЭП, могут быть отнесены климатические факторы, механические воздействия, возникающие при транспортировании и эксплуатации, различные виды силовых полей, действие ионизирующего излучения.

В процессе эксплуатации различают два режима:
\begin{itemize}
	\item сохранение работоспособности ОЭП при воздействии дестабилизирующих факторов с экстремальными значениями (устойчивость);
	\item обеспечение работоспособности ОЭП в нормальных условиях после воздействия на неработающий прибор дестабилизирующих факторов с экстремальными значениями (прочность, стойкость).
\end{itemize}

Наиболее разнообразно влияние климатических факторов: температуры, влажности, давления окружающей среды, воздействия твердых и газообразных примесей, солнечного излучения, ветровой нагрузки, биофакторов.

\textsc{Температура}\marginnote{\allcaps{ТЕМПЕРАТУРА}} окружающей среды оказывает существенное влияние на работу приборов, так как при ее изменении практически все элементы и детали ОЭП меняют свои свойства. 
Диапазон температур, в котором приходится работать ОЭП, весьма широк. 
Даже в земных условиях возможны перепады температуры воздуха от $-80^{\circ}$C (в Антарктиде) до $+55^{\circ}$C (в тропических районах). 
При прямом воздействии Солнца температура нагретой поверхности может быть значительно выше. 
В отдельных случаях требуется обеспечить нормальную работу прибора в еще более жестких температурных условиях. 

Например, температура на поверхности Венеры достигает $300^{\circ}$С, а в условиях космического пространства при затенении от солнечного излучения близка к абсолютному нулю.

Большинство ОЭП эксплуатируется в нормальных температурных условиях. 
Для многих видов приборов, используемых на открытом воздухе, требуется обеспечить нормальную работу в интервале температур $-50\ldots+50^{\circ}$С. 
В отдельных случаях требуется обеспечение работы приборов в экстремальных условиях, указанных выше.

При недостаточном учете влияния перепадов температуры возможны ухудшение качества оптического изображения из-за термооптических аберраций и смещения плоскости изображения за счет температурных деформаций, появление расклеек в компонентах, разрушение оптических деталей вследствие разности показателей расширения оптических материалов и материалов оправ.

Тепловые воздействия на электронные элементы проявляются, в частности, в изменении параметров приемников излучения, номинальных значений параметров и характеристик электрорадиоэлементов, нарушении контактов и пробоях в изоляционных материалах.

В кинематических цепях при изменении температуры возможны ухудшение прочности материалов, повышение трения за счет изменения зазоров и вытекания или загустения смазочного материала. При неравномерном нагреве или охлаждении могут появляться деформации, приводящие к заклиниванию кинематических механизмов.

Весьма\marginnote{\allcaps{ВЛАГА}} серьезные последствия оказывает на приборы попадание \textsc{влаги}. 
Наличие влаги может привести к запотеванию оптических деталей, особенно в сочетании с резким изменением температуры. 
Пары воды, вступая в химическую реакцию с материалами, приводят к коррозии металлов, изменению физико-химических свойств специальных покрытий оптических деталей и изоляционных материалов. 
Под воздействием влаги ухудшаются контактные соединения за счет окисления контактов.

При проектировании предусматривают меры по защите приборов от воздействия влаги. Часто с этой целью приборы герметизируют, а внутренний объем осушают продувкой сухого очищенного воздуха. Могут применяться также специальные влагопоглотители.

\textsc{Давление}\marginnote{\allcaps{ДАВЛЕНИЕ}} окружающей среды оказывает заметное влияние на функционирование ОЭП. При понижении давления воздуха падает значение напряжения пробоя, что особенно важно помнить при использовании высоковольтных элементов. Кроме этого, существенно возрастает скорость испарения смазочного материала, что может привести к повышению трения и заклиниванию элементов кинематики прибора. В связи с уменьшением давления отвод теплоты за счет конвекционного переноса падает, в результате чего резко возрастает вероятность перегрева элементов прибора. Поэтому необходимо либо применять специальные материалы и элементы, рассчитанные на работу в условиях пониженного давления, либо осуществлять герметизацию прибора с созданием нормального рабочего давления внутри.

На\marginnote{\allcaps{ПЕСОК И ПЫЛЬ}} работу ОЭП оказывают влияние не только рассмотренные выше климатические факторы, но и содержащиеся в воздухе \textsc{песок и пыль}. Их механическое воздействие в сочетании с воздействием влаги и нагрева иногда приводит к значительному ухудшению характеристик приборов.

В сочетании с ветровым воздействием наличие в воздухе частиц песка и пыли приводит к абразивному разрушению полированных и окрашенных поверхностей. При этом вследствие матирующего эффекта возможен выход из строя оптических систем.

Для\marginnote{\allcaps{СОЛНЕЧНОЕ ИЗЛУЧЕНИЕ}} приборов, эксплуатируемых на открытом воздухе, необходимо учитывать \textsc{воздействие солнечного излучения}, приводящее к перегревам, нарушениям лакокрасочных покрытий, усилению коррозии при одновременном воздействии кислорода и влаги воздуха, быстрому старению резины, пластмасс и электрической изоляции.

При\marginnote{\allcaps{БИОФАКТОРЫ}} длительной эксплуатации и хранении приборов, а также при эксплуатации в тропических условиях следует учитывать \textsc{влияние биофакторов}, к которым относятся плесневые грибы, насекомые и грызуны. Развитие плесени ухудшает механические и электрические параметры приборов, а также пропускание оптических деталей. Борьба с влиянием этого фактора сводится к герметизации и осушке внутренних объемов приборов, применению стекол группы А, защите оптических деталей специальными покрытиями, использованию фунгицидов. Кроме того, могут быть использованы такие методы, как придание корпусам и наружным деталям простой формы без углублений, пазов, выступов, которые способствуют скоплению грязи и пыли и затрудняют чистку приборов.

Важное\marginnote{\allcaps{МЕХАНИЧЕСКИЕ\break ВОЗДЕЙСТВИЯ}} значение при конструировании ОЭП имеет учет \textsc{влияния механических воздействий}, к которым относятся вибрация, ударные воздействия, транспортировочные перегрузки. При этом следует иметь в виду, что наряду с внешними источниками воздействий на элементы прибора могут оказывать влияние вибрация и удары, обусловленные внутренними источниками, например несбалансированностью вращающихся частей, неточностью изготовления, зазорами, разрушениями соприкасающихся кинематических элементов.

В результате воздействий указанных факторов возможны разрушения отдельных элементов, деталей и паек, нарушение контактов реле, переключателей, потенциометров и коллекторов, повреждение изоляции с возникновением замыканий, самоотвинчивание резьбовых соединений, появление трещин, сколов в оптических и других хрупких деталях.

Механическая прочность конструкции обеспечивается применением соответствующих материалов, способов соединения деталей и может быть повышена за счет использования различных элементов жесткости: косынок, приливов, ребер.

Для предотвращения самоотвинчивания крепежных изделий либо применяют различные фиксаторы, либо устанавливают крепежные детали с использованием клеев, компаундов и герметиков. 

В случаях, когда указанные меры оказываются недостаточными, для защиты от механических воздействий используются демпферы и амортизаторы.

При\marginnote{\allcaps{ВОЗДЕЙСТВИЕ ПОЛЕЙ}} работе ОЭП подвергаются \textsc{воздействию различных полей}: электрического, магнитного, электромагнитного СВЧ, в результате чего могут возникать паразитные наводки, приводящие к ухудшению работы прибора. Источники полей могут находиться как вне, так и внутри прибора. Подавление наводок в большинстве случаев сводится к устранению или ослаблению паразитных связей между источником и приемником наводок путем экранирования и развязывания цепей.

Для защиты от электрических полей или подавления паразитной емкостной связи во всех диапазонах частот используют тонкие листы и пленки, а также проволочные сетки и решетки из материала с хорошей электрической проводимостью.

Для экранирования магнитных низкочастотных полей используют материалы с высокой магнитной проницаемостью (пермаллой, альсифер, технически чистое железо).

Для экранирования высокочастотных полей используют экраны из хорошо проводящих материалов (медь, латунь, алюминий). При действии полей СВЧ на основной материал экрана наносят слой серебра для повышения его электрической проводимости. 

Для защиты от наводок все электрические связи между блоками, по которым передаются измерительные сигналы, необходимо осуществлять экранированными проводами. Принципы расчета и конструирования защитных экранов изложены в соответствующей литературе.

Иногда\marginnote{\allcaps{ВОЗДЕЙСТВИЕ\break ИОНИЗИРУЮЩЕГО\break ИЗЛУЧЕНИЯ}} ОЭП используются в условиях \textsc{воздействия ионизирующего излучения} (на атомных электростанциях для дистанционного наблюдения, при космических исследованиях). 
Такие приборы должны отвечать требованиям радиационной стойкости. 
При воздействии ионизирующего излучения имеют место радиационные и поляризационные эффекты, приводящие к ухудшению оптических свойств материалов, нарушению работы полупроводниковых и электровакуумных приборов, изменению проводимости воздушных промежутков и диэлектрических материалов. 
При конструировании ОЭП, работающих в указанных условиях, прежде всего необходимо применять радиационно-стойкие материалы и элементы.

К ОЭП могут предъявляться также специфические требования, связанные с условиями эксплуатации. К их числу можно отнести, например, такие, которые вытекают из особенностей приборов, эксплуатируемых в состоянии невесомости, глубоко под водой, в шахтах. Кроме того, в некоторых ОЭП отдельные блоки могут работать в нормальных условиях, а остальные~-- в крайне неблагоприятных. Все эти особенности должны оговариваться при разработке ТЗ.

Таким образом, в современных условиях конструктору приходится иметь дело с широким кругом требований, которые находятся в тесном взаимодействии и часто противоречат друг другу, что приводит к многовариантности проектных решений.
\chapter[Регламентация конструкторских работ.\break Конструкторская документация]{Регламентация конструкторских работ.\\ Конструкторская документация}

\newthought{Конструирование} является составной частью проектирования и заключается в разработке конкретного варианта изделия на основе проведенных предварительных исследований. 

Организация процесса проектирования определяется степенью новизны и сложностью решаемой задачи. В зависимости от степени новизны различают:
\begin{itemize}
	\item частичную модернизацию существующего прибора (системы), приводящую к некоторому улучшению одного или нескольких показателей качества за счет изменения параметров, улучшения элементной базы, частичного изменения структуры;
	\item существенную модернизацию, приводящую к значительному улучшению основных показателей качества прибора (системы) за счет существенного изменения параметров и структурной схемы, приводящих к большим конструктивным изменениям;
	\item создание нового прибора (системы), предназначенного для решения известных или принципиально новых задач и основанного на новых принципах действия, использование которых позволяет резко улучшить основные показатели качества.
\end{itemize}

При создании новых ОЭП процессу собственно проектирования~-- опытно-конструкторским работам (ОКР)~-- обычно предшествуют научно-исследовательские работы (НИР).

Целью НИР является решение проблемных вопросов, позволяющее обосновать возможность и целесообразность дальнейшего проектирования, получить необходимую исходную информацию и тем самым предотвратить значительные затраты на проведение проектных работ в случае, когда поставленная задача не может быть решена предлагаемыми средствами.

В рамках НИР изучается состояние разработок по поставленной или родственным задачам. С этой целью анализируются все доступные источники информации, а также опыт промышленности. На основе выдвинутых теоретических положений разрабатываются макеты узлов и прибора в целом. После их изготовления и экспериментальных исследований дается заключение о возможности создания промышленного образца прибора и формулируются рекомендации по проведению ОКР.

Последовательность разработки и изготовления промышленных изделий в настоящее время регламентируется группой государственных стандартов, входящих в Единую систему конструкторской документации (ЕСКД).

ЕСКД устанавливает единый порядок разработки, выполнения, оформления, согласования, внесения изменений, учета и хранения конструкторской документации.

В соответствии с действующим стандартом (ГОСТ~2.103–68) проектирование ОЭП можно представить в виде последовательности этапов, в процессе которых разрабатывают ТЗ, техническое предложение, эскизный, технический и рабочий проекты (рис.~\ref{pic:etapi}).

\begin{figure}[h!]
	\begin{center}
		\includegraphics[width=0.45\linewidth]{etapi.png}
		\caption{Этапы проектирования}
		\label{pic:etapi}
	\end{center}
\end{figure}

Перечисленные этапы позволяют обеспечить изготовление опытного образца, который затем испытывается. 
По результатам испытаний в конструкцию вносятся необходимые изменения и уточнения, и после окончательных испытаний дается заключение о возможности изготовления установочной серии приборов. 
В зависимости от потребностей в данном приборе в дальнейшем осуществляется переход к мелкосерийному, серийному или массовому производству.

Следует отметить, что в зависимости от назначения и области применения прибора (системы), необходимых сроков разработки, а также в связи с внедрением современных методов системного и автоматизированного проектирования последовательность и содержание этапов могут изменяться. 
Например, если на этапе технического предложения получено полное представление о схемном и конструктивном решении прибора, этап эскизного проектирования может не выполняться, и разработчик сразу переходит к техническому проектированию. 
Для ускорения процесса проектирования иногда могут быть совмещены технический и рабочий проекты. 
Однако при создании большинства современных приборов указанная последовательность проектирования выдерживается.

\section{Техническое задание}
Проектирование любого промышленного изделия, в том числе и ОЭП, ведется на основании технического задания (ТЗ).

\noindent
\textsc{Техническое задание} --- это документ, который устанавливает назначение и область применения, технические, качественные и технико-экономические требования, а также определяет необходимые стадии разработки конструкторской документации и ее состав.

ТЗ составляется организацией-заказчиком при возможном участии организации-разработчика с привлечением других заинтересованных организаций. После утверждения и согласования ТЗ принимается к выполнению.

ТЗ на проектирование ОЭП обычно состоит из нескольких разделов.

Вводная часть ТЗ содержит основание для проведения ОКР. В следующем разделе указываются назначение и область применения изделия. Далее излагаются технические требования, предъявляемые к изделию.

К техническим требованиям относятся:
\begin{itemize}
	\item диапазон и точность измерений; 
	\item дальность действия;
	\item чувствительность или разрешающая способность; выходные параметры прибора, обеспечивающие его стыковку с другими системами (если данный ОЭП входит в состав какого-либо комплекса), либо форма представления информации об измеряемой величине (напряжение, код, вход в ЭВМ), если прибор решает самостоятельную задачу. При этом могут предъявляться требования к крутизне и диапазону линейности выходных характеристик прибора;
	\item спектральный диапазон работы, параметры и характеристики излучения исследуемого или измеряемого объекта;
	\item конструктивные требования к схемам и узлам прибора (кинематическим, электрическим, оптическим);
	\item габаритные размеры и масса прибора или отдельных его частей;
	\item требования по видам потребляемой энергии и мощности потребления;
	\item требования к тепловыделению прибора, а при необходимости охлаждения -- параметры системы охлаждения;
	\item требования по соответствию параметров прибора или его конструктивного исполнения определенным ГОСТам, ОСТам и другим нормативным документам.
\end{itemize}


В соответствующем разделе дается полная характеристика условий работы изделия, среди которых можно назвать:
\begin{itemize}
	\item характер помех, определяемый либо конкретным указанием их энергетических, спектральных и пространственных характеристик, либо указаниями общего порядка (например, облачное небо, звездный фон, вид ландшафта);
	\item параметры или характеристики, определяющие прохождение излучения для ОЭП, работающих в полевых или других неблагоприятных условиях (например, данные о среде распространения и ее параметрах, метеорологической дальности видимости, наличии задымленности);
	\item характер размещения прибора и в связи с этим различные динамические факторы условий эксплуатации (вибрация, перегрузки, ударные воздействия);
	\item динамические свойства исследуемого объекта -- частота колебаний, перемещения, распределение параметров по гармоническим составляющим и др.;
	\item метеорологические факторы -- температурный диапазон, влажность, давление, воздействие осадков, пыли, морского тумана, солнечной радиации;
	\item требования к защите прибора от воздействия полей и излучений;
	\item условия хранения и транспортирования;
	\item диапазон возможных отклонений параметров системы энергопитания.
\end{itemize}

В специальном разделе ТЗ излагаются требования по надежности и работоспособности, в том числе:
\begin{itemize}
	\item гарантийный срок службы прибора при обусловленных ТЗ условиях эксплуатации;
	\item периодичность проверок, аттестаций и профилактического ремонта;
	\item режимы работы, в частности время непрерывной работы, периодичность включения;
	\item время готовности прибора к работе от момента включения электропитания;
	\item требования к надежности работы в течение определенного времени с требуемой вероятностью безотказной работы;
	\item требования к жесткости, надежности крепления элементов и узлов, системе амортизации;
	\item требования к безопасности работы с прибором.
\end{itemize}

В ТЗ включаются также разделы, в которых излагаются требования по охране труда, технической эстетике, технологичности, отражаются технико-экономические показатели, специальные требования, учитывающие специфику построения, применения и изготовления прибора.

В одном из заключительных разделов ТЗ отражаются этапы создания прибора и состав технической документации, разрабатываемой на каждом этапе и ОКР в целом, другие аспекты проектирования, имеющие принципиальное значение. После оформления и утверждения ТЗ приступают непосредственно к проектным работам.

Следует отметить, что в процессе выполнения ОКР техническое задание может уточняться по взаимному согласованию заинтересованных сторон в случаях, если будет доказана необоснованность каких-либо требований, показана принципиальная невозможность обеспечения некоторых свойств.

\section{Техническое предложение}

\noindent
ГОСТ~2.118–2013: \textsc{техническое предложение} разрабатывают в целях выявления дополнительных или уточненных технических и эксплуатационных требований к прибору, которые не были отражены в ТЗ и для обоснования которых целесообразно выполнить предварительную конструкторскую проработку и анализ различных вариантов решения.

Наиболее типичными видами работ, проводимых на этапе технического предложения, являются:
\begin{itemize}
	\item научно-технический поиск в целях подбора и изучения всех доступных  материалов по проектируемому изделию;
	\item анализ полученной информации и выявление положений, позволяющих наметить варианты решения поставленной в ТЗ задачи;
	\item установление возможных вариантов схемы и конструкции прибора;
	\item сравнительная, оценка выявленных вариантов по различным показателям, определенным ТЗ;
	\item проверка вариантов на патентную чистоту и конкурентоспособность, оформление заявок на изобретение;
	\item проверка соответствия возможных вариантов требованиям стандартизации, унификации, техники безопасности, эргономики;
	\item предварительная оценка технологичности конструкции прибора.
\end{itemize}

В процессе выполнения указанных работ на данном этапе могут быть проведены различные расчеты, а также экспериментальные исследования с использованием математических моделей и макетов. Для изготовления макетов должна быть разработана конструкторская документация.

Результатом работ на данном этапе должно быть ТЗ, сформулированное с учетом положений, выявленных в процессе теоретических и экспериментальных исследований, а также конструкторская документация (КД), дающая обобщенное представление о выявленных технических решениях.

На рассматриваемом этапе наряду с учетом конструктивных и эксплуатационных особенностей существующих изделий аналогичного назначения необходимо иметь в виду тенденции и перспективы развития отечественной и зарубежной техники в данной области.

Следует уже при проведении данного этапа проектирования стремиться к выбору оптимального варианта прибора, так как это позволит избежать ненужных затрат на последующих этапах и ускорит проектирование. В случае невозможности такого выбора необходимо установить дополнительные требования к последующим этапам.

Конструкторская документация, выпускаемая на этапе технического предложения, включает обобщенные схемы ОЭП, упрощенные чертежи общего вида, габаритный чертеж, ведомость технического предложения, пояснительную записку, патентный формуляр.

Пояснительная записка в соответствии с действующим стандартом (ГОСТ~2.106-68) в общем случае должна включать следующие разделы:
\begin{itemize}
	\item введение (с указанием документов, на основании которых выполняется проектирование);
	\item назначение и область применения  прибора;
	\item технические характеристики прибора;
	\item описание и обоснование выбранной конструкции;
	\item расчеты, подтверждающие работоспособность и надежность выбранного конструктивного решения;
	\item описание организации работ с применением разрабатываемого прибора;
	\item ожидаемые технико-экономические показатели;
	\item уровень оценки по показателям стандартизации, унификации, патентной чистоты.
\end{itemize}

Указанная структура пояснительной записки применима к любому этапу проектирования. При этом в зависимости от особенностей изделия и характера решаемых на том или ином этапе проектирования задач возможно объединение, исключение или введение новых разделов.

Пояснительная записка должна быть оформлена в соответствии с действующим стандартом~(ГОСТ~2.105-79). Чертежи и схемы выполняются с максимальными упрощениями, предусмотренными ЕСКД.

После рассмотрения и утверждения технического предложения его материалы служат основой для проведения последующих этапов проектирования.

\section{Эскизное проектирование}

\noindent
ГОСТ 2.119–2013: \textsc{эскизный проект} является проектной стадией разработки КД и его следует разрабатывать в соответствии с ТЗ с целью установления принципиальных конструктивных решений, дающих общее представление об устройстве, принципах работы и габаритных размерах разрабатываемого изделия, а также данных, определяющих основные параметры, когда это целесообразно сделать до разработки ТП или рабочей КД.

При выполнении эскизного проекта проводят:
\begin{itemize}
	\item проработку возможных вариантов схемного и конструктивного решений прибора;
	\item расчетное обоснование их ожидаемых технических характеристик;
	\item оценку возможности реализации полученных вариантов на основе освоенной промышленностью номенклатуры материалов и комплектующих изделий;
	\item оценку технологичности конструкции и возможностей изготовления прибора в условиях конкретной производственной базы;
	\item проверку принятых технических решений на патентную чистоту и оформление заявок на изобретения в случае положительных результатов патентных исследований;
	\item проверку решений на соответствие требованиям техники безопасности, стандартизации и унификации;
	\item проработку художественно-конструкторских вопросов, оценку прибора по показателям эргономики.
\end{itemize}

С учетом полученных результатов перечисленных работ выполняется сравнительная оценка вариантов по установленным ТЗ показателям и обобщенным критериям оценки качества ОЭП и выбирается оптимальный вариант прибора.

На этапе эскизного проектирования на основе принятых принципиальных решений большое внимание уделяется выявлению на основе принятых принципиальных решений новых изделий и материалов, которые планируется разработать и изготовить другими предприятиями (например, источников и приемников излучения, электромеханических элементов, шарико-подшипников, конструкционных и других материалов). На этом этапе должны быть составлены технические требования к таким изделиям и материалам и определен круг их возможных разработчиков.

С целью более обоснованного выбора оптимального варианта прибора может быть проведено макетирование его отдельных узлов или прибора в целом с последующим исследованием макетов.

Важное значение при эскизном проектировании имеет оценка метрологического обеспечения будущего серийного или массового производства прибора. Это прежде всего относится к прецизионным приборам, поскольку для оценки их точностных возможностей может потребоваться уникальное оборудование (стенды). В некоторых случаях для создания такого оборудования необходимы значительные усилия, вплоть до проведения самостоятельных НИР и ОКР, на выполнение которых должно быть выдано соответствующее ТЗ.

Иногда уже на этапе эскизного проектирования предварительно решают вопросы упаковки и транспортирования, например при проектировании крупногабаритных изделий или изделий, для которых требуется специальная упаковка или средства транспортирования.

На этапе эскизного проектирования при необходимости выполняют и другие работы. В то же время обычно не повторяют работы, проведенные на этапе технического предложения, если они не могут дать дополнительных данных. В этом случае результаты ранее проведенных работ отражаются в пояснительной записке.

\textsc{Конструкторская документация эскизного проекта} включает:
\begin{itemize}
	\item основные схемы прибора;
	\item чертеж общего вида;
	\item чертежи основных сборочных единиц;
	\item габаритный чертеж;
	\item ведомость эскизного проекта;
	\item пояснительную записку.
\end{itemize} 

\textsc{Схемы прибора} разрабатывают на основе выбранного принципа его работы и проведенных расчетов. 
Как правило, выполняются функциональные или принципиальные схемы следующих видов: оптические, электрические, кинематические. 
Схемы должны давать полное представление о принципе работы прибора, взаимосвязях всех его узлов и элементов.

\textsc{Чертеж общего вида}\marginnote{\allcaps{ЧЕРТЁЖ ОБЩЕГО ВИДА}} выполняется с упрощениями, предусмотренными ЕСКД, и должен давать представление о компоновке прибора, взаимодействии его основных составных частей. 
При эскизном проектировании на чертеже общего вида часто используют контурное изображение заимствованных сборочных единиц, покупных и других комплектующих изделий (объективов, электродвигателей, подшипников).

\textsc{Габаритный чертеж}\marginnote{\allcaps{ГАБАРИТНЫЙ ЧЕРТЁЖ}} представляет собой контурное (упрощенное) изображение прибора с габаритными, установочными и присоединительными размерами и является необходимым конструкторским документом, если прибор входит в состав каких- либо сложных систем, комплексов или должен быть установлен (смонтирован) на специальных основаниях. 
Габаритный чертеж разрабатывается по согласованию со смежными организациями и окончательно уточняется на этапе технического проекта.

\textsc{Пояснительная записка}\marginnote{\allcaps{ПОЯСНИТЕЛЬНАЯ\break ЗАПИСКА}} эскизного проекта выполняется с учетом следующих требований к содержанию разделов.

При изложении технической характеристики наряду с указанием свойств прибора приводятся сведения об отклонениях или соответствии требованиям ТЗ и сравнительные данные отечественных и зарубежных аналогов.

Раздел <<Описание и обоснование выбранной конструкции>> наряду с изложением принятых схемных и конструктивных решений может содержать сведения о макетах прибора, методике и результатах их испытаний, сведения о технологичности, дополнительные результаты патентных исследований, сведения о вновь разрабатываемых материалах и комплектующих изделиях.

Пояснительная записка эскизного проекта должна содержать все необходимые расчеты ОЭП, подтверждающие возможность его реализации. При большом объеме расчетов они могут быть оформлены в виде отдельного документа. При этом в пояснительной записке приводятся только результаты расчетов.
К числу наиболее важных относятся следующие виды расчетов ОЭП: энергетический (светотехнический); оптической системы (габаритный, аберрационный); электронного тракта; точности. В зависимости от принципа работы ОЭП проводят и другие расчеты, часто имеющие принципиальное значение: кинематический, динамический, надежности, прочности и жесткости, температурных режимов.

Приложение к пояснительной записке может включать сведения по стандартизации и унификации, материалы художественно-конструкторской проработки, в частности результаты эргономического анализа, требования по технике безопасности и другие материалы, представляющие интерес для всесторонней характеристики проектируемого прибора.

Эскизный проект рассматривается заинтересованными организациями и защищается в установленном порядке. 
Выявленные в результате рассмотрения и защиты замечания либо устраняются, либо по ним намечаются мероприятия для последующих этапов проектирования, после чего протокол о защите утверждается.

\section{Техническое проектирование}

\noindent
ГОСТ 2.120–2013: \textsc{технический проект} является проектной стадией разработки КД и его следует разрабатывать в соответствии с ТЗ с целью выявления окончательных технических решений, дающих  полное представление об устройстве разрабатываемого изделия, и исходных данных для разработки рабочей КД, когда это целесообразно сделать до разработки рабочей КД. 

Основными видами работ являются:
\begin{itemize}
	\item детальная разработка конструкции всего прибора и его составных частей;
	\item разработка принципиальных схем, на основе которых могут быть выполнены монтажные схемы, схемы соединений, осуществлены сборка и настройка оптических и электронных блоков;
	\item окончательное оформление заявок и ТЗ на изготовление новых изделий и материалов;
	\item выявление номенклатуры покупных изделий и согласование их применения;
	\item окончательное согласование габаритных, установочных и присоединительных размеров, мест подключения разъемов с заказчиком и основными потребителями;
	\item анализ конструкции прибора, его узлов и отдельных наиболее сложных и ответственных деталей на технологичность и определение на основе этого анализа возможности использования имеющегося на предприятии оборудования, необходимости приобретения или создания нового технологического оборудования и спецоснастки;
	\item окончательное решение вопросов метрологического обеспечения по выбору средств измерений и методов контроля метрологических характеристик приборов;
	\item проверка принятых технических решений на соответствие требованиям стандартизации, унификации, техники безопасности;
	\item проверка приборов на патентную чистоту, оформление заявок на изобретения;
	\item окончательное решение вопросов транспортирования, хранения и монтажа на месте эксплуатации;
	\item оценка эксплуатационных характеристик приборов, в частности взаимозаменяемости, удобства обслуживания, ремонтопригодности, устойчивости к воздействию факторов внешней среды, возможности быстрого устранения отказов, контроля качества работы, обеспеченности средствами контроля технического состояния и др.
\end{itemize}

Как правило, разработка технического проекта сопровождается большим объемом макетирования. Макеты создаются в целях проверки конструктивных и схемных решений прибора, а также для подтверждения окончательно принятых решений. 

При этом наряду с функционирующими макетами целесообразно делать макеты-муляжи (например, из дерева), на которых можно проверить удобство обслуживания и расположения элементов, т. е. отработать эргономические и художественно-конструкторские показатели.

Выполняемые при техническом проектировании расчеты служат для окончательного установления свойств прибора, выработки требований к узлам и отдельным ответственным деталям. На этом этапе проектирования уточняются такие показатели, как инструментальная составляющая суммарной погрешности, которая может быть выявлена только на основе окончательно принятых конструктивных решений. 

Особое внимание уделяется подбору необходимого оборудования для лабораторных испытаний будущих приборов.

В результате технического проектирования обычно выпускается следующая \textsc{конструкторская документация}: 
\begin{itemize}
	\item чертежи общего вида;
	\item чертежи сборочных единиц;
	\item габаритный чертеж;
	\item чертежи всех схем;
	\item ведомость технического проекта;
	\item пояснительная записка;
	\item приложение к пояснительной записке;
	\item ведомость покупных изделий;
	\item ведомость согласования применения покупных изделий;
	\item патентный формуляр;
	\item карта технического уровня.
\end{itemize}

Пояснительная записка технического проекта включает те же разделы, что и записка к эскизному проекту. Однако в ней особое внимание уделяется обоснованию и описанию конструктивных особенностей прибора, принципов его функционирования, В нее включаются расчеты, выполненные по ходу технического проекта, Существенно расширяется раздел, посвященный описанию организации работ с прибором на месте эксплуатации. В этом разделе даются сведения о приемах и способах работы с прибором, транспортировании, монтаже и хранении, количестве и квалификации обслуживающего персонала.

В приложении к пояснительной записке могут быть приведены расчеты, материалы ху\-до\-жест\-вен\-но-конструкторской проработки.

Технический проект подлежит защите и утверждению заказчиком.

\section{Рабочее проектирование}
Рабочий проект выполняется с целью создания и отработки полного комплекта конструкторской документации ОЭП, достаточной для изготовления опытного образца прибора. Рабочее проектирование может выполняться как самостоятельный этап, но иногда для ускорения процесса проектирования его начинают на этапе технического проекта (технорабочий проект). Этап рабочего проектирования характеризуется тесным взаимодействием конструкторских и технологических подразделений предприятия.

Основными видами работ на этом этапе являются:
\begin{itemize}
	\item детальная разработка конструкции прибора и его узлов с указанием технологических требований к сборке и юстировке;
	\item доведение всех схем до рабочего состояния (выполняются монтажные схемы, на оптических схемах приводятся требования по юстировке);
	\item составление спецификаций и сводных ведомостей покупных и стандартных изделий и деталей, марок и сортаментов применяемых материалов;
	\item разработка ведомостей и чертежей согласования применения готовых изделий;
	\item согласование методик юстировки, настройки, монтажа, испытаний;
	\item составление технического описания, технических условий, инструкций по эксплуатации, формуляра, технического паспорта;
	\item составление ведомости запасного инструмента и принадлежностей (ЗИП);
	\item разработка технологических процессов изготовления наиболее сложных и ответственных деталей.
\end{itemize}

\newthought{Рабочие чертежи}\marginnote{\allcaps{РАБОЧИЕ ЧЕРТЕЖИ}} должны обеспечивать возможность оптимального применения стандартных, покупных и освоенных ранее изделий, рационально ограниченную номенклатуру материалов, покрытий, размеров, резьб, допусков, необходимую степень взаимозаменяемости, экономичные способы изготовления, максимальное удобство при эксплуатации.

В процессе рабочего проектирования выполняются контрольно-сборочные чертежи узлов и прибора в целом для выявления ошибок в рабочих чертежах деталей до их изготовления и сборки. Контрольно-сборочный чертеж вычерчивают по рабочим чертежам деталей путем считывания всех необходимых размеров, проверки правильности простановки допусков на сопрягаемые детали и тщательного переноса размеров в соответствии с необходимым масштабом на поле чертежа.

Все выявленные ошибки и неточности рабочих чертежей устраняются, после чего чертежи проходят нормоконтроль по действующему стандарту (ГОСТ~2.111-68), технологический контроль и утверждаются.

Рабочие чертежи деталей и сборочные чертежи являются основной документацией, руководствуясь которой можно осуществить изготовление \textsc{опытного образца} прибора. Дополнением к ним являются технические условия, содержащие все отсутствующие в чертежах, но необходимые для изготовления и отладки технические требования, а также требования на приемку и испытания.

Технические условия составляют в соответствии с ГОСТ~2.114-70 на основе ТЗ, чертежей и документации технического проекта.
Составление методики юстировки и настройки должно быть увязано с выпуском рабочих чертежей контрольно-юстировочной аппаратуры.

После подготовки и утверждения всей необходимой документации опытное производство предприятия изготавливает опытный образец или партию приборов. Конструкторские подразделения предприятия осуществляют наблюдение за ходом изготовления и оказывают необходимую помощь производству. Возникающие в процессе изготовления замечания к документации исправляются.

Изготовленные опытным производством образцы приборов передаются на всесторонние испытания. При проведении предварительных испытаний проверяют правильность функционирования, соответствие приборов техническим условиям и техническому паспорту. Эти испытания могут проводиться как в условиях заводской испытательной лаборатории (на соответствующих стендах), так и в условиях предполагаемой эксплуатации. Если изготовленные приборы прошли предварительные испытания, их передают на государственные испытания для полной проверки опытного образца прибора на соответствие ТЗ и техническим условиям. Так же, как и предварительные, государственные испытания могут проводиться в лабораторных и полевых условиях.

Государственные испытания осуществляются под руководством государственной комиссии, состоящей из специалистов отраслевых НИИ, представителей заказчика и предприятия-раз\-ра\-бот\-чи\-ка. Испытания регламентируются специальной программой. Государственным испытаниям подвергаются приборы, прошедшие предварительные испытания и снабженные всей необходимой технической документацией, что подтверждается соответствующим актом.

В процессе государственных испытаний фиксируются все замечания. Если они легко устранимы, то испытания продолжаются; если носят принципиальный характер, то опытные образцы и документация возвращаются на доработку, после которой вновь представляются на испытания.

По окончании испытаний составляется акт, где дается заключение о соответствии прибора ТЗ и о возможности его запуска в серийное или массовое производство, а также приводятся замеченные недостатки, которые должны быть устранены в процессе подготовки прибора к следующему этапу производства.

Перед серийным производством обеспечивается технологическая подготовка производства, заключающаяся в проектировании технологического процесса изготовления деталей и сборки, конструировании и изготовлении технологической оснастки, разработке методики контроля технических характеристик прибора и проектировании соответствующей контрольно-юс\-ти\-ро\-воч\-ной аппаратуры.

По окончании этапа технологической подготовки производства может быть изготовлена установочная партия приборов, на которой окончательно отрабатываются конструкторская документация и технологический процесс, а также проверяются наличие требуемых технологической оснастки и контрольно-юстировочной аппаратуры и их возможности. При соответствии установочной партии приборов и технической документации предъявляемым требованиям приборы запускаются в серийное производство.

\section{Конструкторская документация}
На всех этапах жизненного цикла (разработка -- производство -- эксплуатация) ОЭП сопровождает техническая документация (ТД). Состав этой документации и ее содержание регламентируется Государственными стандартами. В настоящее время в стране действует большое количество стандартов, которые сгруппированы по направлениям жизненного цикла изделий в следующие комплексы:
\begin{itemize}
	\item единая система конструкторской документации (ЕСКД);
	\item единая система технологической документации (ЕСТД);
	\item единая система программной документации (ЕСПД);
	\item единая система технологической подготовки производства (ЕСТПП);
	\item единая система защиты изделий и материалов от коррозии, старения и биоповреждений (ЕСЗКС).
\end{itemize}

Государственные стандарты, входящие в ЕСКД, устанавливают взаимосвязанные единые правила и положения по порядку разработки, оформления и обращения конструкторской документации на изделия, разрабатываемые и выпускаемые предприятиями всех отраслей промышленности.

\textsc{Конструкторские документы (КД)}\marginnote{\allcaps{КОНСТРУКТОРСКИЕ\break ДОКУМЕНТЫ}} --- графические и текстовые документы, в отдельности или в совокупности определяющие состав и устройство изделия и содержащие необходимые данные для его разработки и изготовления, контроля, приемки, эксплуатации, ремонта, утилизации.

Стандартам ЕСКД присваивают обозначения по классификационному принципу. Номер стандарта составляется из цифры, присвоенной классу стандартов ЕСКД, одной цифры после точки, обозначающей классификационную группу стандартов в соответствии с таблицей 1, числа, определяющего порядковый номер стандарта в данной группе, и двузначной цифры (после тире), указывающей год регистрации стандарта. Например, обозначение стандарта ЕСКД <<ЕСКД. Схемы. Виды и типы. Общие требования к выполнению>> имеет вид:
\begin{description}
	\item[ГОСТ 2.701-84]:
	\item[ГОСТ] -- категория нормативно-технического документа (государственный стандарт);
	\item[2] -- класс (стандарты ЕСКД);
	\item[7] -- классификационная группа стандартов; 
	\item[01] -- порядковый номер стандарта в группе; 
	\item[84] -- год регистрации стандарта.
\end{description}

Разработка и изготовление любого ОЭП связаны с выпуском конструкторской документации, которая полностью и однозначно определяют все необходимые и достаточные данные для изготовления, настройки и юстировки, приемки, эксплуатации и ремонта как всего прибора в целом, так и его составных частей.

\begin{table*}[ht]
	\fontfamily{ppl}\selectfont
	\begin{tabular}{cl} \hline 
		\toprule
		Шифр группы & Содержание стандартов в группе \\
		\midrule
		0 & Общие положения \\
		1 & Основные положения \\
		2 & Классификация и обозначение изделий в КД \\
		3 & Общие правила выполнения чертежей \\
		4 & Правила выполнения чертежей изделий машиностроения и приборостроения \\
		5 & Правила обращения КД (учет, хранение, дублирование, внесение изменений) \\
		6 & Правила выполнения эксплуатационной и ремонтной документации \\
		7 & Правила выполнения схем \\
		8 & Правила выполнения документов строительных, судостроительных и горных дел \\
		9 & Прочие стандарты \\
		\bottomrule \\
	\end{tabular}
	\label{tab:standart}
	\caption{Перечень классификационных групп стандартов ЕСКД}
\end{table*}

Согласно действующему стандарту (ГОСТ~2.102-68) к конструкторской документации относятся следующие графические и текстовые документы:
\begin{itemize}
	\item чертеж детали, содержащий изображение детали и другие данные, необходимые для ее изготовления и контроля;
	\item сборочный чертеж (СБ), содержащий изображение сборочной единицы и другие данные, необходимые для ее сборки и контроля;
	\item чертеж общего вида (ВО), определяющий конструкцию прибора, взаимодействие его основных составных частей, поясняющий принцип работы изделия, включая форму деталей и характерные размеры, которые облегчают уяснение формы элементов деталей, содержащий предельные отклонения сопрягаемых поверхностей и сопровождаемый техническими требованиями к прибору;
	\item теоретический чертеж (ТЧ), определяющий геометрическую форму прибора и координаты расположения составных частей;
	\item габаритный чертеж (ГЧ), представляющий собой контурное (упрощенное) изображение прибора с габаритными, установочными и присоединительными размерами;
	\item монтажный чертеж (МЧ) -- упрощенное изображение прибора с данными, необходимыми для его установки на месте эксплуатации;
	\item схемы по действующему стандарту (ГОСТ 2.701-84), на которых показаны в виде условных изображений или обозначений составные части изделия и связи между ними;
	\item спецификация -- документ, определяющий состав сборочной единицы, комплекса или комплекта. Спецификация в общем случае состоит из разделов: документация, комплексы, сборочные единицы, детали, стандартные изделия, прочие изделия, материалы, комплекты. Требования к выполнению спецификаций регламентирует действующий стандарт (ГОСТ 2.108-68);
	\item ведомости спецификаций (ВС), ссылочных документов(ВД), покупных изделий (ВП), согласования применения изделий (ВИ), держателей подлинников (ДП), технического предложения (ПТ), эскизного проекта (ЭП), технического проекта (ТП);
	\item пояснительная записка (ПЗ) содержит описание прибора и принципа его действия, а также обоснование принятых при разработке прибора технических и технико-экономических решений;
	\item технические условия (ТУ) содержат требования к прибору и его составным частям и деталям и обычно включают следующие разделы: вводную часть с указанием назначения, области применения прибора и условий его эксплуатации; состав комплекта прибора; технические требования к материалам, отдельным деталям, сборочным единицам и прибору в целом; требования к покрытиям и окраске; методы контроля технических характеристик, порядок приемки, поверок и испытаний; требования к транспортированию и хранению, смазыванию, упаковке; порядок маркировки; указания о гарантийных обязательствах изготовителя;
	\item программа и методика испытаний (ПМ);
	\item патентный формуляр (ПФ);
	\item карта технического уровня и качества изделия (КУ), характеризующая уровень качества прибора, соответствие его технических и экономических показателей достижениям науки и техники и потребностям народного хозяйства;
	\item таблицы (ТБ);
	\item расчеты (РР);
	\item инструкции (И), представляющие собой документы, используемые при изготовлении прибора (сборке, регулировании, контроле, приемке).
\end{itemize}

Помимо конструкторских документов в соответствии с ГОСТ~2.601-68 разрабатывается комплект эксплуатационных документов, в том числе:
\begin{itemize}
	\item техническое описание (ТО), дающее общее представление о приборе, его технических характеристиках, принципе его работы и устройстве, комплектации и другие сведения;
	\item инструкция по эксплуатации (ИЭ), которая может быть частью технического описания, но может быть и самостоятельным документом. В инструкции приводятся методика работы с прибором и его поверки, правила монтажа, подготовки прибора к работе, обращения с прибором, разборки, чистки, смазывания, транспортирования, а также указания по технике безопасности;
	\item технический паспорт (ПС) и формуляр (ФО) -- документы, сопровождающие прибор в процессе эксплуатации. Технический паспорт включает основные номинальные технические характеристики прибора, результаты исследования технических характеристик, состав комплекта, свидетельство о приемке, положения о гарантиях и сведения о рекламациях, номер прибора и номера комплектующих изделий. В формуляре наряду с основными сведениями, приведенными в паспорте прибора, даются сведения о режиме работы, учете времени эксплуатации, отметки о текущем состоянии прибора, его техобслуживании и ремонте;
	\item ведомость ЗИП (ЗИ) устанавливает номенклатуру, назначение и количество запасных частей, инструментов, принадлежностей и материалов, необходимых при эксплуатации и ремонте прибора;
	\item ведомость эксплуатационных документов (ЭД).
\end{itemize}

Состав ремонтных документов определяется ГОСТ~2.602-68. Эти документы предусматривают технически возможное и экономически целесообразное восстановление технических параметров прибора при эксплуатации на различных стадиях.

Важное место в конструкторской документации ОЭП принадлежит схемам. В соответствии с ГОСТ~2.701-84 виды схем обозначаются буквами, а их типы -- цифрами. В оптико-электронном приборостроении используются в основном схемы следующих видов:
\begin{description}
	\item[Э] -- электрические;
	\item[К] -- кинематические;
	\item[Л] -- оптические;
	\item[С] -- комбинированные.
\end{description}

Схемы в зависимости от их типа имеют следующие обозначения:
\begin{description}
	\item[1] -- структурные;
	\item[2] -- функциональные;
	\item[3] -- принципиальные;
	\item[4] -- соединений;
	\item[5] -- подключения;
	\item[6] -- общие;
	\item[7] -- расположения.
\end{description}

Например, схема электрическая функциональная имеет шифр Э2.

Специфическими конструкторскими документами ОЭП являются комбинированная функциональная и оптическая принципиальная схемы.

Функциональная комбинированная схема иллюстрирует процессы преобразования сигналов, происходящие в функциональных цепях прибора и в приборе в целом. Эта схема является основным документом, раскрывающим принцип работы прибора. При выполнении функциональных схем ОЭП руководствуются следующими положениями.

Функциональная схема выполняется без соблюдения масштаба, действительное пространственное расположение составных частей прибора либо не учитывается вообще, либо учитывается приближенно.

При выполнении функциональной комбинированной схемы могут быть использованы условные обозначения, применяемые при выполнении схем других видов (оптических, кинематических, электрических). Схема должна быть выполнена компактно, но без ущерба для ясности и удобства чтения.

Элементы и узлы схемы, являющиеся отдельными функциональными частями, допускается изображать в виде прямоугольников с указанием вида элемента и его характеристик.

При выполнении схемы необходимо пользоваться условными графическими изображениями, установленными ГОСТами. При отсутствии соответствующего стандартизованного условного обозначения элемент на схеме изображают либо в виде, приближенно соответствующем его конструктивному исполнению, либо в виде прямоугольника, внутри которого написано название элемента.

Условные графические обозначения, стандартизованные или построенные на основе стандартизированных обозначений, на схемах не поясняются. Элементы, составляющие функциональные группы или устройства, на схемах допускается выделять штрих-пунктирными линиями, указывая внутри контура наименование или тип группы. Для наглядности допускается изображать элементы схем различных видов, а также отдельные элементы и устройства, не входящие в данный прибор, но необходимые для пояснения принципа его работы.

Технические характеристики элементов или частей схемы следует указывать рядом с графическим обозначением или на свободном поле схемы. На схеме могут быть поясняющие надписи, диаграммы, таблицы, определяющие последовательность процессов во времени.

Механические связи между элементами схемы указываются штриховой линией, электрические и оптические -- сплошной. 

Оптические схемы выполняются в соответствии с ГОСТ~2.412-81 в определенном масштабе.

При разработке ОЭП выполняются и другие схемы, перечисленные выше, если они необходимы.

Структурная схема определяет основные функциональные части изделия, их назначение и взаимосвязи. 

Принципиальная схема определяет полный состав элементов и связей между ними и, как правило, дает детальное представление о принципах работы изделия. 

Схема соединений показывает соединения составных частей изделия и определяет провода, жгуты и кабели, которыми осуществляются эти соединения, а также места их присоединения и ввода (зажимы, соединители, фланцы).
Схема подключения показывает внешнее подключение изделия. 

Общая схема определяет составные части комплекса и соединения их между собой на месте эксплуатации. 

Схема расположения задает относительное положение составных частей изделия, а при необходимости проводов, жгутов, кабелей, светопроводов.

Все перечисленные схемы могут быть использованы при разработке других конструкторских документов, а также при эксплуатации приборов. Правила выполнения схем регламентируются соответствующими стандартами ЕСКД, относящимися к 7-й группе.

При разработке рабочих чертежей деталей, сборочных, общих видов, габаритных и монтажных чертежей, при оформление текстовых документов необходимо руководствоваться действующими стандартами (ГОСТ~2.109-73, ГОСТ~2.108-68, ГОСТ~2.106-68).

Каждый конструкторский документ должен иметь определенное обозначение в соответствии с обезличенной классификационной системой обозначений изделий и документов. Обозначение изделия и его основного конструкторского документа (спецификации и чертежа детали) имеет следующую структуру: индекс организации-разработчика; классификационная характеристика; порядковый регистрационный номер. К обозначениям всех остальных документов добавляются шифры (например, СБ, Э2, ЛЗ, ТУ).

В соответствии с действующим стандартом (ГОСТ~2.501-88) все подлинники, дубликаты и копии конструкторской документации подлежат учету и хранению в отделе технической документации (ОТД). Подлинник для сдачи в ОТД должен иметь необходимые подписи, подтверждающие его соответствие нормам, и предусмотренные согласования со всеми заинтересованными службами. Вносить изменения в конструкторскую документацию или аннулировать ее имеет право только предприятие - держатель подлинников. Основой для этого служит <<Извещение об изменении>>. Изменяемые размеры, слова, знаки, надписи, как правило, зачеркивают так, чтобы можно было легко прочитать зачеркнутое, и рядом с зачеркнутым проставляют новые данные.

\section{Требования к оформлению чертежей оптических деталей}

При изображении оптической детали используют общие правила машиностроительного и приборостроительного черчения, однако вследствие специфики назначения и изготовления оптической детали необходимо указать некоторые дополнительные сведения, а также особые нормативные и технологические требования.

Правила выполнения чертежей и схем оптических изделий установлены ГОСТ~2.412-81, требования и рекомендации по оформлению рабочих чертежей типовых оптических деталей изложены в справочниках оптика-конструктора и оптика-технолога.

Рассмотрим наиболее важные из них.
\begin{enumerate}
	\item Оптические детали (также схемы и узлы) следует изображать на чертеже по ходу луча, идущего слева направо.
	\item Радиусы кривизны сферических поверхностей деталей обозначают буквой $ R $, их выбирают по действующему стандарту (ГОСТ 1807-75) (что обусловлено контролем пробными стеклами и унификацией параметров инструмента). Асферические поверхности линз и зеркал определяют координатами точек поверхности или уравнением кривой, использованной для ее построения. Цилиндрические поверхности задают значением ее радиуса $R$, перед которым пишут <<Цилиндр>>.
	\item В правой верхней части чертежа оптической детали помещают таблицу, состоящую из трех частей: в первой части отражены требования к материалу, из которого изготовлена оптическая деталь, во второй - требования к изготовлению самой оптической детали и в третьей -- ее расчетные данные (заметим, что для оптических сборочных единиц таблица состоит только из требований к изготовлению и оптических характеристик)
	
	В первой части таблицы для деталей из бесцветного оптического стекла помещают следующие требования к материалу: категорию и класс по показателю преломления и средней дисперсии; категорию по оптической однородности; категорию по двойному лучепреломлению; категорию по показателю ослабления; категорию и класс бессвильности; группу, категорию и класс пузырности; категорию по радиационно-оптической устойчивости (стекла серии 100).
	
	Для деталей из цветного оптического стекла в таблице следует указывать категории по спектральной характеристике (показатель поглощения или ослабления), двойному лучепреломлению, категории и классы бессвильности и пузырности.
	
	Для деталей из других оптических материалов (кварцевое стекло, естественные и искусственные кристаллы, оптическая керамика) первую часть таблицы заполняют в соответствии с действующим стандартом (ГОСТ 23136-93) и действующими техническими условиями на эти материалы.
	
	Нормируемые показатели качества материала: двулучепреломление, бессвильность и пузырность (для стекол), поликристалличность, полиморфизм, (способность некоторых кристаллических веществ при одном и том же химическом составе существовать в состояниях с различной атомной кристаллической структурой) посторонние включения и другие локальные неоднородности (для кристаллов).
	
	Заметим, что некоторые из нормируемых показателей качества оказывают влияние не только на оптические характеристики системы, но и на точность конструктивных параметров.
	
	Например, свили -- области, отличающиеся от основной массы стекла химическим составом, а следовательно, оптическими и механическими свойствами, -- вызывают как деформацию волнового фронта отраженного или прошедшего излучения, так и местные погрешности формы  N поверхности в тех участках, где они выходят наружу.
	
	Остаточные напряжения, характеризуемые двойным лучепреломлением, не только искажают волновой фронт, но и влияют на общее $N$ и местное  $\Delta N$ отклонение поверхности.
	
	Вскрывшиеся при обработке рабочей поверхности пузыри не только оказывают некоторое прямое влияние на волновой фронт, но являются дефектами ее чистоты, а также приводят к местным погрешностям формы поверхности, образующимся при их располировывании.
	
	Вторая часть таблицы содержит требования к изготовлению детали, в которой, в зависимости от типа оптической детали, указывают:
	\begin{itemize}
		\item общую $N$ и местную $\Delta N$ погрешности формы рабочей поверхности; 
		\item класс чистоты полированной поверхности $P$;
		\item допустимую клиновидность пластин $\theta$;
		\item пирамидальность призм $\pi$;
		\item допустимую разность равных по номиналу углов призм;
		\item разрешающую способность (при необходимости);
		\item остаточную фокусность пластин и призм $f_{min}$ (при необходимости);
		\item класс точности пробного стекла $R$ или предельное отклонение от расчетного значения радиуса в процентах (для плоских поверхностей при необходимости).
	\end{itemize}
	
	Величина $N$ -- допуск на общее отклонение формы, рабочей поверхности оптической детали от эталона (формы поверхности пробного стекла), выраженный числом интерференционных колец или полос, наблюдаемых при наложении пробного стекла на поверяемую поверхность.
	
	В производственном обиходе интерференционную картину обычно называют цветом. Этот параметр определяет точность, с которой будет выполнен радиус кривизны сферической поверхности или отступление от плоскостности у плоской. Предельное отклонение стрелки кривизны  $h = (\lambda/2)N$. На практике данную погрешность называют общей ошибкой.
	
	Величина $\Delta N$ -- допуск на местное (нерегулярное) отклонение формы рабочей поверхности от эталонной (или иначе -- местные ошибки), выраженное числом интерференционных колец или полос.
	
	Заметим, что в ряде случаев (большие поверхности, асферические поверхности) контроль формы поверхности детали осуществляется не пробными стеклами, а с помощью сферометров, интерферометров и других методов и средств, что обусловливает также и иную систему задания допусков на погрешности формы рабочей поверхности (в процентах, линейной мере, угловой мере, долях длины волны света, дифракционным кружком рассеяния, значением асферичности).
	
	Допуск на местные ошибки устанавливают более жесткий (строгий) по сравнению с допуском на общую ошибку (примерно в 5-10 раз), так как местные погрешности формы более сильно влияют на качество изображения и не могут быть компенсированы (например, изменением воздушных промежутков между компонентами оптической системы).
	
	Обычно поля допусков $N$ и $\Delta N$ устанавливают симметричными относительно номинала и знак отступлений не указывают. В особых случаях их указывают со знаками плюс или минус. При знаке плюс наблюдается воздушный зазор на краю (касание в центре -- <<общий бугор>>), а при знаке минус -- зазор в центре (касание на краю -- <<общая яма>>). Для плоской поверхности это означает, что при знаке плюс она слегка выпуклая, а при знаке минус - слегка вогнутая.
	
	При назначении неодинаковых допусков для разных поверхностей одной детали или разных зон одной и той же поверхности обозначения этих допусков следует указывать с буквенными индексами, каждое в отдельной строке. Эти же индексы следует с $N$ и $\Delta N$ рассчитываются исходя из требуемого качества изображения.
	
	Конструктор может назначить также допуски по аналогии, ориентируясь на их рекомендованные значения (на основании статистических данных, взятых из практики). Естественно что нужно учитывать не только тип детали, но и материал, из которого она изготовлена, возможные технологические методы изготовления, спектральный диапазон работы, ее расположение в оптической системе (установлена она в широком или узком пучке лучей), вид оптической системы, ее конструктивные параметры и характеристики, габаритные размеры детали.
	
	Например, защитное стекло (светофильтр) может стоять как перед объективом (тогда допуски на $N$ и $\Delta N$ должны быть более жесткими), так и за окуляром (где указанные допуски будут шире). Детали, изготовленные из оптических полимеров обычно имеют относительно невысокую точность формы рабочих поверхностей ($N=8\div10$,  $N=1\div2$). Более точная форма поверхности достигается на материалах с высокой твердостью, по сравнению с материалами, имеющими низкую твердость. Допуск на погрешности форм рабочих поверхностей линзы (выполненной из стекла) объектива, работающего в видимом спектральном диапазоне, должен быть более жесткий, чем эти допуски на подобную линзу (выполненной, например, из германия) объектива, работающего в дальней ИК-области спектра. Достигаемая технологическая точность форм рабочих поверхностей зеркал, защитных стекол, линз при их изготовлении зависит от соотношения толщин по оси и наибольших размеров (диаметров) этих деталей.
	
	Допуск на дефекты, чистоты, полированных рабочих поверхностей оптических деталей выражают в классах чистоты $Р$ по ГОСТ~11141-84, которым оговорены размеры и число дефектов -- царапин, точек, их скоплений (к ним относят также вскрытые пузыри, следы недополировок, клея, выколки).
	
	Требования оговорены двенадцатью классами от I до IХа для поверхностей, удаленных от плоскости изображения, и еще более строгим классом Р0 (нулевой) с подразделениями 0-10, 0-20 и 0-40 для поверхностей, расположенных в плоскостях изображения предметов.
	
	Очень трудно не допустить появления царапин и точек на полированных оптических поверхностях. Главными причинами их образования являются загрязнение среды, окружающей рабочее место оптика, загрязнение порошкообразных шлифующих и полирующих материалов, пузырность оптических материалов.
	
	Допустимые клиновидностъ пластин $\theta$, пирамидальностъ $\pi$ и разность одинаковых углов призм $\delta$ рассчитываются исходя из допустимых значений вызываемых ими дефектов: отклонения пучка лучей от расчетного направления и аберраций оптической системы (поперечного хроматизма, комы, дисторсии). 
	
	При отсутствии требований к какому-либо из рассмотренных параметров в соответствующей графе ставят прочерк. В особых случаях в соответствующей графе дается знак сноски, а нормирование параметра приводится текстом в технических требованиях.
	
	Клиновидность -- отклонение от параллельности наружных поверхностей. Клиновидность определяют как разность значений толщины пластины в двух точках, но расположенных не в центре пластины, а по ее краям на противоположных концах пластины, отнесенную к диаметру пластины.
	
	Пирамидальность призмы измеряют автоколлимационным способом. Контролируемую призму помещают на столик, приведенный в горизонтальное положение, и получают автоколлимационное изображение от каждой грани. Разность смещений изображения по вертикали деленная на 2 составит угол пирамидальности.
	
	Аберрации оптических систем -- ошибки, или погрешности изображения в оптической системе, вызываемые отклонением луча от того направления, по которому он должен был бы идти в идеальной оптической системе.
	
	В третьей части таблицы указываются оптические характеристики детали. Так, для линз приводят фокусное расстояние и фокальные отрезки, а также световые диаметры на ее рабочих поверхностях, для призм -- геометрическую длину хода луча и световой диаметр.
	
	Световой диаметр -- диаметр поверхности, пропускающей световой поток.
	
	Расстояние от передней (первой по ходу луча) оптической поверхности до переднего фокуса именуется передним, а расстояние от последней оптической поверхности до заднего фокуса именуется задним вершинным фокусным расстоянием. Согласно действующим стандартам, вершинные фокусные расстояния именуются -- передний фокальный отрезок и задний фокальный отрезок.
	\item Допуски на шероховатость поверхностей различны для рабочих и нерабочих (базовых, технологических, свободных) поверхностей оптических деталей.
	
	\textsc{Шероховатость поверхности} -- совокупность неровностей, образующих микрорельеф поверхности детали. Класс шероховатости поверхности определяется высотой неровностей и средним арифметическим отклонением профиля.
	
	Высота неровностей ($R_z$) определяется как разница (максимальных) высоты пиков и впадин в десяти точках: 
	\begin{equation*}
	R_z = \dfrac{[(H_\text{п1}-H_\text{в1})+(H_\text{п2}-H_\text{в2})+(H_\text{п3}-H_\text{в3})+(H_\text{п4}-H_\text{в4})+(H_\text{п5}-H_\text{в5})]}{5},
	\end{equation*}
	$H_\text{п}$ -- максимальная высота пика, мкм, \\
	$H_\text{в}$ -- минимальная высота впадины, мкм.
	
	Среднее арифметическое отклонение профиля ($R_a$) -- средняя высота неровностей:
	\begin{equation*}
	R_a = \dfrac{y_1+y_2+y_3++y_n}{n},
	\end{equation*}
	$y$ -- координата точки профиля, при координате средней линией равной 0 мкм,\\
	$n$ -- количество точек, стремится к бесконечности (чем больше, тем лучше).
	
	Принято 14 классов шероховатости: 1 -- самый грубый и 14 -- самый гладкий. Например, поверхность 14 класса должна иметь $R_a=0,006-0,01$~мкм, $R_z=0,032-0,05$~мкм.
	
	В соответствии с изменениями №3 к ГОСТ~2.309-73(Обозначения шероховатости поверхностей) приняты следующие обозначения шероховатости на чертежах:
	\begin{figure}[h!]
		\includegraphics[width=0.6\textwidth]{3sher.png}
		\label{pic:3sher}
		\caption{Обозначение шереховатости}
	\end{figure}
	
	Рабочие (оптические) преломляющие и отражающие поверхности большинства деталей (за исключением, например, матовых стекол, экранов) полируются до высоты неровностей профиля по параметру $R_z$, равному 0,05~мкм.
	
	Нерабочие поверхности могут иметь различные значения параметров шероховатости, зависящие от их назначения, свойств материалов деталей, методов их получения и обработки (литье, прессование, штамповка, резание, шлифование, полировка, травление), характеристик и зернистости обрабатывающего инструмента (абразива). Наиболее часто шероховатость таких поверхностей, достигаемая удалением слоя материала, нормируется параметром $R_a$, равным 2,5 мкм.
	
	В случаях, когда материал детали (например, бериллий, карбид кремния, титановые и алюминиевые сплавы, из которых часто изготовляют зеркала космических телескопов) не позволяет получить оптической поверхности, на нее наносят конструкционное покрытие (стеклянное, медное, никелевое), которое затем обрабатывают (полировкой, алмазным точением) для получения требуемых шероховатости и точности формы поверхности.
	
	Заметим, что оптические поверхности деталей, работающие с мощным лазерным излучением, обрабатываются с применением методов глубокого шлифования и полировки для повышения их лучевой прочности.
	
	\item Допуски на толщину (размер) оптических деталей по (вдоль) оси пучка лучей (линз, пластин, клиньев) устанавливаются обычно симметричными ($\pm$), дающими большую свободу действий оптику, по сравнению с односторонним полем допуска, так как кроме толщины детали он должен выдержать также допуск (более строгий) на точность формы рабочих поверхностей ($N$,  $\Delta N$).
	\item На силовую деталь (линзу, зеркало) устанавливают допустимое значение ее децентрировки. Под \textsc{децентрировкой} понимают смещение центра(ов) кривизны ее рабочей поверхности с базовой оси детали или неперпендикулярность ее плоской рабочей поверхности к этой оси. Силовые детали (линзы, сферические и асферические зеркала, граданы) осуществляют силовое преобразование оптического излучения.  В ряде случаев (например, для цилиндрических рабочих поверхностей, деталей с некруглыми боковыми поверхностями) под децентрировкой понимают смещение или непараллельность центра кривизны либо оси цилиндра рабочей поверхности относительно базовых поверхностей.
	
	Согласно действующему стандарту (ГОСТ~2.412-81), децентрировка задается следующим образом: позиционным допуском, допуском формы заданной поверхности, перпендикулярностью (биением) плоской поверхности.
	
	Расчет допустимых значений децентрировки осуществляется исходя из допустимых значений вызываемых ею дефектов (смещения изображения, аберраций: комы, дисторсии, поперечного хроматизма) и соответствующих коэффициентов влияний децентрировок поверхностей на эти дефекты.
	\item На кромках оптических деталей, как правило, наносят фаски. Фаски подразделяют на:
	\begin{itemize}
		\item защитные (технологические), служащие для удаления микротрещин и выколок, появившихся в процессе обработки детали, предохраняющие ее от возможных сколов, трещин и разрушений при закреплении и эксплуатации из-за больших напряжений в этих дефектах под действием различных сил, а также для исключения травм персонала при изготовлении и сборке деталей из-за острых кромок и заусенцев;
		\item конструктивные, служащие для удаления излишков стекла или для базирования детали (центрировка, обеспечение воздушных промежутков между деталями) по плоской, П-образной, конической, сферической формам буртика;
		\item для крепления завальцовкой (закаткой), приклеиванием, планками.
	\end{itemize}
	
	Защитные фаски и фаски для крепления завальцовкой нормализованы для круглых оптических деталей. Размер (ширина) фаски зависит от диаметра детали, от того на склеиваемую или несклеиваемую сторону она наносится, а угол наклона фаски зависит от отношения ее диаметра $D$ к радиусу $R$.
	
	Размер защитных фасок на углах и ребрах некруглых оптических деталей (например, призм) устанавливают в зависимости от длины наиболее короткого ребра. Фаски наносят перпендикулярно биссектрисам трехгранных или двухгранных углов.
	
	\item На преломляющие и отражающие рабочие поверхности оптических деталей обычно наносят оптические покрытия -- тонкие пленки различных веществ: металлов, окислов металлов, диэлектриков, полимерных соединений, кремнийорганических соединений.
	
	Оптические покрытия позволяют изменять оптические характеристики деталей, придавать им новые физические и химические свойства. В зависимости от назначения покрытия подразделяются на следующие группы:
	\begin{itemize}
		\item просветляющие, зеркальные светоделительные, поглощающие (они изменяют интенсивность проходящего и отраженного излучения);
		\item фильтрующие, поляризующие, спектроделителъные (изменяющие спектральный состав, состояние поляризации и фазовые характеристики излучения);
		\item электропроводящие и защитные (они предназначены для обогрева деталей временной и постоянной защиты деталей, изготовленных из химически- и влагонестойких оптических материалов, для гидрофобной и фунгицидной защиты деталей, работающих в условиях морского и тропического климата, а также абразивной защиты недостаточно прочных материалов). Условные обозначения видов покрытий на чертежах оптических деталей указываются в соответствии с ГОСТ~2.412-81. 
	\end{itemize}
	
	Покрытия могут быть одно-, двух-, трех- и многослойные. На чертеже оптической детали, на контуре поверхности ставят условное графическое обозначение покрытия, а на поле чертежа, в технических условиях, после условного графического знака типа покрытия указывают сведения о покрытии.
	
\end{enumerate}

\section{Оформление оптических схем}

Оформление оптических схем согласно ГОСТ~2.412-81 должно выполняться в соответствии со следующими требованиями:
\begin{enumerate}
	\item На оптических схемах детали и узлы, как правило, следует располагать по ходу светового луча, идущего от плоскости предметов слева направо. 
	\item Для сложных приборов оптическую схему основной части прибора и оптические схемы узлов прибора, имеющих самостоятельное назначение, допускается  оформлять отдельными чертежами. На основной схеме такие узлы допускается обводить штрихпунктирной линией.
	\item Все движущиеся детали (вращающиеся или перемещающиеся вдоль или перпендикулярно к оптической оси системы) следует изображать в основном рабочем положении. При необходимости другие положения подвижной детали(например, крайние) могут быть показаны штрихпунктирной линией.
	\item На оптической схеме следует указывать:
	\begin{itemize}
		\item апертурные диафрагмы и положения зрачков;
		\item положения фокальных плоскостей, плоскостей изображения или предмета, положение полевой диафрагмы;
		\item источники света (схематически);
		\item приемники лучистой энергии (схематически или условными графическими обозначениями);
		\item основные оптические характеристики системы в зависимости от типа, при необходимости -- с допусками (увеличение, угловое поле, удаление выходного зрачка, относительное отверстие, предел разрешения, коэффициент светопропускания);
		\item mразличные дополнительные сведения, например расстояние от последней поверхности фотообъектива до плоскости изображения, линейное перемещение окуляра на 1 дптр, при необходимости -- типы и размеры фотокатодов и ПЗС-матриц;
		\item диаметры диафрагмы и размеры зрачков, размеры тела накала или иных светящихся элементов источников света;
		\item воздушные промежутки и другие размеры по оптической оси;
		\item размеры, определяющие пределы перемещения или предельные углы поворота подвижных оптических деталей;
		\item размеры, определяющие положение оптической системы относительно механической части прибора, например размер, определяющий положение объектива микроскопа относительно нижнего среза тубуса;
		\item габаритные или сборочные размеры, например длину базы, высоту выноса (при необходимости).
	\end{itemize}
	\item В таблицах на оптической схеме указывают:
	\begin{itemize}
		\item фокусные расстояния и фокальные отрезки отдельных узлов оптической системы, которые помещают в поле чертежа в виде таблицы;
		\item  размеры световых диаметров оптических деталей и соответствующих им стрелок прогиба ($ sag = R^2 - \sqrt{(R^2 - \dfrac{D^2_\text{св}}{4})} $, где $ R $ -- радиус кривизны поверхности, $ D $ -- световой диаметр), а также толщину по оси (для призм -- длину развертки), которые помещают в поле чертежа в виде таблицы; 
		\item спецификацию -- перечень деталей, входящих в состав оптической схемы с указанием позиции(формат спецификации стандартный), формата и номера чертежа, количества и названия деталей; 
		располагается эта таблица над основной надписью оптической схемы.
	\end{itemize}
\end{enumerate}

Пример оформления оптической схемы представлен на рис.~\ref{pic:3OS}. Пример оформления схемы комбинированной принципиальной показан на рис.~\ref{pic:3Comb}.

\begin{figure}[h!]
	\includegraphics[width=1\textwidth]{3OS.png}
	\label{pic:3OS}
	\caption{Оптическая схема оптико-электронного преобразователя}
\end{figure}

\begin{figure*}[h!]
	\includegraphics[width=1\textwidth]{3Comb.png}
	\label{pic:3Comb}
	\caption{Схема комбинированная принципиальная}
\end{figure*}
\chapter{Линзы и способы их крепления}
\section{Линзы и линзовые блоки (склейки)}

\newthought{Линзы}~--- оптические детали из однородных, прозрачных для оптического диапазона длин волн материалов, ограниченные двумя преломляющими рабочими поверхностями, из которых по крайней мере одна является поверхностью тела вращения (сферическая, асферическая, цилиндрическая, коническая поверхности), применяемые в оптических приборах для преобразования формы пучков излучения и построения изображений различных объектов.

По характеру преобразования пучка различают собирающие и рассеивающие линзы; по сочетанию форм рабочих преломляющих поверхностей (рис.~\ref{pic:5lenses}) их подразделяют на плосковыпуклые (вогнутые) , двояковыпуклые (вогнутые), мениски (с радиусами кривизны, одинаковыми по знаку), бифокальные (с разными радиусами кривизны на частях одной из рабочих поверхностей), линзы Френеля (с плоской и ступенчатой поверхностями) , аксиконы (с плоской и конической поверхностями).

\begin{figure}[h!]
	\includegraphics[width=1\textwidth]{5lenses.png}
	\caption{Линзы}
	\label{pic:5lenses}
\end{figure}

Форма боковой поверхности линзы чаще всего выполняется круглой (цилиндрической), что является наиболее технологичным при изготовлении и закреплении в оправе (иногда форма боковой поверхности выполняется прямоугольной или сегментарной).

\newthought{Конструктивные параметры} линз подразделяют на \textsc{расчетные} и \textsc{конструкторские}.

К \textsc{расчетным параметрам}\marginnote{\allcaps{РАСЧЁТНЫЕ\break ПАРАМЕТРЫ}} (рис.~\ref{pic:5draw},~\ref{pic:5faska}) относят оптические характеристики и показатели качества материала линзы, ее световые диаметры на рабочих поверхностях (диаметр поверхности линзы, пропускающей световой поток), толщину линзы по оптической оси, радиусы кривизны (или параметры формы) преломляющих поверхностей, фокусное расстояние и вершинные фокальные отрезки, допустимые значения погрешностей изготовления оптических поверхностей (погрешности формы, децентрировку\footnote{Децентрировка -- несовпадение оптической оси линзы с геометрической осью}, отклонение толщины по оси), вид оптических покрытий. Эти данные определяются при габаритном, аберрационном, светотехническом расчетах оптической системы.

К \textsc{конструкторским параметрам}\marginnote{\allcaps{КОНСТРУКТОРСКИЕ\break ПАРАМЕТРЫ}} (рис.~\ref{pic:5draw}--\ref{pic:5lensdraw}), относят полный диаметр линзы (или ее размеры, при некруглой форме), параметры фасок, толщину по краю, габаритный размер вдоль оси, чистоту рабочих и шероховатость нерабочих поверхностей, вид покрытия нерабочих (матовых) поверхностей, допуски на погрешности не справочных параметров. Эти параметры определяют в процессе конструирования при окончательном оформлении ее конструкции.

\begin{figure}[h!]
	\includegraphics[width=1\textwidth]{5draw.png}
	\caption{Чертеж линзы без защитных фасок}
	\label{pic:5draw}
\end{figure}

\begin{figure}[h!]
	\includegraphics[width=1\textwidth]{5faska.png}
	\caption{Линза с конструктивной фаской}
	\label{pic:5faska}
\end{figure}

Рассмотрим некоторые аспекты определения конструктивных параметров.
\begin{enumerate}
	
	\item Для закрепления линзы в оправе ее полный диаметр выполняют несколько больше светового. Минимальное значение полного диаметра линзы зависит от светового диаметра и способа закрепления. Окончательный размер полного диаметра округляется до ближайшего (большего) нормального диаметра по действующим стандартам. Поля допусков на полный диаметр линзы должны образовывать в соединении с оправой линзы посадку с зазором, поэтому в зависимости от необходимого значения гарантированного зазора и точности центрирования обычно проставляют следующие допуски на диаметры линз:
	
	$ g6, f7 $ -- высокая точность (технический уровень точности);
	$ h8, f9, e9 $ -- средняя точность (производственный уровень точности);
	$ d9, c11, d11 $ -- пониженная точность (экономический уровень точности).
	
	Заметим, что в соединении линзы с оправой должен быть обеспечен необходимый <<температурный>> зазор, а также то, что точность центрировки линзы в оправе зависит не только от допуска на ее диаметр, но и от выполнения условия самоцентрирования и использования результативной обработки оправы после закрепления линзы.
	
	\item Исходя из требований технологии при конструировании положительных линз необходимо обеспечивать минимальную толщину по их краю в соответствии с рекомендациями, приведенными в соответствующих справочных материалах, а толщина по оси отрицательных линз в зависимости от ее диаметра и необходимой точности формы рабочих поверхностей должна соответствовать рекомендациям действующих стандартов (в пределах от $ 0,05D $ до $ 0,09D  $ в зависимости от диаметра линзы).
	
	Рассчитанные предельные допуски на толщину линз вдоль оси (исходя из их влияния на качество изображения) округляют до ближайшего меньшего значения из ряда значений, приведенного в действующих стандартах [$ \pm(0,005; 0,007;$ $0,010; 0,015; 0,020; 0,025; 0,030-0,05; 0,07; 0,1; 0,2; 0,3; 0,5; 0,7; 1,0) $ мм].
	
	\item Радиусы кривизны рабочих поверхностей сферических линз, полученные при расчетах, округляют до ближайших значений по действующим стандартам, а допуски на них задают в таблице или в технических условиях.
	
	\item На линзах могут быть нанесены следующие фаски (рис.~\ref{pic:5faska}): защитные, конструктивные, для крепления. Параметры защитных фасок и фасок для крепления (размер, угол, допуски) нормализованы (приведены в соответствующих таблицах).
	
	При малой толщине оптической детали по краю размер фаски может быть уменьшен. Фаски на оптических деталях, которые крепятся завальцовкой, должны быть концентричны относительно наружного диаметра.
	
	На выпуклых поверхностях при отношении диаметра к радиусу поверхности больше 1,5 защитную фаску не выполняют; при отношении $ D/R $ от 1,3 до 1,5 фаска допускается, но не является обязательной.
	
	На некоторых линзах, собранных в линзовую систему групповым способом <<насыпным без промежуточных колец>>, защитные фаски на кромках не снимают. Обусловлено это тем, что при применении данного способа крепления линзы в системе устанавливаются друг по другу рабочими поверхностями и кромками (фасками), поэтому значительные погрешности защитных фасок вызывают погрешности воздушных промежутков между компонентами и нарушают центрировку линз в системе (рис.~\ref{pic:5draw}).
	
	Для точной центрировки линзы и обеспечения номинального расстояния между компонентами на соответствующей кромке линзы выполняется конструктивная фаска, которая может быть нанесена не вручную, а при помощи инструмента с последующим контролем ее размера (расположения) и биения (рис.~\ref{pic:5faska}).
	
	\item В качестве материала для линз используется в основном оптическое стекло различных марок. Однако в последнее десятилетие широкое применение получили линзы из оптических полимеров (полиметилметакрилат, полистирол, поликарбонат, сополимер, zeonex), в частности в массовом производстве линз фотографической техники широкого потребления, линз осветительных систем (например, линз Френеля), очковых линз, линз окуляров, лупы, что существенно облегчает их массу и уменьшает стоимость.
	
	Линзы, работающие в инфракрасной и ультрафиолетовой областях спектра, изготавливаются из специальных марок стекол (К515, ИКС), кварцевого стекла (КУ-1), оптической керамики, оптических кристаллов (флюорита, сильвина, фтористого лития, германия).
	
	\item Оптические характеристики линзы: $ f, f' $ -- фокусные расстояния (переднее и заднее); $ S_F, S'_{F'} $  -- передний и задний фокальные отрезки; и расчетные световые диаметры на рабочих поверхностях линз указывают в третьей части таблицы, причем один из фокальных отрезков при необходимости может указываться с допуском.
	
	\item Допуск на децентрировку рабочей поверхности линзы выражают в долях миллиметра и проставляют в поле чертежа, в специальной рамке, содержащем три поля, в первом указывают значок вида допуска децентрировки (позиционный, перпендикулярности или биения плоской поверхности, формы заданной поверхности), во втором -- численное значение допуска, в третьем указывают базы, относительно которых следует контролировать децентрировку (рис.~\ref{pic:5draw}, \ref{pic:5faska}).
	
	При контроле децентрировки круглую линзу (или линзовый блок) устанавливают одной из базовых поверхностей на кольцевую опору, поджимают другой базовой поверхностью к ножевидному упору и приводят во вращение. Измеренное при этом биение центра кривизны рабочей поверхности (или биение плоской поверхности) относительно базовой оси (создаваемой базовыми поверхностями) является мерой децентрировки. Контроль децентрировки некруглых сферических линз, цилиндрических и асферических линз производится с помощью специальных методов и приборов.
	
	Высокий уровень точности центрировки линз соответствует значениям их децентрировки в диапазоне 0,002-0,005 мм, среднему уровню соответствует диапазон 0,005-0,01 мм и пониженному уровню -- 0,01-0,02 мм.
	
	\item На рабочие  поверхности  линзы  могут  быть нанесены различные виды оптических покрытий (просветляющих, зеркальных, светоделительных и поглощающих), а для уменьшения бликов и защиты детали от влияния внешней среды выполняют покрытия их боковых поверхностей и фасок.
\end{enumerate}

\begin{figure}[h!]
	\begin{center}
		\includegraphics[width=0.7\textwidth]{5lensdraw.png}
		\caption{Чертеж линзы без защитных фасок}
		\label{pic:5lensdraw}
	\end{center}
\end{figure}

Одиночные сферические линзы вследствие больших аберраций редко применяются как самостоятельные элементы оптических приборов. Чаще используются комбинации из нескольких линз, склеенные линзы (склейки, линзовые блоки), выполняющие те же функции, что и одиночные линзы, но со значительно меньшими аберрациями.

Большое\marginnote{\allcaps{СКЛЕЙКА ЛИНЗ}} распространение в ОЭП получили склеенные блоки из двух линз (реже трех линз и более) -- положительной и отрицательной, изготовленные из стекол различных марок типа крон и флинт (рис.~\ref{pic:5skleyka}). Они применяются, например, в качестве объективов и оборачивающих линз телескопических приборов. У двухлинзовых склеек могут быть хорошо исправлены сферическая аберрация, хроматизм и кома, другие же аберрации устранить достаточно полно невозможно.

Для склеивания линз (и других оптических деталей) применяют специальные оптические клеи: пихтовый бальзам, бальзамин, бальзамин-М, акриловый, УФ-215М, эпоксидный и другие оптические клеи, которые обладают рядом необходимых свойств и характеристик (высокая прозрачность в спектральном диапазоне, близость показателя преломления к показателям преломлений склеиваемых материалов, оптическая однородность, отсутствие возникновения существенных напряжений при полимеризации, стабильность свойств во времени, тепло- и морозостойкость).

Основные марки оптических клеев и их свойства приведены в действующих стандартах, рекомендации по использованию тех или иных марок клеев при склеивании линз и других оптических деталей в зависимости от условий их работы приведены в соответствующих справочниках.

Чертеж линзового блока оформляется в соответствии с требованиями, приведенными ранее. На чертеже блока, являющегося сборочной единицей, указываются только те параметры, которые должны быть выполнены и проконтролированы в процессе сборки (склейки): центрировка и толщина склеенного блока, отсутствие деформаций наружных рабочих поверхностей, их чистота. Поэтому верхняя часть таблицы -- требования к материалу -- на чертеже склеенного блока линз отсутствует, таблица состоит только из двух частей: требований к сборке и оптических характеристик.

\begin{marginfigure}
	\includegraphics[width=0.6\linewidth]{5skleyka.png}
	\caption{Склейки линз}
	\label{pic:5skleyka}
\end{marginfigure}

В склейке одна из линз является базовой, а другая -- присоединяемой (приклеиваемой). При конструировании деталей, входящих в склейку, как правило, на полный диаметр базовой линзы назначается допуск $ e9 $, для приклеиваемой -- $ d10 $ или $ d11 $. Допускается выполнять приклеиваемую линзу с уменьшенным диаметром по номиналу по сравнению с базовой линзой на 0,2 -- 0,4~мм на диаметр.

При определении базовой и приклеиваемой линз следует учитывать следующее:
\begin{itemize}
	\item в качестве базовой следует выбирать линзу с большей толщиной по краю для удобства базирования при склеивании и контроле готового узла;
	\item в качестве базовой следует выбрать ту деталь, у которой в склейку идет вогнутая поверхность, поскольку клей при склеивании не должен вытекать из соединительного шва, а напротив, должен заполнять все образующиеся пустоты при сопряжении двух линз;
	\item наружный радиус базовой линзы желательно иметь большего значения, чем наружный радиус приклеиваемой линзы для более точного и удобного изготовления склейки;
	\item желательно, чтобы показатель преломления материала базовой линзы не значительно отличался от показателя преломления клея по сравнению с разницей показателей преломления клея и материала приклеиваемой линзы;
	\item необходимо, чтобы радиус базовой поверхности был больше радиуса приклеиваемой поверхности (это касается как базовой, так и приклеиваемой линз);
	\item желательно, чтобы базовая поверхность базовой линзы оставалась базовой поверхностью и для всей склейки;
	\item желательно, чтобы у приклеиваемой линзы базовой являлась та поверхность, которая уходит в склейку.
\end{itemize}

Допуск на децентрировку базовой линзы ставится более жесткий, чем допуск на приклеиваемую линзу. Особенно это касается случая, когда показатель преломления материала приклеиваемой линзы фактически совпадает с показателем преломления клея.

Допуск на суммарную толщину склейки линз рассчитывается следующим образом: $ \Delta d_\text{скл} = \Delta d_\text{баз} + \Delta d_\text{прик} + 0,01$ , например:
\[ d_\text{баз} = 1 \pm 0,01; \]  
\[ d_\text{пр} = 4 \pm 0,02; \] 
\[ d_\text{скл} = 1_{-0,01}^{+0,01}  + 4_{-0,01}^{+0,01} + 0,01 = 5. \]

На наружные рабочие поверхности склейки линз могут быть нанесены оптические покрытия, а боковые поверхности и фаски (матовые поверхности) покрывают защитными эмалями.

\section{Общие требования к оптическим узлам и устройствам}
Сборочные единицы, выполняющие в приборе определенные функции только совместно с другими составными частями, но объединенные в процессе проектирования и изготовления (сборки) в единую систему, называются конструктивными узлами. Они обычно состоят из относительно небольшого количества сопряженных друг с другом деталей, среди которых выделяют рабочую, базовую (оправу, корпус) и вспомогательные (крепежные, ориентирующие, технологические). 

Типичными представителями конструктивных узлов оптических приборов являются узлы крепления оптических деталей и узлы фотоприемников. Прежде чем рассмотреть типовые конструкции таких узлов, перечислим некоторые общие требования к ним и состоящим из них функциональным устройствам.

\begin{enumerate}
	
	\item Конструкция узла должна обеспечить точное расположение рабочей детали (ее рабочих элементов) относительно базовой детали (базового элемента оправы).
	
	\item Крепление должно быть надежным; не допускается изменение положения рабочей детали относительно оправы после закрепления в процессе эксплуатации.
	
	\item В конструкции не должно возникать опасных (объемных) деформаций рабочей и базовой деталей и внутренних напряжений в них при закреплении и в процессе эксплуатации.
	
	При силовом замыкании крепежные элементы не должны вызывать деформации изгиба или кручения. Допускается деформация сжатия (контактная). Для уменьшения деформаций из-за погрешностей размеров, формы и положения элементов деталей между крепежной деталью и оптической следует устанавливать упругие или эластичные прокладки (металлические пружинные кольца, прокладки из пробки, картона, поранита).
	
	Обязательно должно быть обеспечено отсутствие температурных деформаций (или смещений рабочей детали относительно базовой) при перепадах температуры.
	\item Конструкция узла при необходимости должна обеспечивать возможность юстировки рабочей детали. Потребность в юстировке может быть нужна в двух случаях: для точного расположения рабочей детали относительно оправы (например, центрирование линзы при ее сборке относительно базовой оси оправы); для обеспечения требуемого расположения рабочей детали относительно рабочих деталей или баз других узлов (например, фокусировка линзы на фотоприемник). Поэтому для первого случая конструкция узла должна обеспечивать юстировочные подвижки рабочей детали относительно оправы в процессе ее закрепления, для второго~-- подвижки рабочей детали в оправе или вместе с ней (т.е.~всего узла) относительно других узлов в процессе либо после сборки функционального устройства или всего прибора.
	
	\item Конструкция узла должна быть технологичной в отношении изготовления деталей и особенно в отношении их сборки (свободный доступ инструмента, возможность автоматизации сборки, удобство контроля, доступность и простота обслуживания и замены малонадежных элементов).
	
	\item Габаритные размеры узла желательно минимизировать, чтобы обеспечить отсутствие срезания пучка лучей, появление бликов и рассеянного света в системе.
	
	\item Конструкции функциональных устройств должны быть унифицированы по принципам построения, целевым характеристикам, присоединительным размерам, номиналам электропитания.
	
	\item Фактически ни одно функциональное устройство не обходится без юстировки его показателей качества, поэтому надо знать необходимые юстировочные операции типовых функциональных устройств, методику их выполнения (с перечнем необходимого контрольного оборудования), уметь рассчитывать требования к юстировке и заложить в конструкции устройств возможность ее осуществления.
	
\end{enumerate}

Выполнение перечисленных требований основывается на использовании принципов и правил конструирования соединений, узлов и функциональных устройств.

\section{Конструкции узлов крепления круглых оптических деталей и\break линзовых систем}

К основным способам крепления линз и других круглых оптических деталей относятся: крепление завальцовкой, крепление приклеиванием, крепление резьбовым кольцом. При необходимости, когда приходится учитывать особые условия и требования, связанные с габаритными размерами, назначением, условиями эксплуатации оптических деталей, могут использоваться вспомогательные способы крепления: проволочным кольцом, прижимными планками, накладным кольцом, специальными элементами или специальной конструкцией оправы. Указанные названия способов крепления определены видом замыкания рабочей (оптической) детали с базовой (оправой) в соединении или видом крепежной детали.

\newthought{При креплении завальцовкой} \marginnote{\allcaps{КРЕПЛЕНИЕ\break ЗАВАЛЬЦОВКОЙ}} линза поджимается к опорному уступу тонким буртиком, выполненным на оправе. 
Операция завальцовки производится на токарном или сверлильном станках при помощи роликов, специальных инструментов или ультразвуком. 
В результате тонкий буртик оправы (рис.~\ref{pic:5zavalcovka}) деформируется и загибается по всей окружности на специально выполненную фаску линзы.

\begin{figure}[h!]	
	\includegraphics[width=1\textwidth]{5zavalcovka.png}
	\caption[Крепление линзы завальцовкой]{Крепление линзы завальцовкой: а -- размеры буртика;\breakб~-- общий вид}
	\label{pic:5zavalcovka}
\end{figure}

Чтобы не образовалось сколов кромки линзы при загибании буртика, на внутренней поверхности оправы выполняют выборку $ l_2 $. Размеры элементов загибаемого буртика берутся из справочника конструктора оптико-механических приборов. 

\newthought{Преимущества}:
\begin{itemize}
	\item простота и компактность конструкции соединения;
	\item закрепляющий тонкий буртик, обладая упругими свойствами, обеспечивает силовое замыкание линзы без напряжений, а также компенсирует температурные деформации;
	\item возможна автоматизация сборки соединения;
	\item в процессе завальцовки линзы можно выполнять ее частичную центрировку.
\end{itemize} 

\newthought{Недостатки}:
\begin{itemize}
	\item конструкция неразборная; 
	\item существует ограничение на массу закрепляемой линзы (склеенного блока); \marginnote{Ограничение объясняется тем, что тонкий (до 0,1 мм) загибаемый буртик не обеспечивает надежного крепления массивных линз. Выполнение буртика большей толщины исключает его упругие свойства и может привести к выколкам кромки линзы в процессе завальцовки и ухудшить эксплуатационные свойства соединения.} поэтому крепление завальцовкой рекомендуется применять для линз от 6 до 80 мм, а склеенных блоков -- до 50 мм.
	\item технология завальцовки предполагает наличие специальных оборудования, приспособлений и определенной квалификации сборщика;
	\item точность центрирования линзы в оправе может быть несколько ниже, чем при других способах крепления из-за того, что закрепляющий буртик ложится на матовую поверхность нецентрированной фаски линзы.
\end{itemize}

При креплении завальцовкой для оправ обычно используют легко деформируемые, но упругие материалы. 
Наилучшим из них является латунь ЛС59-1. 
Также применяются алюминиевые сплавы Д1, Д6, Д16, В95 и низкоуглеродистые стали Сталь 20, Сталь 30.

Для уменьшения отражения света от стенок оправы подвергаются чернению, а на внутренних поверхностях выполняется рифление.

Завальцовка иногда используется при креплении несклеенных блоков (состоящих из двух-трех) линз и весьма часто -- при креплении сеток, светофильтров, защитных стекол и других деталей, имеющих круглую форму.

\newthought{Крепление приклеиванием} \marginnote{\allcaps{КРЕПЛЕНИЕ\break ПРИКЛЕИВАНИЕМ}} линз и других круглых оптических деталей к оправам в настоящее время является все более и более используемым. 
Причиной этому служит появление новых клеящих веществ с оптимальными свойствами для соединения оптических деталей с оправами (обеспечение надежности соединения, эластичность, отсутствие деформаций в слоях клея, хорошая адгезия к различным материалам, способность сохранять свойства при внешних воздействиях, стабильность во времени).

В результате данные узлы крепления имеют ряд положительных качеств:
\begin{itemize}
	\item конструктивная простота узла крепления, а также снижение его массы и габаритных размеров;
	\item возможность закрепления линз, крепление которых традиционными способами затруднено, например линз малого диаметра (до 6~мм), с крутыми радиусами кривизны и тонкими краями (рис.~\ref{pic:5glue}), при некруглой форме базовых поверхностей;
	\item отсутствие деформаций и напряжений в оптической детали при внешних воздействиях на узел крепления (например, при изменении температуры) благодаря упругим свойствам клеящих веществ;
	\item возможность корректировки положения оптической детали до момента затвердевания клеящего вещества;
	\item обеспечение герметизации соединения;
	\item относительная простота автоматизации процесса сборки.
\end{itemize}

\begin{figure}[h!]
	\begin{center}
		\includegraphics[width=0.6\textwidth]{5glue.png}
		\caption{Крепление линз объектива приклеиванием}
		\label{pic:5glue}
	\end{center}
\end{figure}

Заметим, что этот способ наиболее часто применяется также для крепления линз приклеиванием в случаях, когда они имеют некруглую форму боковых поверхностей.

Качество соединения линзы с оправой зависит от согласованности материалов, входящих в узел крепления компонентов. Для этого необходимо знать физико-механические свойства клеящего вещества, линзы и оправы.

Материал, из которого изготовлены линзы для последующего крепления приклеиванием, может быть любым. 
Чистота обработки поверхности стекла в месте крепления не оказывает существенного влияния на скрепляющее свойство. 
Поэтому приклеиваемая поверхность линзы может быть шлифованной.

Оправы, к которым приклеиваются линзы, изготавливают из алюминиевых сплавов, латуни, стали, титана и его сплавов: ВТ1-0, ОТ4, ВТ-5, ВТ-16. 
Из перечисленных материалов титан благодаря тепловым свойствам, близким к стеклу, является наиболее оптимальным для изготовления оправ линз. Особенность оправ~-- их антикоррозийное покрытие (химическое оксидирование для сталей и анодное оксидирование для цветных металлов).

На чертежах в соответствии с ГОСТ~2.313-82 клеевые швы изображают жирной линией (на разрезах это может быть некоторая область). 
К этой линии подводят выноску, на которой ставят знак К (рис.~\ref{pic:5glue}).

\begin{figure}[h!]
	\includegraphics[width=1\textwidth]{5glue1.png}
	\label{pic:5glue1}
	\caption[Крепление линз приклеиванием]{Крепление линз приклеиванием: а -- с базированием на рабочие элементы оправы; б -- с базированием на клеевой шов;\breakв -- с комбинированным базированием}
\end{figure}

На рис.~\ref{pic:5glue1} показаны три типовые конструкции крепления одиночных линз приклеиванием. 
Мениск базируется в отверстие на уступ оправы (рис.~\ref{pic:5glue1}~а). 
Закрепляющий клеевой шов образуется за счет заполнения клеем сопряженных фасок на линзе и оправе. 
Плосковыпуклая линза устанавливается фаской на клеевой шов (рис.~\ref{pic:5glue1}~б), который наносится на рабочую поверхность оправы. 
Плосковогнутая линза базируется на уступ оправы в осевом направлении (рис.~\ref{pic:5glue1}~в). 
Для образования клеевого шва в посадке линзы и оправы выполняется увеличенный до 0,5 мм зазор, так как очень тонкие слои клея теряют упругие свойства. При невозможности увеличить толщину клеевого слоя между линзой и оправой (например, для повышения точности базирования) в последней выполняется специальная канавка, которая заполняется клеем (рис.~\ref{pic:5glue1}~в).

Способ крепления оптических деталей приклеиванием имеет некоторые недостатки. 
Увеличение объема или усадка клеящего вещества после отвердевания могут вызвать напряжения в линзе. 
Поскольку зазор между линзой и оправой заполнен клеящим веществом, то при перепадах температур, из-за различных расширений этих деталей, возможно расклеивание или возникновение напряжений и деформаций. 

Большая длительность сушки клеящих веществ (от нескольких часов до суток) требует особой технологии, снижает производительность сборки. 
Некоторые компоненты клеящих веществ при определенных условиях (в вакууме) начинают испаряться, что может привести к загрязнению линз. 
Крепление, как правило, неразборное, поэтому не подлежит восстановлению. 
Базирование линзы в оправе на клеевой шов (рис.~\ref{pic:5glue1}~б) не обеспечивает высокую точность ее положения. 
Крепление линз и склеек с большой массой недостаточно надежно и требует его дублирования прижимными деталями.

Влияние ряда недостатков может быть уменьшено. 
Например, для повышения точности расположения линз в конструкции узла необходимо предусмотреть базирование линзы непосредственно на рабочие поверхности оправы (рис.~\ref{pic:5glue1}~а). 

В целом, несмотря на указанные недостатки, способ крепления линз приклеиванием имеет много положительных свойств, выгодно отличающих его от других, главным образом благодаря конструктивной простоте, экономичности, надежности и возможности автоматизации сборки.

\newthought{Крепление резьбовым кольцом}\marginnote{\allcaps{КРЕПЛЕНИЕ\break РЕЗЬБОВЫМ КОЛЬЦОМ}} применяется как разъемное крепление отдельных линз, склеенных и составных линзовых блоков и других круглых оптических деталей. 
Оптическая деталь прижимается к опорному уступу оправы кольцом, имеющим наружную (или внутреннюю) резьбу, по которой оно завинчивается в оправу (рис.~\ref{pic:5ring} а). 
Кольцо завинчивается в оправу специальным ключом, вставляемым в специально выполненные шлицы или отверстия, а для кольца с внутренней резьбой выполняется накатка (рис.~\ref{pic:5ring}~б).

\begin{figure}[h!]
	\begin{center}
		\includegraphics[width=0.5\textwidth]{5ring.png}
		\label{pic:5ring}
		\caption[Крепление линзы резьбовым кольцом]{ Крепление линзы резьбовым кольцом: а -- кольцо с наружной резьбой; б -- кольцо с внутренней резьбой}
	\end{center}
\end{figure}

Резьбовым кольцом рекомендуется крепить линзы с диаметром свыше 10 мм вследствие технологических трудностей выполнения внутренней резьбы в оправах меньшего диаметра, а также из-за относительно больших (по сравнению с размером линзы) габаритных размеров кольца.

Преимуществами данного способа являются обеспечение надежного разъемного крепления, простота сборки и демонтажа, отсутствие ограничений крепления линз относительно большого диаметра (до 300 мм). 

К недостаткам относятся следующие:
\begin{itemize}
	\item конструкция менее технологична, чем при креплении завальцовкой или приклеиванием (так как требуется наличие дополнительной детали, крепежной резьбы в оправе, необходимо предохранять резьбовое кольцо от самоотвинчивания);
	\item узел имеет увеличенные, особенно в осевом направлении, габаритные размеры; затруднена автоматизация сборки соединения;
	\item невозможна юстировка линзы в оправе в процессе сборки;
	\item не всегда можно обеспечить равномерный по всей окружности прижим линзы к опорному уступу оправы, что связано с перекосами кольца в резьбовом соединении, погрешностями формы и положения (отклонение от перпендикулярности) торца кольца и опорного уступа, разнотолщинностью (клиновидностью) линзы по краю;
	\item при работе соединения в условиях перепада температур, из-за его жесткости могут возникнуть либо деформации линзы, либо смещения из-за уменьшения усилия прижатия или даже возникающего зазора между линзой и резьбовым кольцом.
\end{itemize}

Для устранения последних недостатков между резьбовым кольцом и линзой устанавливают пружинное кольцо (рис.~\ref{pic:5ring1}). Вследствие упругости кольца в осевом направлении достигается равномерное по всей окружности распределение давления на линзу и компенсируется влияние температурных деформаций. 

В данной конструкции узла линза поджимается к трем выступам, выполненным в опорном торце оправы. Пружинное кольцо также имеет три выступа, которые ориентированы напротив выступов оправы с помощью винта I, позволяющего   пружинному кольцу смещаться вдоль оси, но ограничивающего его разворот. В результате достигается геометрическая определенность соединения, что обеспечивает минимальные деформации при прижиме.

\begin{figure}[h!]
	\begin{center}
		\includegraphics[width=0.65\textwidth]{5ring1.png}
		\caption{ Крепление линзы с установкой пружинного кольца }
		\label{pic:5ring1}
	\end{center}
\end{figure}

\newthought{Крепление проволочным кольцом} \marginnote{\allcaps{КРЕПЛЕНИЕ\break ПРОВОЛОЧНЫМ\break КОЛЬЦОМ}} применяется для закрепления линз диаметром 20-80~мм при невысоких требованиях к точности их центрирования и герметичности соединения. Он используется в основном для закрепления линз и зеркал в осветительных системах и не силовых оптических деталей (светофильтров, защитных стекол, матовых и молочных рассеивателей).

На рис.~\ref{pic:5ring2} изображены два варианта крепления линзы проволочным кольцом. В первой конструкции (рис.~\ref{pic:5ring2}~а) линза помещена между опорным уступом и выступающей частью проволочного кольца, установленного в прямоугольную кольцевую канавку оправы. Ширина канавки равна диаметру (толщине) проволоки, а глубина -- половине диаметра. Кольцо имеет вырез и изготавливается из стальной углеродистой пружинной проволоки (иногда латунной или бронзовой) диаметром 0,4--1,0мм (рис.~\ref{pic:5ring2}~в).
\begin{figure}[h!]
	\includegraphics[width=1\textwidth]{5ring2.png}
	\caption[Крепление проволочным кольцом]{ Крепление проволочным кольцом: а~-- оправа с прямоугольной канавкой; б~-- оправа с конической канавкой; в -- проволочное кольцо }
	\label{pic:5ring2}
\end{figure}

Условие установки проволочного кольца в оправу: $ D_{min} < D_\text{оп} $, где $ D_{min} $ -- диаметр сжатого кольца; $ D_\text{оп} $ -- диаметр отверстия оправы; а условие крепления линзы в оправе: $ d_{max} < D_\text{л} $, где $ d_{max} $ -- внутренний диаметр кольца, установленного в оправу; $ D_\text{л} $ -- диаметр линзы.

Крепление проволочным кольцом конструктивно простое и технологичное. Кольцо может быть быстро установлено или снято.

Недостатком такого способа крепления является возможность смещения и перекашивания линзы в оправе, которые возникают из-за осевого зазора (обусловленного погрешностями размеров канавки, толщины линзы по краю, диаметра проволоки).

Во второй конструкции в оправе выполнена конусная канавка под проволочное кольцо (рис.~\ref{pic:5ring2}~б). В месте контакта кольца с наклонной плоскостью канавки возникает сила реакции, осевая составляющая которой прижимает линзу к опорному уступу. Для надежности соединения угол   конусной канавки должен быть меньше угла трения.

\newthought{Крепление пружинящими планками} \marginnote{\allcaps{КРЕПЛЕНИЕ\break ПРУЖИНЯЩИМИ\break ПЛАНКАМИ}} применяется для линз, работающих в условиях перепадов температур, динамических воздействий и в случаях, когда они имеют не круглую форму боковых поверхностей. 
Упруго деформируясь, планки компенсируют действие факторов, ухудшающих качество соединения. 
Пример крепления пружинящими планками показан на рис.~\ref{pic:5planka}.

\begin{figure}[h!]
	\includegraphics[width=1\textwidth]{5planka.png}
	\caption[Крепление пружинящими планками]{ Крепление пружинящими планками: а -- тремя планками; \breakб -- кольцевой планкой }
	\label{pic:5planka}
\end{figure}

Планки изготавливают из лент холоднокатаной инструментальной или пружинной стали (65Г, У8А), нейзильбера (НМцб5-20) и устанавливают через 120$^\circ$ по окружности оправы~(рис.~\ref{pic:5planka}~а). Каждая планка привинчивается к оправе двумя винтами.

Для крепления линз малого диаметра три планки заменяют одним кольцом с тремя пружинящими выступами (рис.~\ref{pic:5planka}~б). Возможны и другие конструктивные реализации пружинящих планок и их соединений с оправой.

Конструкция узла характеризуется достаточной точностью и надежностью соединения линзы с оправой.

Преимуществами, которыми обладает данный способ крепления, являются следующие: возможность регулировать усилие прижима; создание упругого соединения, позволяющее компенсировать погрешность осевых размеров сопрягаемых деталей и их изменение от внешних воздействий; возможность разборки конструкции; отсутствие необходимости в специальном оборудовании и квалифицированном персонале для сборки узла. К недостаткам следует отнести: нетехнологичность конструкции, содержащей большое количество крепежных элементов; сложность автоматизации сборки; невозможность юстировки в процессе сборки.

\newthought{Крепление накладным кольцом} \marginnote{\allcaps{КРЕПЛЕНИЕ\break НАКЛАДНЫМ\break КОЛЬЦОМ}} применяют для крепления крупногабаритных линз, а также других круглых оптических деталей (защитных стекол, зеркал) с диаметром, превышающим 200-300~мм. 
Крепление накладным кольцом относится к индивидуальным способам крепления. 
Его реализация зависит от конкретных геометрических параметров оправы и линзы, их допустимых отклонений, а также от температурного режима работы соединения.

Схема крепления показана на рис.~\ref{pic:5nakladnoe}. 
Линзу устанавливают в оправу на фаску опорного буртика, выполненную под углом 135$^\circ$ или по касательной к рабочей поверхности линзы, к которому она прижимается накладным кольцом. 
Рабочая поверхность прижимного кольца тоже выполняется по касательной к поверхности линзы и крепится к оправе болтами Для компенсации погрешностей изготовления соответствующих размеров оправы, линзы и накладного кольца и их изменений при отклонениях температуры между контактирующими поверхностями линзы и кольца устанавливается упругая прокладка.

\begin{marginfigure}
	\begin{center}
		\includegraphics[width=0.6\linewidth]{5nakladnoe.png}
		\caption{Крепление линзы накладным кольцом}
		\label{pic:5nakladnoe}
	\end{center}
\end{marginfigure}

Для центрирования линзы в оправе могут применяться вкладыши (например, полоски фольги толщиной 0,005 мм), которые устанавливают в зазор между посадочным отверстием оправы и линзой. 
Иногда линзу центрируют в оправе сдвигом (наклоном) винтами с последующей фиксацией герметиком.
Для компенсации возможных пережатий или смещений линзы при перепадах температуры в ряде случаев между диаметрами линз и отверстиями оправы устанавливают термокомпенсаторы.

Данный способ крепления обладает рядом недостатков: трудоемкость сборки узла из-за подгонки прижимного кольца; увеличенные габаритные размеры конструкции, так как накладное кольцо выступает за пределы оправы; невозможность автоматизации сборки.

К преимуществам способа следует отнести: надежность крепления; возможность частичной юстировки положения линзы относительно оправы; возможность применения термокомпенсаторов.

\newthought{Специальные способы крепления} \marginnote{\allcaps{СПЕЦИАЛЬНЫЕ\break СПОСОБЫ\break КРЕПЛЕНИЯ}} --- способы, крепления, применяемые в особенных случаях, например: крепление линзы в оправе стопорными винтами (или привинчивание линзы винтами к оправе через просверленные в ней отверстия); заливка линзы в оправе зубным или глетоглицериновым цементом; заформовка линзы в оправу из термопластичных пластмасс; обжатие линзы <<хомутовыми>> или разъемными оправами. 
На рис.~\ref{pic:5obj} изображена конструкция проекционного объектива, линзы которого закреплены в разъемной пластмассовой оправе.

\begin{figure*}[h!]	
	\includegraphics[width=1\textwidth]{5obj.png}
	\caption{ Проекционный объектив }
	\label{pic:5obj}
\end{figure*}

\section{Конструкции линзовых систем}

К линзовым системам оптических приборов относятся объективы, окуляры, оборачивающие системы, системы смены увеличения, конденсоры и коллекторы. Как правило, эти функциональные устройства состоят из нескольких или большого количества линз и склеенных блоков (в некоторых из них содержатся также и другие оптические детали: сетки, зеркала, защитные стекла, светофильтры, рассеиватели).

В зависимости от способа установки и сопряжения этих оптических деталей с несущим элементом (корпусом) устройства конструкции линзовых систем подразделяют на насыпные, насыпные в оправах, резьбовые, комбинированные и специальные. 

В насыпных конструкциях линзы (и прочие детали) устанавливаются последовательно друг за другом (насыпаются) непосредственно в корпусную деталь. Необходимые воздушные промежутки между компонентами выдерживаются здесь с помощью промежуточных колец (рис.~\ref{pic:5condensor}) либо точным изготовлением их конструктивных параметров (диаметров и фасок, рис.~\ref{pic:5photoobj}). Точность центрировки компонентов системы обуславливается погрешностями центрировки самих линз, зазорами их посадок в корпус, наклонами из-за перекоса опорного торца корпуса, клиновидностями промежуточных колец или биением опорных фасок, а также несоосностью посадочных рабочих поверхностей корпуса (рис.~\ref{pic:5condensor}~б,~в).

\begin{figure*}[h!]
	\begin{center}
		\includegraphics[width=1\textwidth]{5condensor.png}
		\caption{ Насыпные с промежуточными кольцами конструкции конденсоров }
		\label{pic:5condensor}
	\end{center}
\end{figure*}

\begin{figure}[h!]
	\begin{center}
		\includegraphics[width=0.4\textwidth]{5photoobj.png}
		\caption{ Насыпная без промежуточных колец, конструкция фотообъектива }
		\label{pic:5photoobj}
	\end{center}
\end{figure}

Так как обеспечить высокую точность центрировки многокомпонентной линзовой системы из-за перечисленных погрешностей весьма сложно, а юстировка центрировки при насыпной конструкции затруднена или невозможна, то ее используют обычно в конструкциях осветительных систем (конденсоров, коллекторов), окуляров и относительно простых объективов. Данная конструкция не используется также в случаях, когда линзы системы существенно отличаются друг от друга по световому диаметру.

Насыпная конструкция (особенно без промежуточных колец) является наиболее технологичной, так как содержит минимально возможное количество деталей. Поэтому наблюдается устойчивая тенденция все более частого ее использования в конструкциях линзовых узлов приборов.

Насыпная в оправах конструкция отличается от предыдущей тем, что линзы и компоненты вначале закрепляются тем или иным способом (чаще всего завальцовкой или приклеиванием) в своих оправах, а затем устанавливаются последовательно в корпусную деталь (рис.~\ref{pic:5photoobj1},~\ref{pic:5photoobj2}).

\begin{figure}[h!]
	\includegraphics[width=1\textwidth]{5photoobj1.png}
	\caption{ Насыпная в оправах \breakконструкция фотообъектива }
	\label{pic:5photoobj1}
\end{figure}

\begin{figure}[h!]
	\includegraphics[width=1\textwidth]{5photoobj2.png}
	\caption{ Фотографический \breakпроекционный объектив }
	\label{pic:5photoobj2}
\end{figure}

Воздушные промежутки между компонентами обеспечиваются точным выполнением соответствующих конструктивных размеров оправ компонентов (при необходимости воздушный промежуток может юстироваться). Точность центрировки компонентов системы обуславливается погрешностями расположения центров кривизны их поверхностей и неперпендикулярностью плоских поверхностей относительно базовых осей оправ, зазорами посадок оправ в корпус, наклонами оправ из-за перекоса опорного торца корпуса и клиновидности оправ, несоосностью посадочных рабочих поверхностей корпуса (рис.~\ref{pic:5photoobj1}).

Насыпная в оправах конструкция применяется обычно при конструировании многокомпонентных фотобъективов (рис.~\ref{pic:5photoobj1}), микрообъективов, проекционных и фотограмметрических объективов (рис.~\ref{pic:5photoobj2}), зеркально-линзовых объективов.

На рис.~\ref{pic:5photoobj2} представлена конструкция многолинзового проекционного объектива, компоненты которого завальцованы каждый в собственную оправу и установлены в общий корпус. Силовое замыкание выполняется резьбовым кольцом.

В резьбовых конструкциях линзы и компоненты закрепляются каким-либо способом в своих оправах, которые соединяются по резьбе с корпусной деталью (рис.~\ref{pic:5photoobj3}).

\begin{figure}[h!]
	\begin{center}
		\includegraphics[width=0.8\textwidth]{5photoobj3.png}
		\caption{ Резьбовая конструкция оправы фотообъектива }
		\label{pic:5photoobj3}
	\end{center}
\end{figure}

Резьбовая конструкция является наименее технологичной из рассмотренных выше, так как более трудоемка при изготовлении и сборке, поэтому в настоящее время используется относительно других гораздо реже. В этой конструкции практически невозможно осуществлять юстировку центрировки компонентов системы.

В комбинированных конструкциях компоненты линзовых, систем сопрягаются с несущей (корпусной) деталью различными способами: непосредственно устанавливаются в корпус, насыпаются в оправах или их оправы соединяются с корпусом по резьбе (рис.~\ref{pic:5helios}).

На рис.~\ref{pic:5helios} изображена конструкция фотообъектива <<Гелиос-44Н>>, компоненты 1 и 2 которого закреплены в оправе, соединяемой с корпусом по резьбе, а компоненты 3 и 4 установлены в корпусную деталь насыпным способом. 

На рис.~\ref{pic:5Canon} изображена конструкция фотообъектива Canon, в которой присутствуют рассмотренные ранее различные конструкции крепления линз.

В специальных конструкциях линзы и компоненты устанавливаются и закрепляются в корпусной детали нетрадиционным способом. На рис.~\ref{pic:5obj} представлена конструкция проекционного объектива, линзы которого (две из них 1 и 3 выполнены из полистирола, а третья 2 -- из силикатного стекла) установлены в призматических канавках литой пластмассовой общей оправы, выполненной из двух цилиндрических половинок 4, 5. Крепление линз осуществляется обжимом их половинками оправы, вставленной в корпусную деталь. Благодаря упругости тонких буртиков призматических канавок производится беззазорное сопряжение линз с оправой. Точность расположения линз достигается точным литьем элементов оправы.

Более подробные сведения о конструкциях тех или иных видов и типов линзовых систем: объективов (например, фотографических, телескопических, проекционных, микроскопических, зеркально-линзовых), окуляров, осветительных систем изложены в справочниках и специальной литературе.	
\begin{figure}[h!]
	\begin{center}
		\includegraphics[width=0.7\textwidth]{5Helios.png}
		\caption{ Фотообъектив <<Гелиос-44Н>> }
		\label{pic:5helios}
	\end{center}
\end{figure}

\begin{figure*}[h!]
	\includegraphics[width=1\textwidth]{5Canon.png}
	\label{pic:5Canon}
	\caption{ Фотообъектив Canon }
\end{figure*}

\chapter{Призмы, зеркала и способы их крепления}
\section{Призмы}

\newthought{Призмы} \marginnote{\allcaps{ПРИЗМЫ}} --- оптические детали или оптические системы деталей (объединенные в единый блок) с плоскими рабочими поверхностями (гранями) на которых происходит преломление или отражение оптического излучения.

В оптических приборах призмы применяют в следующих целях:
\begin{itemize}
\item для изменения хода лучей, направления оптической оси системы и направления линии визирования;
\item оборачивания изображения;
\item уменьшения габаритных размеров системы;
\item разделения или объединения пучков лучей, полей или изображений;
\item вращения  изображения  или  компенсации  поворота изображения;
\item сканирования изображения или модулирования излучения;
\item разложения света в спектр;
\item поляризации света;
\item юстировки и аттестации приборов, создания измерительных баз.
\end{itemize}

Призмы подразделяют обычно на две группы: отражательные и спектральные. 

К группе спектральных относят также поляризационные, модулирующие и отклоняющие излучение призмы на основе физических эффектов в их материалах при воздействии на них электрических или магнитных полей.

Самой многочисленной группой являются отражательные призмы, на примере которых рассмотрим некоторые аспекты их конструирования.
По своему действию на световой пучок отражательные призмы подобны зеркалам, однако в ряде случаев призмы более эффективны, чем зеркала.

Преимущества отражательных призм по отношению к зеркалам:
\begin{itemize}
\item углы между гранями призмы неизменны, тогда как углы между зеркалами должны регулироваться с большой точностью при сборке и могут разъюстироваться в процессе эксплуатации;
\item потери света у призм от граней с полным внутренним отражением равны нулю, тогда как при отражении от поверхностей зеркал потери довольно велики; кроме того, отражающие покрытия зеркал с течением времени могут портиться;
\item конструкция крепления призм в оправах, как правило, проще, чем у системы зеркал, и обладает меньшими габаритными размерами;
\item для некоторых призм нет эквивалентных зеркальных систем (например, для призмы Дове, полупенты, некоторых видов спектральных призм).
\end{itemize}
 
Замена отражательных призм зеркалами целесообразна в случаях:
\begin{itemize}
\item когда имеют значение масса прибора, так как зеркала значительно легче призм; 
\item при высокой стоимости оптического материала;
\item для достижения требуемого качества изображения, так как призмы являются источниками хроматических и других аберраций, особенно в случаях их работы в сходящемся пучке лучей.
\end{itemize}

Рабочие и нерабочие поверхности (грани) призмы представляют собой плоскости. Рабочие поверхности подразделяют на преломляющие, через которые световой пучок входит в призму или выходит из нее, и отражательные, от которых пучок отражается при прохождении внутри призмы.

Число рабочих граней и взаимное их расположение определяют ход пучка внутри призмы и все преобразования пучка, которые при этом происходят.
Если осевой луч проходит внутри призмы в одной плоскости, то такую призму называют плоской. Если осевой луч идет в двух плоскостях, такая призма называется пространственной.

Сечение призмы плоскостью, в которой проходит осевой луч пучка, называется главным, сечением, призмы; у плоских призм одно главное сечение, у пространственных главных сечений столько, сколько плоскостей, в которых проходит осевой луч.

Отражательные призмы подразделяют на простые (их называют также одинарными), выполненные из одной заготовки материала, и составные (призменные блоки), представляющие собой комбинации из двух или большего числа простых призм, объединенных в единый блок с помощью склейки или закрепления в оправе (рис.~\ref{pic:6prism}).

\begin{figure*}[h!]
	\caption[Отражательные призмы с одним и двумя отражениями]{ Отражательные призмы с одним (а, в) и двумя отражениями (б) }
	\includegraphics[width=1\textwidth]{6prism.png}
	\label{pic:6prism}
\end{figure*}

Основными целевыми характеристиками отражательных призм являются: угол отклонения светового пучка, линейное смещение пучка, оборачивание изображения, степень возможности разделения или совмещения пучков лучей.

Углом отклонения называется угол между направлениями осевого луча до и после призмы, причем промежуточные отклонения луча внутри призмы во внимание не принимаются.

Линейным, смещением пучка называют расстояние между параллельными направлениями осей падающего на призму и выходящего из призмы пучка лучей. Если это расстояние равно нулю (направления осей падающего и прошедшего пучка совпадают), то такие призмы называют призмами прямого зрения (видения).

Оборачивание изображения зависит от числа отражающих граней и их расположения в пространстве.

Плоские призмы с четным числом отражающих граней дают прямое изображение. При наклоне такой призмы в главной плоскости выходящий пучок лучей не отклоняется.

Плоские призмы с нечетным числом отражающих граней дают зеркальное изображение предмета. При наклоне их в плоскости главного сечения лучи отклоняются на двойной угол.

Для оборачивания изображения в плоскости, нормальной к главному сечению, одна из отражающих граней призмы заменяется крышей, которая представляет собой две отражающие поверхности, образующие двугранный угол 90$ ^\circ $, симметрично расположенные относительно главного сечения призмы (рис.~\ref{pic:6roof}).

\begin{figure}[h!]
	\includegraphics[width=1\textwidth]{6roof.png}
	\caption{ Призмы с крышей }
	\label{pic:6roof}
\end{figure}

Степень возможности разделения или объединения пучка лучей призмой (призменным блоком) определяется способностью разделять (объединять) пучок на две, три или более составляющих (как правило, это составные призмы типа призмы-куб, Кестерса, цветоделительной).

Типовые простые призмы, имеют условное обозначение в виде двух букв и числа, разделенных знаком тире.

Первая буква указывает число отражающих граней призмы (А -- одно отражение, Б -- два, В -- три), вторая --- характер ее конструкции (Р --  равнобедренная, П -- пентапризма\footnote{Пентапризма~-- общее название оптического устройства, служащее для поворота оси светового потока на 90$^\circ$ и удлинения его пути за счёт двух и более отражений от зеркальных поверхностей, обеспечивает минимальные внешние габаритные размеры всей сложной оптической системы.}, У~-- полупента, С~-- ромбическая, Л~-- призма Лемана). Число обозначает угол отклонения осевого луча в градусах. При этом крыша считается за одну грань. Обозначается крыша индексом <<К>> у первой буквы. Для пространственных призм указываются углы отклонения в соответствующих плоскостях по ходу луча.

Типовые составные призмы, имеют другие условные обозначения. Буквой обозначают тип призмы (например, А -- Аббе-призма, К -- куб-призма, Б -- башмачная, П -- Пехана-призма), цифрой -- угол отклонения.

Составные призмы применяются в тех случаях, когда простые призмы не могут обеспечить необходимые целевые характеристики или не могут быть установлены в сходящемся пучке лучей (так как разворачиваются в наклонную плоскопараллельную пластинку и вносят большие аберрации) либо когда требуется уменьшить габаритные размеры системы.

На рис.~\ref{pic:6sostav} и ~\ref{pic:6spatial} представлены составные и светоделительные призмы.
\begin{figure}[h!]
	\includegraphics[width=1\textwidth]{6sostav.png}
	\caption[Составные призмы]{ Составные призмы:\break а, б -- башмачные, в -- Пехана, \breakг -- Аббе }
	\label{pic:6sostav}
\end{figure}

\begin{figure}[h!]
	\begin{center}
		\caption[Пространственные призменные системы]{ Пространственные \breakпризменные системы }
		\includegraphics[width=0.7\textwidth]{6spatial.png}
		\label{pic:6spatial}
	\end{center}
\end{figure}

На рис.~\ref{pic:6spatial} приведены составные пространственные призмы, использующиеся как оборачивающие призменные системы --- призменные системы Малафеева-Порро первого (рис.~\ref{pic:6spatial}~а) и второго рода (рис.~\ref{pic:6spatial}~б).

\newthought{Конструктивные параметры} призм, так же как и для линз, подразделяют на \textsc{расчетные} и \textsc{конструкторские}.

К \textsc{расчетным параметрам}\marginnote{\allcaps{РАСЧЁТНЫЕ\break ПАРАМЕТРЫ}} относят: оптические характеристики и показатели качества материала призмы, ее световые диаметры $ O_\varnothing $ на рабочих поверхностях, длину хода луча в призме $ l $, допустимые значения погрешностей изготовления рабочих оптических поверхностей (погрешности формы $ N , \Delta N$), погрешности углов призмы, влияющие на качество изображения, пирамидальность $ \pi $, вид оптических покрытий, а также (при необходимости) значения допустимой фокусности $ f_{min} $ и предел разрешения $ \varepsilon $. 
Эти данные определяются при габаритном, аберрационном и светотехническом расчетах оптической системы.

К \textsc{конструкторским параметрам}\marginnote{\allcaps{КОНСТРУКТОРСКИЕ\break ПАРАМЕТРЫ}} относят габаритные размеры призмы (которые зависят от типа призмы, ее световых диаметров, запаса для крепления, юстировки), параметры фасок на ребрах и углах, допуски на углы, не влияющие на качество изображения, класс чистоты рабочих полированных поверхностей, шероховатость рабочих и нерабочих поверхностей, покрытия матовых поверхностей. 
Эти параметры получают в процессе разработки ее окончательной конструкции.

\section[Узлы крепления одиночных призм и призменных систем]{Узлы крепления одиночных призм и\\ призменных систем}

Призмы и призменные системы, применяемые в оптических приборах, характеризуются многообразием форм и размеров. В связи с этим существует большое количество разнообразных конструкций узлов крепления одиночных призм и составных блоков призм, достаточно полно рассмотреть которые в объеме курса лекций не представляется возможным. Рассмотрим некоторые из них.

Типовые способы крепления одиночных призм обычно классифицируют по виду основной детали, осуществляющей прижим (замыкание) призмы к рабочей поверхности оправы. Поэтому различают крепление призм накладкой, прижимными планками (лапками), угольниками, установочными винтами, пружинами, специальными деталями. Для крепления призм также используют клеи и замазки.

При разработке узла крепления любой призмы необходимо соблюдать следующие рекомендации:
\begin{enumerate}[leftmargin=*]
\item Для обеспечения точности положения призмы и исключения деформаций изгиба рабочая плоскость (плоскости) оправы должна быть чисто обработана ($ R_a $=1,25 $ \div $ 3,2 мкм) и иметь высокую степень плоскостности. При относительно больших размерах призмы на рабочей поверхности делают выборку (для выполнения принципа геометрической определенности соединения), и тогда призма базируется на два выступа по краям, либо применяют базирование на три опорные площадки.
\item Чтобы не нарушать требуемые условия преломления и отражения на рабочих гранях призмы, рекомендуется базировать призму на нерабочую грань. При базировке призмы на рабочую грань (грани) сопряжение ее с оправой должно происходить за пределами светового диаметра. При необходимости использовать для крепления или ориентации призмы грани, работающие с полным внутренним отражением, необходимо минимизировать площадь контакта между гранью и деталью крепления, например, реализовать контакт между указанными элементами по линии или точечный контакт.
\item Не допускается контакт крепежных элементов с ребрами призмы во избежание выколок стекла.
\item Между призмой и крепежным элементом (за исключением пружины) следует ставить эластичную прокладку из пробки, картона, паронита, текстолита, противоосыпочной резины или силиконового герметика, которая компенсирует погрешности размеров деталей, равномерно распределяет усилие прижима на большую площадь, предотвращает появление температурных деформаций и смещений.
\item Сопряжение призмы с рабочей поверхностью оправы отнимает три степени свободы. Базирование призмы двумя рабочими плоскостями на две плоскости оправы отнимает пять степеней свободы. Оставшиеся степени свободы призмы отнимаются соответствующим количеством ориентирующих планок.
\item Для юстировки призм следует предусматривать в конструкции их узлов возможность выполнения необходимых юстировочных подвижек.
\item Крепление склеенных призменных блоков осуществляется за одну из них, а именно -- базовую, в качестве которой выбирается наиболее массивная призма. Приклеиваемые к ней другие призмы должны иметь меньшую массу и по возможности не касаться элементов оправы.
\end{enumerate}

\newthought{Крепление накладкой} \marginnote{\allcaps{КРЕПЛЕНИЕ НАКЛАДКОЙ}} применяется для любых призм, имеющих параллельное расположение нерабочих граней (прямоугольная призма, пентапризма, призма Шмидта). 
Например, прямоугольная призма установлена нерабочей гранью на плоскость оправы (рис.~\ref{pic:6nakladka}). 
К оправе 1 призма прижимается накладкой~3 через эластичную прокладку 2. 
Накладка крепится винтами к двум стойкам, жестко соединенным с оправой. Ориентация призмы вдоль плоскости оправы выполняется при помощи трех ограничительных (ориентирующих) планок~4.

\begin{figure}[h!]	
	\includegraphics[width=0.8\textwidth]{6nakladka.png}
	\caption{ Крепление прямоугольной призмы накладкой }
	\label{pic:6nakladka}
\end{figure}

Крепление накладкой характеризуется универсальностью, относительной простотой оправы и простотой сборки, надежностью, реализуется принцип полного внутреннего отражения на отражающих гранях, возможна юстировка призмы в оправе.

\newthought{Крепление уголками} \marginnote{\allcaps{КРЕПЛЕНИЕ УГОЛКАМИ}} основано на использовании в узле крепления нескольких угольников различной формы (Г-образных, Z-образных). 
Дополнительно к угольникам применяют ориентирующие планки. 
Типовые конструкции крепления данным способом показаны на рис.~\ref{pic:6ugol}.

\begin{figure}[h!]
	\begin{center}
		\caption{ Крепление призм угольниками }
		\includegraphics[width=0.8\textwidth]{6ugol.png}
		\label{pic:6ugol}
	\end{center}
\end{figure}

Пентапризма (рис.~\ref{pic:6ugol}~а) прижимается к плоскости оправы двумя угольниками 1. Усилие прижима призмы создается в результате деформации эластичных прокладок, помещенных между призмой и угольниками. Ориентирование призмы на плоскости оправы произведено при помощи угловой планки 2, которая контактирует с преломляющими гранями призмы. После установки призмы в рабочее положение планка фиксируется штифтами.

На рис.~\ref{pic:6ugol}~б показан пример крепления угольниками прямоугольной призмы. Особенностью данной конструкции является то, что в нерабочих гранях закрепляемой призмы выполнены прямоугольные канавки. Здесь для прижима призмы к основанию оправы используются элементы канавок, через которые низкими угольниками 1 призма прижимается к основанию. Ограничение бокового перемещения призмы обеспечивается ориентирующими планками 2. Планки винтами крепятся к основанию оправы.

Этот способ крепления призм получил довольно широкое практическое применение благодаря простоте и надежности. Недостатками являются усложнение технологии изготовления снабженных канавками призм и возможность их скола при креплении.

\newthought{Крепление прижимными планками} \marginnote{\allcaps{КРЕПЛЕНИЕ\break ПРИЖИМНЫМИ\break ПЛАНКАМИ}} применяется для крепления сложных призм (призм с крышей, склеенных блоков), когда необходимо применить нетрадиционное базирование призмы, например -- прямоугольная призма устанавливается на гипотенузную грань.

Характерным для этого способа крепления является более сложная конструкция оправы или прижимных планок. Как правило, оправа охватывает призму с трех сторон. При этом одна (или две) поверхности базирующие, а на других --- закрепляются прижимные планки. Прижимные планки бывают различными по конструкции: в виде лапок, угольников, пластин, согнутых пластин и т. п. На рис. 7 а показано крепление прямоугольной призмы с крышей. Призма установлена на катетную грань между стенками оправы с гарантированным зазором, который при необходимости выбирается эластичной прокладкой. Две пары прижимных планок ограничивают перемещение призмы в продольном и вертикальном направлениях.

На рис.~\ref{pic:6planka}~б показано крепление прямоугольной призмы, установленной гипотенузной гранью на базирующую поверхность оправы. Четыре прижимные планки привинчены к вертикальным стенкам оправы. Контакт планок с призмой выполнен по краевым зонам ее входной и выходной поверхностей. Планки выполняют упругими, либо между ними и призмой устанавливают эластичные прокладки.

\begin{figure}[h!]
	\includegraphics[width=1\textwidth]{6planka.png}
	\caption{ Крепление призм \breakприжимными планками }
	\label{pic:6planka}
\end{figure}

Недостатком этого способа крепления является неуниверсальность и сложность деталей узла. В частности, требуется изготавливать сложные формы оправ и оригинальные прижимающие планки для каждого типа призм. Базирование на рабочие поверхности может нарушить требуемые условия отражения и вызвать воздействие недопустимых усилий. Юстировка призмы практически невозможна.

\newthought{Крепление установочными винтами} \marginnote{\allcaps{КРЕПЛЕНИЕ\break УСТАНОВОЧНЫМИ\break ВИНТАМИ}} рекомендуется применять для закрепления относительно больших призм с размером грани, превышающим 30 мм, когда требуется распределить усилие прижима равномерно по всей грани.

В варианте крепления пентапризмы, показанном на рис.~\ref{pic:6screw}, замыкание призмы на основание осуществляется тремя установочными винтами. 
Между призмой и винтами помещены эластичная и металлическая прокладки. 
Размеры пластин соответствуют размерам грани призмы. 
Металлическая пластина позволяет равномерно распределить усилие прижима по всей грани, а также предотвращает возникновение повреждений призмы при сборке. 

В свою очередь, эластичная прокладка компенсирует погрешности формы металлической пластины и изменение размеров деталей при изменении температуры. Смещения и повороты призмы вдоль основания оправы ограничиваются двумя ориентирующими планками, закрепленными на основании оправы.

\begin{figure}[h!]
	\includegraphics[width=0.9\textwidth]{6screw.png}
	\caption{ Крепление пентапризмы установочными винтами }
	\label{pic:6screw}
\end{figure}

Данная конструкция характеризуется простотой используемых деталей и надежностью фиксирования призмы. Установочными винтами можно регулировать усилие прижима. Призма защищена от внешних воздействий корпусом оправы с трех сторон.

В конструкции необходимо предусмотреть защиту от самоотвинчивания установочных винтов.

\newthought{Крепление пружинами} \marginnote{\allcaps{КРЕПЛЕНИЕ\break ПРУЖИНАМИ}} широко используется для закрепления призм, имеющих относительно много рабочих граней (например, прямоугольная призма с крышей). 
Это позволяет упростить конструкцию узла крепления. 
Вместе с этим пружины целесообразно использовать для прижима призмы при ее установке непосредственно в корпусную деталь прибора. 
В узлах крепления призм применяют пружины различных форм: тарельчатые, плоские, изогнутые, которые обычно изготавливают из пружинной стали 30X13, 65Г, У8А.

На рис.~\ref{pic:6spring} изображено крепление пружиной прямоугольной призмы с крышей. К специальной фаске на ребре призмы (90$ ^\circ $) приклеивается цилиндрический шарнир. Шарнир установлен в паз, который выполнен в корпусной детали. Вместе они образуют направляющую вращения, относительно которой можно выполнять регулировочные наклоны призмы.

Силовое замыкание призмы осуществляется изогнутой пластинчатой пружиной с закругленными краями, которые опираются на грани крыши. Регулировка усилия пружины выполняется двумя установочными винтами. При этом сила давления на призму, работающую при ударных нагрузках должна быть в 15-20 раз больше массы призмы, при работе в лабораторных приборах~-- от двух до пяти раз. Для ограничения перемещения призмы в плоскости, перпендикулярной к главному сечению (вид А-А), между призмой и крышкой корпуса установлена пластина с эластичной прокладкой.

\begin{figure}[h!]
	\begin{center}
		\includegraphics[width=0.7\textwidth]{6spring.png}
		\caption{ Крепление прямоугольной призмы с крышей пружиной }
		\label{pic:6spring}
	\end{center}
\end{figure}

Данная конструкция крепления позволяет выполнять угловые юстировочные подвижки призмы. Для этого при помощи двух установочных винтов изменяют усилие прижима пружины к призме. Наличие пружины обеспечивает устойчивость крепления к воздействию вибраций, толчков, а также компенсирует воздействие на узел крепления изменений температуры.

К недостаткам следует отнести необходимость физического контакта пружины с рабочими отражающими гранями призмы, который может привести не только к повреждению призмы при сборке и юстировке, но и к исчезновению эффекта полного внутреннего отражения. Поэтому площадь контакта должна быть минимальной, по возможности вне зоны светового пучка.

На рис.~\ref{pic:6planspring} показан фрагмент крепления призмы плоской пружиной. Здесь призма установлена нерабочей гранью на плоское основание оправы. Боковое перемещение призмы ограничено установочными планками. Выставленное положение призмы фиксируется пластинчатой пружиной, которая осуществляет давление на перемещающийся во втулке стержень. Для равномерного распределения усилия прижима между стержнем и гранью призмы помещены металлическая и эластичная прокладки.

\begin{figure}[h!]
	\begin{center}
		\includegraphics[width=0.6\textwidth]{6planspring.png}
		\caption{ Крепление призмы, плоской пружиной }
		\label{pic:6planspring}
	\end{center}
\end{figure} 

\newthought{Крепление приклеиванием} \marginnote{\allcaps{КРЕПЛЕНИЕ\break ПРИКЛЕИВАНИЕМ}} призм, особенно имеющих небольшую массу, экономически существенно выгоднее других способов крепления из-за сложности формы призм, отличающая их от круглых оптических деталей.

Рассмотрим пример крепления прямоугольной призмы, у которой для использования эффекта полного внутреннего отражения, отражающая гипотенузная грань должна быть полностью свободна. 
Это требование наиболее просто можно обеспечить приклеиванием призмы к оправе по ее нерабочей грани (рис.~\ref{pic:6glue}). 
Здесь призма приклеена к боковой поверхности оправы в виде угольника, который может юстироваться с помощью винтов. 
Приклеивание осуществляется либо по всей поверхности сопряжения, либо по его периметру, либо по его части, например по цилиндрическому пятну 1 выборки в оправе, куда заливается герметик или другое клеящее вещество.

\begin{figure}[h!]
	\begin{center}
		\includegraphics[width=0.8\textwidth]{6glue.png}
		\caption{ Призма, приклеенная к оправе одной гранью }
		\label{pic:6glue}
	\end{center}
\end{figure} 

Недостатком такого крепления является ограничение массы приклеиваемой призмы.
Клеящий слой не позволяет обеспечить надежное скрепление массивной призмы и оправы, особенно при консольном положении призмы.

Конструкция оправы (рис.~\ref{pic:6gluescrew}) позволяет осуществить надежное крепление массивной прямоугольной призмы приклеиванием. В данной конструкции высокие требования к положению призмы 1 обеспечиваются подготовкой базирующих поверхностей оправы 4. Призма установлена на краевые зоны преломляющих граней и прижимается к базовым плоскостям оправы упорным винтом 4 через пластину 2. Зазоры между оправой и призмой заполняются клеящим веществом 5. После затвердевания клея винт 4 можно удалить, а образовавшуюся пустоту заполнить клеем. Для исключения потери эффекта полного внутреннего отражения пластина 2 должна иметь выборку, соответствующую световому размеру пучка лучей на гипотенузной грани, в противном случае на гипотенузную грань призмы следует нанести зеркальное отражающее покрытие с соответствующей защитой.
\begin{marginfigure}
	\begin{center}
		\includegraphics[width=0.7\linewidth]{6gluescrew.png}
		\caption{ Крепление призмы приклеиванием со вспомогательным винтом }
		\label{pic:6gluescrew}
	\end{center}
\end{marginfigure}

Клеящие материалы, которые рекомендуется применять для крепления призм к оправам -- это различные герметики, компаунды, эпоксидные смолы.
Оправы для крепления призм приклеиванием, как правило, изготавливают из алюминиевых сплавов с последующим чернением (анодным оксидированием).
Специальные способы крепления. 
К таким способам относят крепление с использованием специальных крепящих деталей, либо комбинации крепящих деталей различных видов, а также крепление зажимом призм с деформированием элементов оправы, крепление формовкой призм в пластмассовые оправы, постановкой их на оптический контакт.

\newthought{Призмы,\marginnote{\allcaps{КОНСТРУКЦИИ УЗЛОВ \break КРЕПЛЕНИЯ \break ПРИЗМЕННЫХ \break СИСТЕМ}} представляющие собой систему из двух или большего числа простых типовых призм}, соединенных в единый блок с помощью склейки или закрепления в оправе, называются составными, или сложными. В склеенном блоке одна из призм, как правило, самая большая (массивная), является несущей, к ней приклеиваются остальные призмы. В узле крепления такая призма выполняет функцию базовой детали, определяющей положение других призм из склеенного блока. Крепление простых склеенных призменных блоков может быть осуществлено одним из рассмотренных выше способов крепления одиночных призм.

На рис.~\ref{pic:6shoes} приведена конструкция узла крепления башмачной призмы в оправе при помощи пружины, представляющей собой разрезанную упругую цилиндрическую трубку, которая вставляется с натягом между корпусом и прокладной пластиной, передающей силовое замыкание на призму. Базирование осуществляется по двум граням несущей призмы. Для компенсации погрешностей изготовления сопрягаемых поверхностей оправы (выполненной точным литьем) и призмы между одной из ее рабочих поверхностей и поверхностью -- оправы установлена упругая прокладка. Клин башмачной призмы установлен с необходимым воздушным промежутком (с помощью станиолевых прокладок или нанесением в вакууме дистанционных алюминиевых полосок) и приклеен к несущей призме с помощью боковых стеклянных пластин.

\begin{figure}[h!]
	\includegraphics[width=1\textwidth]{6shoes.png}
	\caption{ Узел крепления склеенной башмачной призмы }
	\label{pic:6shoes}
\end{figure}

\section{Зеркала}

\newthought{Зеркало} \marginnote{\allcaps{ЗЕРКАЛО}} --- оптическая деталь, у которой рабочая поверхность (или одна из рабочих поверхностей) отражает оптическое излучение. Отражение происходит от зеркального покрытия, нанесенного на рабочую поверхность, либо от самой полированной рабочей поверхности детали, выполненной из материала, обладающего хорошей отражательной способностью (например, медные или алюминиевые сплавы, коррозионно-стойкая сталь).

Зеркальное покрытие может быть нанесено на внешнюю рабочую поверхность (наружное покрытие) или заднюю рабочую поверхность (внутреннее покрытие).
Зеркала подразделяют на плоские (с плоской рабочей поверхностью) и силовые (со сферической или асферической рабочей поверхностью).

Плоские зеркала по своему действию на световой пучок подобны отражательным призмам, а силовые~-- линзам, так как они преобразуют волновой фронт и создают изображение.

Достоинствами зеркал по сравнению с призмами и линзами являются: меньшая масса, простота конструкций, меньшее значение вносимых аберраций (в том числе отсутствие хроматизма у зеркал с наружным отражением), исключение требований к ряду показателей качества материала зеркал с наружным отражением, а также возможность создания зеркал больших размеров (до нескольких и более метров в диаметре). 

\newthought{Конструктивные параметры} зеркал подразделяют на \textsc{расчетные} и \textsc{конструкторские}.

К \textsc{расчетным конструктивным параметрам}\marginnote{\allcaps{РАСЧЁТНЫЕ\break ПАРАМЕТРЫ}} зеркала, получаемым в результате габаритного, аберрационного и светотехнического расчетов оптической системы, относят: световые размеры рабочей поверхности(ей) и параметры ее формы (плоскость, радиус сферы, уравнение и координаты точек для асферики), значение коэффициента отражения от рабочей (зеркальной) поверхности [при работе в УФ-области спектра (1~-- 380 нм), видимой (380~- 780 нм), инфракрасной (780~нм~-- 1 мм)]; допустимые значения погрешностей изготовления рабочей поверхности (общая и местная погрешности формы, децентрировка).

К \textsc{конструкторским параметрам}\marginnote{\allcaps{КОНСТРУКТИВНЫЕ\break ПАРАМЕТРЫ}}, получаемым в процессе разработки конструкции зеркала, относят: материал зеркала, его габаритные размеры (зависящие от световых размеров, способа крепления, необходимой технологической и конструктивной жесткости, запаса для юстировки), шероховатость и класс чистоты поверхностей, параметры фасок, вид покрытий.

Рассмотрим некоторые особенности определения конструктивных параметров зеркал:
\begin{enumerate}[leftmargin=*]
\item Выбор материала зеркала зависит от его назначения (для построения изображения, осветительное), условий эксплуатации (температурный режим, нагрузки), требований к массе и габаритным размерам, возможности реализации необходимой технологии изготовления. 

Чаще всего зеркала изготавливают из традиционных материалов: оптического стекла (ЛК5, ЛК7, К8), плавленого кварца (КУ, KB, UVFS), ситаллов (СО115М, СО-ЗЗМ, церодур).

Развитие космической аппаратуры, создание мощных лазерных и адаптивных систем, криогенных телескопов (требующих охлаждения зеркал до температур жидкого гелия 4К и жидкого азота 77К) привели к изготовлению зеркал из нетрадиционных материалов (меди, алюминиевых сплавов, титана, бериллия, кремния, карбида кремния, боросиликата, графитоэпоксида.

Основными единичными показателями качества материала, используемого для зеркал, являются: его плотность $ \rho $ (чем она меньше, тем лучше); модуль упругости $ Е $ (чем он больше, тем лучше); температурный коэффициент линейного расширения $ \alpha $ (чем он меньше, тем лучше); теплопроводность $ \lambda $ (чем она больше, тем лучше); удельная теплоемкость $ С $ (чем она меньше, тем лучше).

Комплексными показателями качества материала являются: его удельная жесткость $ E/\rho $ , пропорциональная деформация под действием собственного веса, позволяющая оценить стабильность формы рабочей поверхности зеркала при изготовлении, закреплении и эксплуатации под действием нагрузок; температурная стабильность $ \lambda/\alpha $, характеризующая термодеформации зеркала при изменении температуры; коэффициент Максутова $ \psi = E\,\lambda/(\alpha\,\rho\,C) $, которым пользуются для ориентировочной интегральной оценки качества материала для зеркал.

Заметим, что конструктор обычно старается использовать для изготовления зеркал материалы с низким значением коэффициента $ \alpha $, однако низкая теплопроводность (температуропроводность $ q $) материала зеркала не позволяет выровнять температуру в его объеме при изменении теплового потока, что вызывает неравномерность напряжений и температурные деформации рабочей поверхности (эффект края). На это обстоятельство указывал Д.~Д.~Максутов, приводивший ряд преимуществ металлических зеркал перед стеклянными.

Металлы и другие теплопроводные материалы позволяют реализовывать альтернативный подход к решению проблемы температурной стабильности зеркал за счет их высокой теплопроводности.

Решающим аргументом в пользу ряда нетрадиционных материалов является принципиально более высокая удельная жесткость последних. Бериллий, карбид кремния, кремний превосходят традиционные материалы по этому показателю в 2-5 раз.

Особенно следует отметить карбид кремния, который сочетает удельную жесткость бериллия с температурной стабильностью лучших сверхмалорасширяющихся материалов, что позволяет создавать из этого материала зеркала с качественно новыми служебными свойствами.

Ряд нетрадиционных материалов не позволяет получать непосредственно на них рабочую поверхность оптического качества из-за пористости ($ SiC $), инородных включений ($ А1 $), токсичности при обработке ($ Be $), отсутствия технологии достижения требуемого качества поверхности ($ Ti $). В этих случаях на рабочую поверхность таких материалов наносят специальные конструкционные покрытия (стеклянные, медные, никелевые, хромовые), которые затем доводятся и полируются до оптического качества. 

\item В таблице на чертеже оптической детали, в разделе <<Требования к материалу>>, для зеркал с наружным отражением указываются категории \textit{бессвильности}, \textit{пузырности} (включения) и \textit{двойного лучепреломления}.

Вскрытые при обработке пузыри и вышедшие на поверхность зеркала свили образуют местные дефекты формы поверхности, которые искажают волновой фронт пучка, отраженного от зеркала. Пузыри (и приравненные к ним включения) влияют также на класс чистоты полированной поверхности. Двойное лучепреломление характеризует остаточные напряжения в материале зеркала, при их отсутствии затруднительно обеспечить требуемые значения $ N, \Delta N $ и возникает увеличение деформаций из-за воздействия собственного веса зеркала при его закреплении. Для зеркал с внутренним отражением указывают также и другие нормируемые для используемого материала оптические показатели качества.

\item Вид зеркального (светоделительного) покрытия выбирается в зависимости от назначения, размеров и условий работы зеркал.
Основными характеристиками всех видов покрытий являются оптические свойства (коэффициент отражения $ \rho $ и коэффициент пропускания $ \tau $), химическая стойкость, механическая и термическая прочность.

Зеркальные покрытия подразделяются на металлические, диэлектрические и металлодиэлектрические.

Простейшими металлическими отражающими покрытиями, широко используемыми для изготовления зеркал, являются металлические пленки серебра, алюминия, хрома, никеля, родия, палладия.

Серебрение дает наибольший коэффициент отражения (до 0,96), но оно наименее химически стойкое из всех покрытий -- от действия атмосферы очень быстро тускнеет и теряет отражающие свойства. В современных оптических приборах серебрение применяется только для зеркал с внутренним отражением, где защита слоя покрытия легко осуществляется нанесением на серебро тонкого слоя меди (электролитическим способом) и еще слоя защитного лака. Серебрение выполняется двумя методами -- химическим (например, из раствора азотнокислого серебра) и испарением в вакууме.

Алюминирование имеет коэффициент отражения до 0,86 и выполняется методом испарения в вакууме. Химическая стойкость алюминирования значительно выше серебрения, и при работе в лабораторных условиях оно не требует защиты; при работе во влажной атмосфере необходима защита, которая осуществляется нанесением на алюминий прозрачного слоя другого вещества (сернистого цинка, одноокиси кремния и др.). Недостатком алюминирования является низкая механическая прочность покрытия, что делает его непригодным в тех случаях, когда по условиям работы зеркало подвергается механическим воздействиям.

Благодаря хорошо освоенной технологии и высокому коэффициенту отражения алюминирование является в настоящее время основным видом покрытия для зеркал с наружным отражением, не подвергающихся механическим воздействиям.

Хромирование -- наиболее стойкое и прочное из металлических покрытий, в большинстве случаев оно не требует защиты; коэффициент отражения возможен до 0,55; применяется при работе зеркал в сложном тепловом режиме (фары, рефлекторы дуговых осветителей и т.п.), а также для зеркал полевых приборов, подвергающихся атмосферным и механическим воздействиям.

Из других зеркальных металлических покрытий часто применяют покрытия родием и палладием. Они обладают высокой химической стойкостью и механической прочностью. Коэффициенты отражения: у родиевого (с подслоем никеля или хрома) до 0,78, а у покрытия палладием до 0,68. Эти покрытия обладают высокой стойкостью при работе в агрессивных средах (морской воде, растворах кислот, щелочей и т.п.), а также при воздействии относительно высоких температур (рефлекторы прожекторов, светопроекторов), характеризуются более высокой стоимостью и сложностью технологии нанесения.

Стойкие зеркальные покрытия с коэффициентом отражения, близким к единице, получают нанесением на подложку многослойных пленочных покрытий из диэлектрических материалов. Детали с такими покрытиями получили название интерференционных зеркал.

Для изготовления интерференционных зеркал используют покрытия из нечетно чередующихся слоев диэлектриков с большими и малыми показателями преломления и оптической толщиной 0,5$ \lambda $, т.е. создают покрытия, имеющие <<антипросветляющие>> свойства. В отличие от просветляющих покрытий наружный слой интерференционного покрытия должен иметь показатель преломления больший, чем показатель подложки. С увеличением числа слоев коэффициент отражения увеличивается. Так, 15-17-слойные покрытия из пленок $ SiO_2 $ и $ TiO_2 $, $ MgF $ или $ ZnS $ имеют коэффициент отражения не менее 99\% в широкой области спектра.

Покрытия зеркал холодного света отражают свет видимой части спектра и практически полностью прозрачны для ИК-лучей, что весьма важно при применении таких зеркал в осветительных системах кинопроекционной аппаратуры.

Прочные зеркальные покрытия с коэффициентом отражения $ \rho_\lambda $ = 92 $ \div $ 99 \% получают нанесением на металлические пленки одного или нескольких слоев диэлектриков (например, тугоплавких веществ $ SiO_2 $, $ TiO_2 $ или $ ZrO_2 $). Такие покрытия называют металлодиэлектрическими, они имеют высокий коэффициент отражения в широкой полосе спектра с меньшим числом слоев пленки, чем у диэлектрических многослойных покрытий.

Светоделителъные покрытия (полупрозрачные зеркала) делят световой поток на отраженный и проходящий и характеризуются отношением коэффициента отражения $ \rho_\lambda $ к коэффициенту пропускания $ \tau_\lambda $. Это отношение может быть получено в широком диапазоне нанесением на подложку металлических пленок разной толщины или пленок из диэлектриков. Например, светоделительные покрытия с помощью алюминирования или серебрения можно получить практически с любым соотношением между коэффициентом отражения и коэффициентом пропускания $ \rho/\tau $. 

\item Толщина зеркала зависит от его световых размеров (габаритных размеров), способа крепления и главным образом от требуемой точности рабочей поверхности.

Чем точнее должна быть форма рабочей поверхности зеркала, тем оно должно быть толще. Толстые зеркала меньще деформируются при изготовлении, закреплении и эксплуатации. Например, прогиб $ f $ круглого зеркала в виде сплошного диска, установленного горизонтально и опирающегося по периметру на три точки пропорционален четвертой степени его диаметра $ D $ и обратно пропорционален второй степени его толщины $ d $:
\[ f_{max} = 1,365*10^6\gamma (1-\mu)D^4/4\,E\,d^2 = k\,D^4/d^2, \]

где $ \gamma $ -- удельный вес материала зеркала, $ \mu $ -- коэффициент Пуассона, $ E $ -- модуль упругости, $ k  $ -- коэффициент, учитывающий механические свойства материала.

Рекомендуется применять следующие соотношения между толщиной $ d $ и наибольшим размером $ l $ (для круглого -- диаметром) зеркала, выполненного из традиционного материала:

\begin{enumerate}
\item особо точное зеркало ($ N $=0,05$ \div $0,5;  $ \Delta N $=0,02$ \div $0,1, зеркала интерферометров, концевые отражатели дальномеров, резонаторы лазеров, зеркала телескопов): $ d \ge (1/5 \div 1/7)l $;
\item точное зеркало ($ N=1\div2 $;  $ \Delta N=0,1\div0,2 $, рабочие зеркала наблюдательных, визирных, измерительных приборов): $ d \ge (1/8 \div 1/10)l $.
\item неответственное зеркало ($ N=3 \div 10 $,  $ N=0,3 \div 1 $, зеркала осветительных систем, и систем, не требующих высокого качества изображения): $ d \ge (1/11 \div 1/25)l $.
\end{enumerate}

Уменьшить толщину зеркала можно, применив при его конструировании следующие приемы:

\begin{itemize}
\item создание облегченной конструкции (сотовая структура, выполнение выборок в теле зеркала (рис.~\ref{pic:6lightmirror}~а), толщина переменного сечения (рис.~\ref{pic:6lightmirror}~б), коробчатая форма (рис.~\ref{pic:6lightmirror}~в), где к зеркалу 2 с рабочей поверхностью 1 напекается пластина 3;

\begin{figure}[h!]
	\caption{ Облегченные зеркала }
	\includegraphics[width=1\textwidth]{6lightmirror.png}
	\label{pic:6lightmirror}
\end{figure}

\item разгрузка зеркала при изготовлении и креплении;
\item разработка металлостеклянной конструкции зеркала (при этом в металлической подложке выполняют выборки, уменьшающие массу конструкции зеркала).

Металлостеклянное зеркало создают путем напекания тонкой стеклянной заготовки (пластинки) толщиной в несколько миллиметров и более (при значительных размерах зеркала) на основу зеркала, выполненную из металла, сплавов, кристаллических и других материалов. Чаще всего основу таких зеркал изготавливают из металлических сплавов (титановых, коррозионно-стойкой стали, сплава ковар, алюминиевых сплавов, бериллия), для которых имеются марки стекол с близкими значениями коэффициента линейного расширения (желательно, чтобы $ \alpha $ стекла было бы меньше  $ \alpha $ материала, а их разница была бы не более $1*10^{-7} $).

После спекания стекло обрабатывается до толщины 0,2-0,3~мм и полируется до достижения рабочей поверхности требуемой точности.

Металлостеклянное зеркало обладает рядом высоких конструкционных качеств:

\begin{itemize}
\item благодаря металлической основе его можно выполнить меньшим по толщине при достаточной жесткости, а также прочности под воздействием динамических нагрузок;
\item основа зеркала может выполнять роль оправы, что снижает общую массу узла и упрощает его сборку и юстировку;
\item базовые поверхности зеркала (шейки валов под подшипники, посадочные диаметры и торцы) после полирования рабочей поверхности (как правило, до нанесения зеркального покрытия) могут быть обработаны окончательно в размер от рабочей поверхности, что может исключить необходимость его юстировки.
\end{itemize}

Заметим, что для исключения возможных деформаций стеклометаллического зеркала из-за внутренних напряжений необходимо осуществлять отжиг основы зеркала после ее механической обработки и термоциклическую обработку узла после напекания и грубого шлифования стекла.

\end{itemize}

\item Как правило, отражающий слой зеркала, используемого для построения изображения, наносят на его наружной стороне, чтобы избежать влияния отклонений характеристик материала и погрешностей изготовления преломляющей рабочей поверхности зеркала (например, погрешности формы, клиновидности) на качество изображения.

Зеркало с задней отражающей поверхностью не рекомендуется устанавливать в сходящихся пучках лучей, так как возможно возникновение двоения изображения, а при наклонном положении еще и хроматизма, астигматизма, асимметрии и других аберраций.

\item Конструктивные формы и размеры зеркал зависят от их назначения, положения в оптической системе, световых размеров (диаметра), способа их закрепления.

Наибольшее разнообразие форм имеют плоские зеркала, они бывают круглые и квадратные (если расположены нормально или под небольшим углом к пучку лучей), прямоугольные, эллиптические, многоугольные (если расположены под углом к пучку лучей).

Сферические и асферические зеркала (параболические, гиперболические, эллиптические), осевые и вне осевые обычно имеют круглую форму. Часто такие зеркала имеют внутреннее отверстие для прохождения пучка лучей, базирования зеркала или закрепления в нем других элементов (например, бленды).

Зеркала могут быть изготовлены также в виде бипризм, пирамид, конусов и полигонов (рис.~\ref{pic:6planmirror}), которые используются для разделения пучка лучей, сканирования изображения, модуляции светового потока, как эталоны углов.

Особую конструкцию имеют составные и гибкие (адаптивные) зеркала, формой рабочих поверхностей которых управляют для компенсации влияния рефракций и турбулентности атмосферы, погрешностей оптической системы и ее юстировки (рис.~\ref{pic:6adaptmirror}).

\begin{figure}[h!]
%	\caption{ Плоские зеркала: а -- одиночное зеркало, б -- зеркальный ромб, \\в -- двухсторонняя пирамида, г -- угловое зеркало, д -- четырехсторонняя пирамида, е -- светоделительное зеркало, ж -- зеркальный полигон }
	\includegraphics[width=1\textwidth]{6planmirror.png}
	\label{pic:6planmirror}
\end{figure}

\begin{figure}[h!]
%	\caption{ Адаптивное зеркало:\\ 1 -- отдельное элементарное зеркало, 2 -- цилиндр из пьезокерамики, 3 -- основание,\\ 4 -- юстировочный винт; 5 -- электрод }
	\includegraphics[width=1\textwidth]{6adaptmirror.png}
	\label{pic:6adaptmirror}
\end{figure}

\item Допуски на точность изготовления рабочих и базовых поверхностей зеркала (погрешности формы и чистоты рабочих поверхностей, погрешность посадочного диаметра (или размера) базовой поверхности, децентрировка, клиновидность рабочих поверхностей плоских зеркал с внутренним отражением) определяются  функциональным  назначением зеркала, характеристиками и требованиями к качеству оптической системы.

Например, зеркала современных телескопов и лазерных систем должны обеспечивать качество изображения и расходимость излучения на уровне предела, ограничиваемого дифракцией. Это означает, что среднеквадратичное отклонение формы оптической поверхности зеркал от заданной не должно превышать сотых долей рабочей длины волны ($ \lambda/50 \div \lambda/70 $), в линейной мере оно составляет менее 0,01 мкм. С учетом размеров зеркал (0,5~м и более) это позволяет характеризовать их как наиболее точные изделия современного приборостроения.

Допуск на погрешность формы обычно задают в виде, зависящем от метода контроля. Например, при контроле сферической поверхности с помощью интерферометра он задается в долях длины волны $ \lambda $, при контроле сферометром -- в процентах отклонения от номинального радиуса $ \Delta R $ \%, при контроле пробным стеклом -- количеством колец $ N, \Delta N $.

Заметим, что допуски на общую погрешность формы зеркал, установленных наклонно к световому пучку, более жесткие, чем для установленных перпендикулярно, а для местных погрешностей и чистоты рабочей поверхности -- наоборот, более широкие.

Для плоских зеркал с внутренним отражением их клиновидность вызывает хроматизм, а в случае светоделительных -- еще и двоение изображения. Допуск на клиновидность таких зеркал наиболее жесткий (до $ 4-6'' $).

Для сферических зеркал на чертежах проставляется допуск на их центрировку.

\item На кромках зеркал наносят фаски, их нерабочие (матовые) поверхности могут быть окрашены эмалью, а рабочие поверхности покрыты оптическими защитными или электропроводящими покрытиями.
\end{enumerate}


\section{Узлы крепления зеркал и зеркальных систем}

Особенностью\marginnote{Требования к узлам крепления зеркал} оптических зеркал, которую необходимо учитывать при разработке конструкции крепления, является их повышенная чувствительность к деформациям -- изгибу зеркала и местным искажениям формы отражающей поверхности. 

Поэтому, применяя различные способы крепления (при помощи планок, скоб, угольников, резьбовых колец, пружин и других прижимных элементов, а также клеев и замазок), необходимо соблюдать следующие условия:
\begin{enumerate}
\item Конструкция крепления зеркала должна обеспечивать статически и геометрически определенное соединение --- базирование на три точки (площадки).
\item При разработке конструкции необходимо соблюдать принцип силового замыкания соединений: сила, прижимающая зеркало, должна проходить через опорные площадки оправы. В узлах крепления больших (массивных) зеркал, кроме того, необходимо применять принцип равномерного распределения его массы путем введения в конструкцию дополнительных опор (механическая разгрузка), а также гидравлических или пневматических разгрузок.
\item Между прижимающей деталью и зеркалом рекомендуется ставить упругие прокладки, чтобы не вызывать локальных (в местах контакта) напряжений.
\item Следует обеспечивать необходимую жесткость конструкции, используя зеркала и оправы «сотовой» структуры, а также зеркала, не нуждающиеся в оправах.
\item Следует предусматривать необходимую юстировку зеркала относительно оправы, либо оправы зеркала относительно корпусных деталей и баз устройства (системы).
\end{enumerate}

К мерам, позволяющим снизить воздействия на узел крепления зеркала колебаний температуры, относятся следующие:
\begin{itemize}
\item обеспечение необходимого температурного зазора в посадке зеркала в оправу;
\item подбор материалов зеркала и оправы с близкими значениями коэффициентов линейного расширения (например, стекло «крон» и металл титан; зеркало из кварца или ситалла, а оправа из сплава инвар);
\item применение промежуточных (между зеркалом и оправой) компенсационных элементов, термокомпенсаторов;
\item консольное крепление зеркала;
\item изготовление зеркал из металла или в виде металлостеклянного зеркала, не нуждающихся в оправах и обладающих хорошей температурной стабильностью.
\end{itemize}

Круглые зеркала могут быть закреплены в оправах теми же способами, что и рассмотренные выше способы крепления круглых линз. 

\newthought{Крепление прижимными планками} \marginnote{\allcaps{КРЕПЛЕНИЕ\break ПРИЖИМНЫМИ\break ПЛАНКАМИ}} (лапок, угольников, пластин) применяется для точных зеркал различных форм и размеров. 
Зеркало устанавливается на три выступающие площадки плоской оправы (рис.~\ref{pic:6plankamirror}~а). 
Площадки могут быть заменены прокладками из алюминиевой фольги, если размер зеркала не превышает 50~мм. 
Прижим зеркала осуществляется Z-образными планками, которые винтами крепятся к оправе в местах расположения выступающих площадок. 
Для компенсации погрешностей сопряжения <<зеркало~-- планка>> между ними помещается эластичная прокладка (картон, пробка, паронит).

Если силы трения не обеспечивают неподвижность зеркала, планки выполняют дополнительную функцию --- ограничивают перемещение зеркала вдоль оправы путем создания контакта планок по краю зеркала. 
Для этого на планках выполняют выступ~В.

\begin{figure}[h!]
	\includegraphics[width=1\textwidth]{6plankamirror.png}
	\caption[Крепление зеркала Z-образными и Г-образными планками]{ Крепление зеркала: \breakа -- Z-образными планками, \breakб -- Г-образными планками }
	\label{pic:6plankamirror}
\end{figure}

Придание прижимным планкам Г-образной формы (рис.~\ref{pic:6plankamirror}~б) позволяет регулировать усилие прижима зеркала за счет смещения планок в пределах зазора в отверстиях под крепежные винты.

Недостатками конструкции являются: ограниченность в компенсации воздействия колебаний температуры, увеличение габаритных размеров и снижение технологичности узла крепления из-за применения нескольких крепежных элементов.

К положительным свойствам можно отнести простоту сборки узла, возможность крепить зеркала любой конфигурации. Конструкция удовлетворяет основным требованиям к узлам крепления зеркал.

\newthought{Крепление пружинами} \marginnote{\allcaps{КРЕПЛЕНИЕ\break ПРУЖИНАМИ}} основано на создании замыкающего усилия для прижима зеркала к оправе при помощи пружины (проволочной, мембранной, пластинчатой). 
На рис.\ref{pic:6springmirror} приведена конструкция крепления круглого зеркала проволочной пружиной. 
Здесь зеркало устанавливается на кольцевой уступ оправы по посадке с гарантированным зазором. 
Прижим зеркала осуществляется диском с выборкой в центре, на который воздействует винтовая пружина сжатия (в конструкции могут быть применены и другие типы пружин). 
Усилие прижима регулируется винтом, положение которого фиксируется гайкой. 
Для равномерного распределения усилия по окружности соединение диска со штоком может быть шарнирное.

\begin{figure}[h!]
	\begin{center}
		\includegraphics[width=0.4\textwidth]{6springmirror.png}
		\caption{ Крепление зеркала пружиной }
		\label{pic:6springmirror}
	\end{center}
\end{figure}

При креплении рассмотренным способом зеркал средних и больших диаметров, чтобы исключить возможные изгибающие моменты, зеркало должно базироваться на три площадки. Поэтому между зеркалом и уступом оправы через 120$ ^\circ $ помещают металлические прокладки либо выполняют в уступе выборки, создающие на нем три опорные площадки. В прижимающей зеркало пластине должны быть выполнены выборки, чтобы образованные выступы были сориентированы напротив установленных прокладок.

Достоинством крепления пружиной является обеспечение стабильности формы и положения зеркала при механических воздействиях и воздействии колебания температуры, так как возникающие возмущения компенсируются за счет деформации пружины.

К рассмотренным механическим способам крепления, для круглых зеркал можно добавить крепление при помощи резьбового кольца или проволочным кольцом, аналогично креплению линз.

\newthought{Крепления приклеиванием} \marginnote{\allcaps{КРЕПЛЕНИЕ\break ПРИКЛЕИВАНИЕМ}} зеркал отличаются в зависимости от размеров, формы и назначения зеркал в оптической системе.

Для зеркал неответственных систем (осветительных c $ N=5,\, \Delta N=0,5 $) возможно крепление по плоскости с опорой на равномерный сплошной слой клеящего вещества, например, герметика марки УТ-34. На рис.~\ref{pic:6sphermirror} показано крепление сферического зеркала 3 диаметром 48~мм из стекла К8 в оправе~2 из алюминиевого сплава. Зеркало помещено на слой герметика 1 толщиной 0,5~мм, нанесенного на плоскость оправы. Эта конструкция, из-за разделения оправы и зеркала клеящим слоем, не обеспечивает высокой точности положения зеркала относительно базовых поверхностей оправы. 

\begin{figure}[h!]
	\begin{center}
		\includegraphics[width=0.6\textwidth]{6sphermirror.png}
		\caption{ Крепление сферического зеркала на слой клеящего вещества }
		\label{pic:6sphermirror}
	\end{center}
\end{figure}

В конструкции узла, показанного на рис.~\ref{pic:6poyasok}, устранен недостаток неопределенного базирования зеркала. Для этого зеркало~1 установлено на специально выполненные в оправе~3 опорные пояски~4 шириной 0,5~мм. Клеящее вещество~2 залито в промежутки между поясками. В оправе выполнены отверстия для выдавливания излишков клеящего вещества. Эта конструкция позволяет разделить функции: элементы оправы обеспечивают базирование (ориентацию) зеркала, клеящее вещество обеспечивает соединение (закрепление) зеркала с оправой.

\begin{figure}
	\begin{center}
		\includegraphics[width=0.4\textwidth]{6poyasok.png}
		\caption[Крепление зеркала с установкой на опорные пояски]{ Крепление зеркала с\breakустановкой на опорные пояски }
		\label{pic:6poyasok}
	\end{center}
\end{figure}

Для крепления сферических зеркал (рис.~\ref{pic:6cylindric}), имеющих в центре отверстие, можно использовать его внутреннюю цилиндрическую поверхность.
\begin{figure}[h!]
	\begin{center}
		\includegraphics[width=0.3\textwidth]{6cylindric.png}
		\caption[Крепление зеркала за цилиндрическое отверстие]{ Крепление зеркала за\breakцилиндрическое отверстие }
		\label{pic:6cylindric}
	\end{center}
\end{figure}

Оправа~2 представляет собой полую ось с фланцем. Она изготовлена из алюминиевого сплава~Д16Т, имеет черное покрытие~-- анодное оксидирование. Особенностью оправы является наличие центрирующих поясков, на которые устанавливается зеркало~3. 
Поскольку именно эти элементы определяют относительное положение зеркала, в рабочем чертеже оправы следует установить допуск отклонения от перпендикулярности между осевыми и радиально расположенными поясками. Клеящее вещество~1 заливают в промежутки между поясками по цилиндрической и плоской части оправы. Глубина промежутков, как правило, не превышает~0,5~мм.

\newthought{Специальные способы крепления}\marginnote{\allcaps{СПЕЦИАЛЬНЫЕ\break СПОСОБЫ\break КРЕПЛЕНИЯ}}:крепления крупногабаритных зеркал, металлостеклянных, консольные виды крепления, крепления свариванием, спеканием или постановкой на глубокий оптический контакт деталей (зеркал и оснований). 

Крепление крупногабаритных зеркал (более 200~мм) отличается от обычных, рассмотренных выше, способов крепления. 
Базирование таких зеркал только на три опоры хотя и является статически определенным, но приводит к недопустимым прогибам отражающей поверхности из-за статической деформации зеркала. 
Причем значение деформации прямо пропорционально четвертой степени диаметра зеркала. 
Поэтому в конструкциях крепления таких зеркал количество опор в направлении силы тяжести увеличивается. 
При этом важно сохранить принцип трехточечного базирования и учитывать изменяющееся положение зеркала в процессе работы. Существуют различные системы осевой и радиальной разгрузки при креплении зеркал в оправах (Гребба, Ласселя, пневматическая, гидравлическая).

Выполнение указанных требований рассмотрим на примере крепления главного зеркала объектива телескопа (рис.~\ref{pic:6razgruzka}). Зеркало установлено на восемнадцать разгрузочных опор~1. 
Соединение разгрузочных опор с зеркалом выполнено герметиком. 
Каждые три опоры объединены треугольной платформой~2. 
В центре тяжести платформы установлен шаровой шарнир. 
По две платформы через эти шарниры соединены с плечом рычага. 
Рычаги закреплены на общей оправе также через шаровые шарниры~3. 
Такая конструкция обеспечивает равномерное распределение массы зеркала относительно всей поверхности. 
Соединение опор с зеркалом эластичным материалом позволяет существенно упростить конструкцию разгружающей опоры и компенсировать погрешности сопрягаемых элементов, что не вызывает деформацию зеркала. 
Вместе с этим конструкция требует тщательной сборки и настройки, прежде всего, выравнивания усилий, создаваемых разгрузочными устройствами.

\begin{figure}[h!]
	\includegraphics[width=0.85\textwidth]{6razgruzka.png}
	\caption[Разгрузка массы зеркала на 18 опор]{ Разгрузка массы зеркала на 18 опор: 1 -- опоры, 2 -- разгрузочная площадка, 3 -- сферический шарнир }
	\label{pic:6razgruzka}
\end{figure}

На рис.~\ref{pic:6razgruzka1}~ показаны примеры конструкций для разгрузки массы зеркала. Изображения зеркал с сотовой структурой показаны на рис.~\ref{pic:6channelMirror}.

\begin{figure*}[h!]
	\includegraphics[width=1\textwidth]{6razgruzka1.png}
	\caption{ Пример конструкций для разгрузки массы зеркала }
	\label{pic:6razgruzka1}
\end{figure*}

\begin{figure*}[h!]
	\includegraphics[width=1\textwidth]{6channelMirror.png}
	\caption{ Примеры зеркал с сотовой структурой }
	\label{pic:6channelMirror}
\end{figure*}


\newthought{К узлам крепления зеркальных систем} \marginnote{\allcaps{КРЕПЛЕНИЕ ЗЕРКАЛЬНЫХ СИСТЕМ}} предъявляются те же требования, что и к узлам крепления одиночных зеркал. 
Кроме того, узел крепления зеркальной системы должен обеспечивать требуемое взаимное расположение рабочих (отражающих) поверхностей отдельных зеркал. 
В наиболее ответственных случаях предусматриваются механизмы регулировки угла между зеркалами.

На рис.~\ref{pic:6doublemirror} приведена конструкция узла крепления двухзеркальной системы с углом отклонения лучей 180$ ^\circ $. 
Зеркала~5 установлены на опорные поверхности корпуса~3, выполненные каждая в виде трех выступающих площадок, и прижимаются к ним при помощи винтов~8 через упругие пластины~6. 
Каждый из упоров~9 пластины располагается напротив опорной площадки. 
В основании корпуса выполнены отверстия для входа и выхода светового пучка. 
Винты~8 завинчены в крышки~7 с гайками~1 и фиксируются контргайками~2. 
Крышки привинчены к корпусу винтами~4.

\begin{figure}[h!]
	\begin{center}
		\includegraphics[width=0.7\textwidth]{6doublemirror.png}
		\caption{ Двухзеркальная система с углом отклонения лучей 180$ ^\circ $ }
		\label{pic:6doublemirror}
	\end{center}
\end{figure}



\chapter{Оптические компоненты и их конструктивные узлы}

\section{Аберрации оптических систем}


\section{Дифракционные решётки}
\newthought{Дифракционная решётка} \marginnote{\allcaps{ДИФРАКЦИОННАЯ\break РЕШЁТКА}} --- оптический прибор, предназначенный для анализа спектрального состава оптического излучения.


\section{Источники оптического излучения}
Источники некогерентного излучения:
\begin{itemize}
	\item тепловые;
	\item люминесцентные;
	\item газоразрядные;
	\item светодиоды;
	\item естественные объекты.
\end{itemize}

При тепловом излучении поток излучения и его спектральный состав определяет температура. 
Тепловое излучение происходит в широком спектральном диапазоне и выходит из излучателя во все стороны.

Люминесцентное излучение возбуждается электромагнитным полем, в результате чего атомы и электроны спонтанно переходят с высоких уровней на более низкие.
Люминесцентное излучение выходит из излучателя во все стороны, но спектральный диапазон его уже, чем у теплового.

Газоразрядным источником излучения называют прибор, в котором излучение оптического диапазона спектра возникает в результате электрического разряда в атмосфере инертных газов, паров металла или их смесей. Газоразрядные лампы в большинстве случаев применяются для искусственного освещения.

Импульсная лампа --- газоразрядный прибор с двумя основными токоведущими электродами (катодом и анодом) и газовым промежутком между ними, рассчитанным на возникновение там в необходимые моменты времени мощных импульсных (искровых) электрических разрядов с интенсивным световым излучением. 
В импульсной лампе присутствует также третий управляющий электрод. 
Для возникновения светового импульса лампу подключают к конденсатору, при разряде которого через лампу возникает короткая вспышка большой мощности и энергетической светимости.
Излучение импульсным ламп применяется в качестве накачки активной среды лазера.


Принцип действия излучающих полупроводниковых диодов основан на явлении электролюминесценции при протекании тока в структурах с $ p-n $-переходом.

Естественными источниками излучения являются: Солнце, звёзды, собственное тепловое излучение Земли, живых организмов и тел.

\section{Приёмники оптического излучения}

Приёмник оптического излучения --- элемент или устройство, предназначенное для приёма и преобразования энергии оптического излучения в какие-либо другие виды энергии.

Приёмники излучения:
\begin{itemize}
	\item тепловые;
	\item фотоэлектрические на внутреннем и внешнем фотоэффекте;
	\item фотохимические;
	\item другие.
\end{itemize}

Тепловые приёмники излучения (ПИ) основаны на преобразовании оптического излучения сначала в тепловую энергию, а потом в электрическую и отличаются друг от друга физическими принципами работы.

Болометры --- ПИ, основанные на изменении сопротивления чувствительного элемента под действием тепла, возникающего при падении потока оптического излучения.

Термоэлементы --- ПИ, использующие термоэлектрический эффект.

Калориметры --- ПИ, в котором поглощённая часть падающей энергии оптического излучения преобразуется в тепло, а затем часть тепловой энергии пропорциональная входной оптической величине, в чувствительном элементе калориметра преобразуется в сигнал измерительной информации (как правило электрический).

Пироприёмники --- ПИ, основанный на пироэлектрическом эффекте, который заключается в изменении поляризации пироэлектрического кристалла при изменении его температуры.

Фотоэлектрические ПИ на внутреннем фотоэффекте:
\begin{itemize}
	\item фотодиоды;
	\item фоторезисторы;
	\item фототранзисторы;
	\item сканисторы;
	\item фототиристоры;
	\item приборы с зарядовой связью (ПЗС).
\end{itemize}

Фоторезисторы (ФР) --- ПИ, принцип действия которого основан на эффекте фотопроводимости. 
Эффект фотопроводимости заключается в изменении сопротивления под действием оптического потока.

Конструктивно ФР состоит из тонкого слоя фоточувствительного полупроводникового материала с электродами в виде плёнок, которые не подвергаются коррозии, наносимых испарением в вакууме из золота, платины или серебра. 
Фоточувствительный слой ФР из $ CdS $ и $ CdSe $ наносят пульверизацией на стеклянную или керамическую подложку, реже испарением в вакууме и спеканием порошкообразной массы. ФР на основе $ PbS $ и $ PbSe $ изготавливают химическим осаждением фоторезистивного слоя на подложку из стекла и кварца.
Для защиты резистивного слоя от действия атмосферы его покрывают лаком или заделывают в герметичный корпус.

ФР используют в тепловизорах, радиометрах, теплопеленгаторах, в спектральных приборах, в системах световой сигнализации и защиты, в системах контроля и измерения геометрических размеров, скоростей движения объектов, температуры, управления механизмами, для определения качественного и количественного состава твердых, жидких и газообразных сред.

Фотоэлектрические ПИ на внешнем фотоэффекте:
\begin{itemize}
	\item вакуумные и ионные (газонаполненные) фотоэлементы (ФЭ) или вакуумные диоды;
	\item фотоэлектронные умножители (ФЭУ);
	\item электронно-оптические преобразователи (ЭОП)
\end{itemize}

К фотохимическим ПИ относят различные фоточувствительные фотографические материалы.

\section{Оптическое волокно}


\section{Модуляторы}

Модуляция --- изменение сигнала-носителя энергии в соответствии с передаваемой информацией.
Для ОЭП в большинстве случаев модуляция заключается в изменении одного из параметров (чаще всего амплитуды) потока излучения по заданному закону.

\include{Detail}
\include{Adjustment}
\end{document}
