\chapter{Упругие чувствительные элементы}

\newthought{Упругими элементами} называют детали, упругие деформации которых полезно используются в работе различных механизмов и устройств приборов, аппаратов, информационных машин. 
Упругие чувствительные элементы~(УЧЭ) подразделяют на два класса~--- стержневые пружины (плоские, спиральные и винтовые) и оболочки.

Классификация УЧЭ по назначению:
\begin{itemize}
	\item измерительные;
	\item натяжные;
	\item заводные;
	\item пружины кинематических устройств;
	\item пружины амортизаторов;
	\item разделители сред;
	\item токоведущие упругие элементы;
	\item пружины фрикционных и храповых муфт.
\end{itemize}

Основные характеристики УЧЭ:
\begin{itemize}
	\item упругая характеристика;
	\item чувствительность пружин;
	\item жёсткость;
	\item жёсткость системы соединения пружин;
	\item индекс винтовой пружины;
	\item материал;
	\item упругий гистерезис и упругое последействие;
	\item погрешности.
\end{itemize}

Основные параметры винтовых пружин:
\begin{itemize}
	\item диаметр проволоки $ d $;
	\item средний диаметр $ d_\text{ср} $;
	\item угол подъема витков $ \alpha $
	\item число рабочих витков $ i_\text{р} $;
	\item высота пружины $ h_0 $;
	\item шаг витка $ t $;
	\item зазор между витками $ s $.
\end{itemize}

Проектирование винтовых пружин, как правило, производится в следующей последовательности:
\begin{enumerate}
	\item Выбор проволоки.
	\item Расчёт $ d_\text{ср} $, $ d $ и $ i_\text{р} $.
	\item Расчёт на устойчивость.
	\item Определение нелинейности.
\end{enumerate}