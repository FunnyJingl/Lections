\chapter{Оптические компоненты и их конструктивные узлы}

\section{Аберрации оптических систем}


\section{Дифракционные решётки}
\newthought{Дифракционная решётка} \marginnote{\allcaps{ДИФРАКЦИОННАЯ\break РЕШЁТКА}} --- оптический прибор, предназначенный для анализа спектрального состава оптического излучения.


\section{Источники оптического излучения}
Источники некогерентного излучения:
\begin{itemize}
	\item тепловые;
	\item люминесцентные;
	\item газоразрядные;
	\item светодиоды;
	\item естественные объекты.
\end{itemize}

При тепловом излучении поток излучения и его спектральный состав определяет температура. 
Тепловое излучение происходит в широком спектральном диапазоне и выходит из излучателя во все стороны.

Люминесцентное излучение возбуждается электромагнитным полем, в результате чего атомы и электроны спонтанно переходят с высоких уровней на более низкие.
Люминесцентное излучение выходит из излучателя во все стороны, но спектральный диапазон его уже, чем у теплового.

Газоразрядным источником излучения называют прибор, в котором излучение оптического диапазона спектра возникает в результате электрического разряда в атмосфере инертных газов, паров металла или их смесей. Газоразрядные лампы в большинстве случаев применяются для искусственного освещения.

Импульсная лампа --- газоразрядный прибор с двумя основными токоведущими электродами (катодом и анодом) и газовым промежутком между ними, рассчитанным на возникновение там в необходимые моменты времени мощных импульсных (искровых) электрических разрядов с интенсивным световым излучением. 
В импульсной лампе присутствует также третий управляющий электрод. 
Для возникновения светового импульса лампу подключают к конденсатору, при разряде которого через лампу возникает короткая вспышка большой мощности и энергетической светимости.
Излучение импульсным ламп применяется в качестве накачки активной среды лазера.


Принцип действия излучающих полупроводниковых диодов основан на явлении электролюминесценции при протекании тока в структурах с $ p-n $-переходом.

Естественными источниками излучения являются: Солнце, звёзды, собственное тепловое излучение Земли, живых организмов и тел.

\section{Приёмники оптического излучения}

Приёмник оптического излучения --- элемент или устройство, предназначенное для приёма и преобразования энергии оптического излучения в какие-либо другие виды энергии.

Приёмники излучения:
\begin{itemize}
	\item тепловые;
	\item фотоэлектрические на внутреннем и внешнем фотоэффекте;
	\item фотохимические;
	\item другие.
\end{itemize}

Тепловые приёмники излучения (ПИ) основаны на преобразовании оптического излучения сначала в тепловую энергию, а потом в электрическую и отличаются друг от друга физическими принципами работы.

Болометры --- ПИ, основанные на изменении сопротивления чувствительного элемента под действием тепла, возникающего при падении потока оптического излучения.

Термоэлементы --- ПИ, использующие термоэлектрический эффект.

Калориметры --- ПИ, в котором поглощённая часть падающей энергии оптического излучения преобразуется в тепло, а затем часть тепловой энергии пропорциональная входной оптической величине, в чувствительном элементе калориметра преобразуется в сигнал измерительной информации (как правило электрический).

Пироприёмники --- ПИ, основанный на пироэлектрическом эффекте, который заключается в изменении поляризации пироэлектрического кристалла при изменении его температуры.

Фотоэлектрические ПИ на внутреннем фотоэффекте:
\begin{itemize}
	\item фотодиоды;
	\item фоторезисторы;
	\item фототранзисторы;
	\item сканисторы;
	\item фототиристоры;
	\item приборы с зарядовой связью (ПЗС).
\end{itemize}

Фоторезисторы (ФР) --- ПИ, принцип действия которого основан на эффекте фотопроводимости. 
Эффект фотопроводимости заключается в изменении сопротивления под действием оптического потока.

Конструктивно ФР состоит из тонкого слоя фоточувствительного полупроводникового материала с электродами в виде плёнок, которые не подвергаются коррозии, наносимых испарением в вакууме из золота, платины или серебра. 
Фоточувствительный слой ФР из $ CdS $ и $ CdSe $ наносят пульверизацией на стеклянную или керамическую подложку, реже испарением в вакууме и спеканием порошкообразной массы. ФР на основе $ PbS $ и $ PbSe $ изготавливают химическим осаждением фоторезистивного слоя на подложку из стекла и кварца.
Для защиты резистивного слоя от действия атмосферы его покрывают лаком или заделывают в герметичный корпус.

ФР используют в тепловизорах, радиометрах, теплопеленгаторах, в спектральных приборах, в системах световой сигнализации и защиты, в системах контроля и измерения геометрических размеров, скоростей движения объектов, температуры, управления механизмами, для определения качественного и количественного состава твердых, жидких и газообразных сред.

Фотоэлектрические ПИ на внешнем фотоэффекте:
\begin{itemize}
	\item вакуумные и ионные (газонаполненные) фотоэлементы (ФЭ) или вакуумные диоды;
	\item фотоэлектронные умножители (ФЭУ);
	\item электронно-оптические преобразователи (ЭОП)
\end{itemize}

К фотохимическим ПИ относят различные фоточувствительные фотографические материалы.

\section{Оптическое волокно}


\section{Модуляторы}

Модуляция --- изменение сигнала-носителя энергии в соответствии с передаваемой информацией.
Для ОЭП в большинстве случаев модуляция заключается в изменении одного из параметров (чаще всего амплитуды) потока излучения по заданному закону.
